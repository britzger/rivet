\documentclass[preprint,12pt]{elsarticle}

\input{preamble}
\usepackage{hyperref}

\begin{document}

\begin{frontmatter}
  
  \title{Rivet user manual}

  \author[a]{Andy Buckley}
  \author[b]{Jonathan Butterworth}
  \author[c]{David Grellscheid}
  \author[c]{Hendrik Hoeth}
  \author[d]{Leif L\"onnblad}
  \author[b]{James Monk}
  \author[e]{Holger Schulz}
  \author[f]{Frank Siegert\corref{author}}

  \address[a]{PPE Group, School of Physics, University of Edinburgh, UK.}
  \address[b]{HEP Group, Dept. of Physics and Astronomy, UCL, London, UK.}
  \address[c]{IPPP, Durham University, UK.}
  \address[d]{Theoretical Physics, Lund University, Sweden.}
  \address[e]{Institut f\"ur Physik, Berlin Humboldt University, Germany.}
  \address[f]{Physikalisches Institut, Freiburg University, Germany.}

  \cortext[author]{Corresponding author.\\\textit{E-mail address:} frank.siegert@cern.ch}

  \begin{abstract}
    This is the manual and user guide for the Rivet system for the
    validation and tuning of Monte Carlo event generators. As well as the core
    Rivet library, this manual describes the usage of the \kbd{rivet} program and
    the AGILe generator interface library. The depth and level of description is
    chosen for users of the system, starting with the basics of using validation
    code written by others, and then covering sufficient details to write new
    Rivet analyses and calculational components.
  \end{abstract}

  \begin{keyword}
    Event generator; simulation; validation; tuning; QCD
  \end{keyword}

\end{frontmatter}


{\bf PROGRAM SUMMARY}

\begin{small}
  \noindent
  {\em Manuscript Title: Rivet user manual}\\
  {\em Authors: Andy Buckley, Jonathan Butterworth, Hendrik Hoeth,
    Leif L\"onnblad, James Monk, Holger Schulz, Frank Siegert}\\
  {\em Program Title: Rivet}\\
  {\em Journal Reference:}                                      \\
  % Leave blank, supplied by Elsevier.
  {\em Catalogue identifier:}                                   \\
  % Leave blank, supplied by Elsevier.
  {\em Licensing provisions:}                                   \\
  % enter "none" if CPC non-profit use license is sufficient.
  {\em Programming language: C++, Python}\\
  {\em Computer: PC running Linux, Mac}\\
  % Computer(s) for which program has been designed.
  {\em Operating system: Linux, Mac OS}\\
  % Operating system(s) for which program has been designed.
  {\em RAM: 20M} bytes\\
  % RAM in bytes required to execute program with typical data.
  {\em Number of processors used: 1}                              \\
  % If more than one processor.
  {\em Supplementary material:}                                 \\
  % Fill in if necessary, otherwise leave out.
  {\em Keywords:} Event generator, simulation, validation, tuning, QCD  \\
  % Please give some freely chosen keywords that we can use in a
  % cumulative keyword index.
  {\em Classification: 11.9 Event Reconstruction and Data Analysis}\\
  % Classify using CPC Program Library Subject Index, see (
  % http://cpc.cs.qub.ac.uk/subjectIndex/SUBJECT_index.html)
  % e.g. 4.4 Feynman diagrams, 5 Computer Algebra.
  {\em External routines/libraries: HepMC, GSL, FastJet, Python, Swig, Boost, YAML}\\
  % Fill in if necessary, otherwise leave out.
  {\em Nature of problem:}\\
  Experimental measurements from high-energy particle colliders should be
  defined and stored in a general framework such that it is simple to compare
  theory predictions to them. Rivet is such a framework, and contains at the
  same time a large collection of existing measurements.
  \\
  {\em Solution method:}\\
  Rivet is based on HepMC events, a standardised output format provided by many
  theory simulation tools. Events are processed by Rivet to generate histograms
  for the requested list of analyses, incorporating all experimental phase
  space cuts and histogram definitions.
  \\
  {\em Restrictions:}\\
  Can not calculate statistical errors for correlated events as they appear
  in NLO calculations.
  \\
  {\em Unusual features:}\\
  It is possible for the user to implement and use their own custom analysis
  as a module without having to modify the main Rivet code/installation.
  \\
  {\em Additional comments:}\\
  % Provide any additional comments here.
  \\
  {\em Running time:}\\
  Depends on the number and complexity of analyses being applied, but typically
  a few hundred events per second.
  \\

\end{small}


\section{Introduction}
\label{sec:intro}
\input{intro}

\cleardoublepage

\part{Getting started with Rivet}
\label{part:gettingstarted}
\input{gettingstarted}

\cleardoublepage

\part{Selected analyses}
\label{part:selectedanalyses}
\input{selectedanalyses}

\cleardoublepage

\part{How Rivet works}
\label{part:writinganalyses}
\input{writinganalyses}

\section{Conclusions}
\label{sec:conclusions}
We have presented a users' guide for the Rivet event-generator validation
system. This manual is intended to be a
guide to using Rivet, rather than a comprehensive
reference to the application programming interface (API) of the Rivet
library. Rivet is a C++ class library, which provides the infrastructure and
calculational tools for simulation-level analyses for high energy collider
experiments, enabling physicists to
validate event generator models and tunings with minimal effort and maximum
portability. It is designed to scale effectively to large numbers of analyses
for truly global validation, by transparent use of an automated result caching
system.

In addition to an introduction to the philosophy behind the framework, we
have given examples on how to implement the user's own analysis module.
A selected list of available analyses has been given as an example of the
flexibility of the full framework.


\cleardoublepage

\part{Appendices}
\appendix

\section{Typical \kbd{agile-runmc} commands}
\label{app:agilerunmc}
\input{agilerunmc}

\section{Acknowledgements}
\label{app:acknowledgements}
\input{acknowledgements}


\cleardoublepage

\bibliographystyle{elsarticle-num}
\bibliography{refs,selectedanalyses}

\end{document}
