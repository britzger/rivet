\makeatletter
\renewcommand{\d}[1]{\ensuremath{\mathrm{#1}}}
\let\old@eta\eta
\renewcommand{\eta}{\ensuremath{\old@eta}\xspace}
\let\old@phi\phi
\renewcommand{\phi}{\ensuremath{\old@phi}\xspace}
\providecommand{\pT}{\ensuremath{p_\perp}\xspace}
\providecommand{\pTmin}{\ensuremath{p_\perp^\text{min}}\xspace}
\makeatother

\section{LEP analyses}\typeout{Handling analysis ALEPH_1991_S2435284}
\subsection[ALEPH\_1991\_S2435284]{ALEPH\_1991\_S2435284\,\cite{Decamp:1991uz}}
\textbf{Hadronic Z decay charged multiplicity measurement}\newline
\textbf{Experiment:} ALEPH (LEP 1) \newline
\textbf{Spires ID:} \href{http://www.slac.stanford.edu/spires/find/hep/www?rawcmd=key+2435284}{2435284}\newline
\textbf{Status:} \newline
\textbf{Authors:}
 \penalty 100
\begin{itemize}
  \item Andy Buckley $\langle\,$\href{mailto:andy.buckley@durham.ac.uk}{andy.buckley@durham.ac.uk}$\,\rangle$;
\end{itemize}
\textbf{References:}
 \penalty 100
\begin{itemize}
  \item Phys. Lett. B, 273, 181 (1991)
\end{itemize}
\textbf{Run details:}
 \penalty 100
\begin{itemize}

  \item Hadronic Z decay events generated on the Z pole (\ensuremath{\sqrt{s}} = 91.2 GeV)\end{itemize}

\noindent The charged particle multiplicity distribution of hadronic Z decays, as measured on the peak of the Z resonance using the ALEPH detector at LEP. The unfolding procedure was model independent, and the distribution was found to have a mean of $20.85 \pm 0.24$, Comparison with lower energy data supports the KNO scaling hypothesis. The shape of the multiplicity distribution is well described by a log-normal distribution, as predicted from a cascading model for multi-particle production.

\clearpage


\clearpage

\typeout{Handling analysis ALEPH_1996_S3196992}
\subsection[ALEPH\_1996\_S3196992]{ALEPH\_1996\_S3196992\,\cite{Buskulic:1995au}}
\textbf{Measurement of the quark to photon fragmentation function}\newline
\textbf{Experiment:} ALEPH (LEP Run 1) \newline
\textbf{Spires ID:} \href{http://www.slac.stanford.edu/spires/find/hep/www?rawcmd=key+3196992}{3196992}\newline
\textbf{Status:} \newline
\textbf{Authors:}
 \penalty 100
\begin{itemize}
  \item Frank Siegert $\langle\,$\href{mailto:frank.siegert@durham.ac.uk}{frank.siegert@durham.ac.uk}$\,\rangle$;
\end{itemize}
\textbf{References:}
 \penalty 100
\begin{itemize}
  \item Z.Phys.C69:365-378,1996
  \item DOI: \href{http://dx.doi.org/10.1007/s002880050037}{10.1007/s002880050037}
\end{itemize}
\textbf{Run details:}
 \penalty 100
\begin{itemize}

  \item $e^+e^-\to$ jets with $\pi$ and $\eta$ decays turned off.\end{itemize}

\noindent Earlier measurements at LEP of isolated hard photons in hadronic Z decays, attributed to radiation from primary quark pairs, have been extended in the ALEPH experiment to include hard photon production inside hadron jets. Events are selected where all particles combine democratically to form hadron jets, one of which contains a photon with a fractional energy $z > 0.7$. After  statistical subtraction of non-prompt photons, the quark-to-photon fragmentation function, $D(z)$, is extracted directly from the measured 2-jet rate.

\clearpage


\clearpage

\typeout{Handling analysis ALEPH_1996_S3486095}
\subsection[ALEPH\_1996\_S3486095]{ALEPH\_1996\_S3486095\,\cite{Barate:1996fi}}
\textbf{Studies of QCD with the ALEPH detector.}\newline
\textbf{Experiment:} ALEPH (LEP 1) \newline
\textbf{Spires ID:} \href{http://www.slac.stanford.edu/spires/find/hep/www?rawcmd=key+3486095}{3486095}\newline
\textbf{Status:} \newline
\textbf{Authors:}
 \penalty 100
\begin{itemize}
  \item Holger Schulz $\langle\,$\href{mailto:holger.schulz@physik.hu-berlin.de}{holger.schulz@physik.hu-berlin.de}$\,\rangle$;
\end{itemize}
\textbf{References:}
 \penalty 100
\begin{itemize}
  \item Phys. Rept., 294, 1--165 (1998)
\end{itemize}
\textbf{Run details:}
 \penalty 100
\begin{itemize}

  \item Hadronic Z decay events generated on the Z pole (\ensuremath{\sqrt{s}} = 91.2 GeV)\end{itemize}

\noindent Summary paper of QCD results as measured by ALEPH at LEP 1. The publication includes various event shape variables, multiplicities (identified particles and inclusive), and particle spectra.

\clearpage


\clearpage

\typeout{Handling analysis ALEPH_2004_S5765862}
\subsection[ALEPH\_2004\_S5765862]{ALEPH\_2004\_S5765862\,\cite{Heister:2003aj}}
\textbf{Jet rates and event shapes at LEP I and II}\newline
\textbf{Experiment:} ALEPH (LEP Run 1 and 2) \newline
\textbf{Spires ID:} \href{http://www.slac.stanford.edu/spires/find/hep/www?rawcmd=key+5765862}{5765862}\newline
\textbf{Status:} \newline
\textbf{Authors:}
 \penalty 100
\begin{itemize}
  \item Frank Siegert $\langle\,$\href{mailto:frank.siegert@durham.ac.uk}{frank.siegert@durham.ac.uk}$\,\rangle$;
\end{itemize}
\textbf{References:}
 \penalty 100
\begin{itemize}
  \item Eur.Phys.J.C35:457-486,2004
  \item DOI: \href{http://dx.doi.org/10.1140/epjc/s2004-01891-4}{10.1140/epjc/s2004-01891-4}
  \item http://cdsweb.cern.ch/record/690637/files/ep-2003-084.pdf
\end{itemize}
\textbf{Run details:}
 \penalty 100
\begin{itemize}

  \item $e^+ e^- \to$ jet jet (+ jets)\end{itemize}

\noindent Jet rates, event-shape variables and inclusive charged particle spectra are measured in $e^+ e^-$ collisions at CMS energies between 91 and 209 GeV. The previously published data at 91.2 GeV and 133 GeV have been re-processed and the higher energy data are presented here for the first time.

\clearpage


\clearpage

\typeout{Handling analysis DELPHI_1995_S3137023}
\subsection[DELPHI\_1995\_S3137023]{DELPHI\_1995\_S3137023\,\cite{Abreu:1995qx}}
\textbf{Strange baryon production in Z hadronic decays at Delphi}\newline
\textbf{Experiment:} DELPHI (LEP 1) \newline
\textbf{Spires ID:} \href{http://www.slac.stanford.edu/spires/find/hep/www?rawcmd=key+3137023}{3137023}\newline
\textbf{Status:} \newline
\textbf{Authors:}
 \penalty 100
\begin{itemize}
  \item Hendrik Hoeth $\langle\,$\href{mailto:hendrik.hoeth@cern.ch}{hendrik.hoeth@cern.ch}$\,\rangle$;
\end{itemize}
\textbf{References:}
 \penalty 100
\begin{itemize}
  \item Z. Phys. C, 67, 543--554 (1995)
\end{itemize}
\textbf{Run details:}
 \penalty 100
\begin{itemize}

  \item Hadronic Z decay events generated on the Z pole (\ensuremath{\sqrt{s}} = 91.2 GeV)\end{itemize}

\noindent Measurement of the $\Xi^-$ and $\Sigma^+(1385)/\Sigma^-(1385)$ scaled momentum distributions by DELPHI at LEP 1. The paper also has the production cross-sections of these particles, but that's not implemented in Rivet.

\clearpage


\clearpage

\typeout{Handling analysis DELPHI_1996_S3430090}
\subsection[DELPHI\_1996\_S3430090]{DELPHI\_1996\_S3430090\,\cite{Abreu:1996na}}
\textbf{Delphi MC tuning on event shapes and identified particles.}\newline
\textbf{Experiment:} DELPHI (LEP 1) \newline
\textbf{Spires ID:} \href{http://www.slac.stanford.edu/spires/find/hep/www?rawcmd=key+3430090}{3430090}\newline
\textbf{Status:} \newline
\textbf{Authors:}
 \penalty 100
\begin{itemize}
  \item Andy Buckley $\langle\,$\href{mailto:andy.buckley@durham.ac.uk}{andy.buckley@durham.ac.uk}$\,\rangle$;
  \item Hendrik Hoeth $\langle\,$\href{mailto:hendrik.hoeth@cern.ch}{hendrik.hoeth@cern.ch}$\,\rangle$;
\end{itemize}
\textbf{References:}
 \penalty 100
\begin{itemize}
  \item Z.Phys.C73:11-60,1996
  \item DOI: \href{http://dx.doi.org/10.1007/s002880050295}{10.1007/s002880050295}
\end{itemize}
\textbf{Run details:}
 \penalty 100
\begin{itemize}

  \item \ensuremath{\sqrt{s}} = 91.2 GeV, $e^+ e^- \ensuremath{\to} Z^0$ production with hadronic decays only\end{itemize}

\noindent Event shape and charged particle inclusive distributions measured using 750000 decays of Z bosons to hadrons from the DELPHI detector at LEP. This data, combined with identified particle distributions from all LEP experiments, was used for tuning of shower-hadronisation event generators by the original PROFESSOR method.  This is a critical analysis for MC event generator tuning of final state radiation and both flavour and kinematic aspects of hadronisation models.

\clearpage


\clearpage

\typeout{Handling analysis DELPHI_2002_069_CONF_603}
\subsection{DELPHI\_2002\_069\_CONF\_603}
\textbf{Study of the b-quark fragmentation function at LEP 1}\newline
\textbf{Experiment:} DELPHI (LEP 1) \newline
\textbf{Status:} \newline
\textbf{Authors:}
 \penalty 100
\begin{itemize}
  \item Hendrik Hoeth $\langle\,$\href{mailto:hendrik.hoeth@cern.ch}{hendrik.hoeth@cern.ch}$\,\rangle$;
\end{itemize}
\textbf{References:}
 \penalty 100
\begin{itemize}
  \item DELPHI note 2002-069-CONF-603 (ICHEP 2002)
\end{itemize}
\textbf{Run details:}
 \penalty 100
\begin{itemize}

  \item Hadronic Z decay events generated on the Z pole (\ensuremath{\sqrt{s}} = 91.2 GeV)\end{itemize}

\noindent Measurement of the b-quark fragmentation function by DELPHI using 1994 LEP 1 data. The fragmentation function for both weakly decaying and primary b-quarks has been determined in a model independent way. Nevertheless the authors trust $f(x_B^\text{weak})$ more than $f(x_B^\text{prim})$.

\clearpage


\clearpage

\typeout{Handling analysis DELPHI_2003_WUD_03_11}
\subsection{DELPHI\_2003\_WUD\_03\_11}
\textbf{4-jet angular distributions at LEP (note)}\newline
\textbf{Experiment:} DELPHI (LEP 1) \newline
\textbf{Status:} \newline
\textbf{Authors:}
 \penalty 100
\begin{itemize}
  \item Hendrik Hoeth $\langle\,$\href{mailto:hendrik.hoeth@cern.ch}{hendrik.hoeth@cern.ch}$\,\rangle$;
\end{itemize}
\textbf{References:}
 \penalty 100
\begin{itemize}
  \item Diploma thesis WUD-03-11, University of Wuppertal
\end{itemize}
\textbf{Run details:}
 \penalty 100
\begin{itemize}

  \item Hadronic Z decay events generated on the Z pole (\ensuremath{\sqrt{s}} = 91.2 GeV)\end{itemize}

\noindent The 4-jet angular distributions (Bengtsson-Zerwas, K\"orner-Schierholz- Willrodt, Nachtmann-Reiter, and $\alpha_{34}$) have been measured with DELPHI at LEP 1 using Jade and Durham cluster algorithms.

\clearpage


\clearpage

\typeout{Handling analysis JADE_OPAL_2000_S4300807}
\subsection[JADE\_OPAL\_2000\_S4300807]{JADE\_OPAL\_2000\_S4300807\,\cite{Pfeifenschneider:1999rz}}
\textbf{Jet rates in $e^+e^-$ at JADE [35--44 GeV] and OPAL [91--189 GeV].}\newline
\textbf{Experiment:} JADE_OPAL (PETRA and LEP) \newline
\textbf{Spires ID:} \href{http://www.slac.stanford.edu/spires/find/hep/www?rawcmd=key+4300807}{4300807}\newline
\textbf{Status:} \newline
\textbf{Authors:}
 \penalty 100
\begin{itemize}
  \item Frank Siegert $\langle\,$\href{mailto:frank.siegert@durham.ac.uk}{frank.siegert@durham.ac.uk}$\,\rangle$;
\end{itemize}
\textbf{References:}
 \penalty 100
\begin{itemize}
  \item Eur.Phys.J.C17:19-51,2000
  \item arXiv: \href{http://arxiv.org/abs/hep-ex/0001055}{hep-ex/0001055}
\end{itemize}
\textbf{Run details:}
 \penalty 100
\begin{itemize}

  \item $e^+ e^- \to$ jet jet (+ jets)\end{itemize}

\noindent Differential and integrated jet rates for Durham and JADE jet algorithms.

\clearpage


\clearpage

\typeout{Handling analysis OPAL_1998_S3780481}
\subsection[OPAL\_1998\_S3780481]{OPAL\_1998\_S3780481\,\cite{Ackerstaff:1998hz}}
\textbf{Measurements of flavor dependent fragmentation functions in $Z^0 \ensuremath{\to} q \bar{q}$ events}\newline
\textbf{Experiment:} OPAL (LEP 1) \newline
\textbf{Spires ID:} \href{http://www.slac.stanford.edu/spires/find/hep/www?rawcmd=key+3780481}{3780481}\newline
\textbf{Status:} \newline
\textbf{Authors:}
 \penalty 100
\begin{itemize}
  \item Hendrik Hoeth $\langle\,$\href{mailto:hendrik.hoeth@cern.ch}{hendrik.hoeth@cern.ch}$\,\rangle$;
\end{itemize}
\textbf{References:}
 \penalty 100
\begin{itemize}
  \item Eur. Phys. J, C7, 369--381 (1999)
  \item hep-ex/9807004
\end{itemize}
\textbf{Run details:}
 \penalty 100
\begin{itemize}

  \item Hadronic Z decay events generated on the Z pole (\ensuremath{\sqrt{s}} = 91.2 GeV)\end{itemize}

\noindent Measurement of scaled momentum distributions and total charged multiplicities in flavour tagged events at LEP 1. OPAL measured these observables in uds-, c-, and b-events separately. An inclusive measurement is also included.

\clearpage


\clearpage

\typeout{Handling analysis OPAL_2004_S6132243}
\subsection[OPAL\_2004\_S6132243]{OPAL\_2004\_S6132243\,\cite{Abbiendi:2004qz}}
\textbf{Event shape distributions and moments in $e^+ e^-$ \ensuremath{\to} hadrons at 91--209 GeV}\newline
\textbf{Experiment:} OPAL (LEP 1 \& 2) \newline
\textbf{Spires ID:} \href{http://www.slac.stanford.edu/spires/find/hep/www?rawcmd=key+6132243}{6132243}\newline
\textbf{Status:} \newline
\textbf{Authors:}
 \penalty 100
\begin{itemize}
  \item Andy Buckley $\langle\,$\href{mailto:andy.buckley@cern.ch}{andy.buckley@cern.ch}$\,\rangle$;
\end{itemize}
\textbf{References:}
 \penalty 100
\begin{itemize}
  \item Eur.Phys.J.C40:287-316,2005
  \item hep-ex/0503051
\end{itemize}
\textbf{Run details:}
 \penalty 100
\begin{itemize}

  \item Hadronic $e^+ e^-$ events at 4 representative energies (91, 133, 177, 197). Runs with \ensuremath{\sqrt{s}} above the Z mass need to have ISR suppressed, since the data has been corrected to remove radiative return to the Z.\end{itemize}

\noindent Measurement of $e^+ e^-$ event shape variable distributions and their 1st  to 5th moments in LEP running from the Z pole to the highest LEP 2 energy of 209 GeV.

\clearpage


\section{Tevatron analyses}\typeout{Handling analysis CDF_1988_S1865951}
\subsection[CDF\_1988\_S1865951]{CDF\_1988\_S1865951\,\cite{Abe:1988yu}}
\textbf{CDF transverse momentum distributions at 630 GeV and 1800 GeV.}\newline
\textbf{Experiment:} CDF (Tevatron Run I) \newline
\textbf{Spires ID:} \href{http://www.slac.stanford.edu/spires/find/hep/www?rawcmd=key+1865951}{1865951}\newline
\textbf{Status:} \newline
\textbf{Authors:}
 \penalty 100
\begin{itemize}
  \item Christophe Vaillant $\langle\,$\href{mailto:c.l.j.j.vaillant@durham.ac.uk}{c.l.j.j.vaillant@durham.ac.uk}$\,\rangle$;
  \item Andy Buckley $\langle\,$\href{mailto:andy.buckley@cern.ch}{andy.buckley@cern.ch}$\,\rangle$;
\end{itemize}
\textbf{References:}
 \penalty 100
\begin{itemize}
  \item Phys.Rev.Lett.61:1819,1988
  \item DOI: \href{http://dx.doi.org/10.1103/PhysRevLett.61.1819}{10.1103/PhysRevLett.61.1819}
\end{itemize}
\textbf{Run details:}
 \penalty 100
\begin{itemize}

  \item QCD min bias events at \ensuremath{\sqrt{s}} = 630 GeV and 1800 GeV, $|\eta| < 1.0$.\end{itemize}

\noindent Transverse momentum distributions at 630 GeV and 1800 GeV based on data from the CDF experiment at the Tevatron collider.

\clearpage


\clearpage

\typeout{Handling analysis CDF_1990_S2089246}
\subsection[CDF\_1990\_S2089246]{CDF\_1990\_S2089246\,\cite{Abe:1989td}}
\textbf{CDF pseudorapidity distributions at 630 and 1800 GeV}\newline
\textbf{Experiment:} CDF (Tevatron Run 0) \newline
\textbf{Spires ID:} \href{http://www.slac.stanford.edu/spires/find/hep/www?rawcmd=key+2089246}{2089246}\newline
\textbf{Status:} \newline
\textbf{Authors:}
 \penalty 100
\begin{itemize}
  \item Andy Buckley $\langle\,$\href{mailto:andy.buckley@cern.ch}{andy.buckley@cern.ch}$\,\rangle$;
\end{itemize}
\textbf{References:}
 \penalty 100
\begin{itemize}
  \item Phys.Rev.D41:2330,1990
  \item DOI: \href{http://dx.doi.org/10.1103/PhysRevD.41.2330}{10.1103/PhysRevD.41.2330}
\end{itemize}
\textbf{Run details:}
 \penalty 100
\begin{itemize}

  \item QCD min bias events at \ensuremath{\sqrt{s}} = 630 and 1800 GeV. Particles with $c \tau > 10$mm should be set stable.\end{itemize}

\noindent Pseudorapidity distributions based on the CDF 630 and 1800 GeV runs from 1987. All data is detector corrected. The data confirms the UA5 measurement of a $\d{N}/\d{\eta}$ rise with energy faster than $\ln{\sqrt{s}}$, and as such this analysis is important for constraining the energy evolution of minimum bias and underlying event characteristics in MC simulations.

\clearpage


\clearpage

\typeout{Handling analysis CDF_1991_S2313472}
\subsection[CDF\_1991\_S2313472]{CDF\_1991\_S2313472\,\cite{Abe:1991rk}}
\textbf{W-boson \pT measurement in $p\bar{p}$ collisions at $\sqrt{s}=1.8~\TeV$}\newline
\textbf{Experiment:} CDF (Tevatron) \newline
\textbf{Spires ID:} \href{http://www.slac.stanford.edu/spires/find/hep/www?rawcmd=key+2313472}{2313472}\newline
\textbf{Status:} \newline
\textbf{Authors:}
 \penalty 100
\begin{itemize}
  \item Holger Schulz $\langle\,$\href{mailto:hschulz@physik.hu-berlin.de}{hschulz@physik.hu-berlin.de}$\,\rangle$;
\end{itemize}
\textbf{References:}
 \penalty 100
\begin{itemize}
  \item Phys.Rev.Lett.66:2951-2955,1991
\end{itemize}
\textbf{Run details:}
 \penalty 100
\begin{itemize}

  \item QCD events with W+- production and electronic decays\end{itemize}

\noindent This is a CDF analysis from run 1, where the distribution of the transverse momentum of W candidates that decay  electronically, is measured. The  electron is required to be within $\left|\eta\right| < 1.1$, to have a transverse energy of $E_\perp > 20~\GeV$ and a $p_\perp > 12~\GeV$. The neutrino is required to produce a missing energy of  $E_{\perp, \text{ miss}}>20~\GeV$.

\clearpage


\clearpage

\typeout{Handling analysis CDF_1994_S2952106}
\subsection[CDF\_1994\_S2952106]{CDF\_1994\_S2952106\,\cite{Abe:1994nj}}
\textbf{CDF Run I color coherence analysis.}\newline
\textbf{Experiment:} CDF (Tevatron Run 1) \newline
\textbf{Spires ID:} \href{http://www.slac.stanford.edu/spires/find/hep/www?rawcmd=key+2952106}{2952106}\newline
\textbf{Status:} \newline
\textbf{Authors:}
 \penalty 100
\begin{itemize}
  \item Lars Sonnenschein $\langle\,$\href{mailto:Lars.Sonnenschein@cern.ch}{Lars.Sonnenschein@cern.ch}$\,\rangle$;
\end{itemize}
\textbf{References:}
 \penalty 100
\begin{itemize}
  \item Phys.Rev.D50,5562,1994
  \item DOI: \href{http://dx.doi.org/10.1103/PhysRevD.50.5562}{10.1103/PhysRevD.50.5562}
\end{itemize}
\textbf{Run details:}
 \penalty 100
\begin{itemize}

  \item QCD events at \ensuremath{\sqrt{s}} = 1800 GeV. Leading jet \pTmin = 100 GeV.\end{itemize}

\noindent CDF Run I color coherence analysis. Events with $\ge 3$ jets are selected and Et distributions of the three highest-\pT jets are obtained. The plotted quantities are the $\Delta{R}$ between the 2nd and 3rd leading jets in the \pT and pseudorapidity of the 3rd jet, and $\alpha = \mathrm{d}{\eta}/\mathrm{d}{\phi}$, where $\mathrm{d}{\eta}$ is the pseudorapidity difference between the 2nd and 3rd jets and $\mathrm{d}{\phi}$ is their azimuthal angle difference.  Since the data has not been detector-corrected, a bin by bin correction is applied, based on the distributions with ideal and CDF simulation as given in the publication.

\clearpage


\clearpage

\typeout{Handling analysis CDF_1996_S3108457}
\subsection[CDF\_1996\_S3108457]{CDF\_1996\_S3108457\,\cite{Abe:1995rw}}
\textbf{Properties of High-Mass Multijet Events}\newline
\textbf{Experiment:} CDF (Tevatron Run 1) \newline
\textbf{Spires ID:} \href{http://www.slac.stanford.edu/spires/find/hep/www?rawcmd=key+3108457}{3108457}\newline
\textbf{Status:} \newline
\textbf{Authors:}
 \penalty 100
\begin{itemize}
  \item Frank Siegert $\langle\,$\href{mailto:frank.siegert@durham.ac.uk}{frank.siegert@durham.ac.uk}$\,\rangle$;
\end{itemize}
\textbf{References:}
 \penalty 100
\begin{itemize}
  \item Phys.Rev.Lett.75:608-612,1995
  \item DOI: \href{http://dx.doi.org/10.1103/PhysRevLett.75.608}{10.1103/PhysRevLett.75.608}
\end{itemize}
\textbf{Run details:}
 \penalty 100
\begin{itemize}

  \item Pure QCD events without underlying event.\end{itemize}

\noindent Properties of two-, three-, four-, five-, and six-jet events... Multijet-mass, leading jet angle, jet \pT.

\clearpage


\clearpage

\typeout{Handling analysis CDF_1996_S3349578}
\subsection[CDF\_1996\_S3349578]{CDF\_1996\_S3349578\,\cite{Abe:1996nn}}
\textbf{Further properties of high-mass multijet events}\newline
\textbf{Experiment:} CDF (Tevatron Run 1) \newline
\textbf{Spires ID:} \href{http://www.slac.stanford.edu/spires/find/hep/www?rawcmd=key+3349578}{3349578}\newline
\textbf{Status:} \newline
\textbf{Authors:}
 \penalty 100
\begin{itemize}
  \item Frank Siegert $\langle\,$\href{mailto:frank.siegert@durham.ac.uk}{frank.siegert@durham.ac.uk}$\,\rangle$;
\end{itemize}
\textbf{References:}
 \penalty 100
\begin{itemize}
  \item Phys.Rev.D54:4221-4233,1996
  \item DOI: \href{http://dx.doi.org/10.1103/PhysRevD.54.4221}{10.1103/PhysRevD.54.4221}
  \item arXiv: \href{http://arxiv.org/abs/hep-ex/9605004}{hep-ex/9605004}
\end{itemize}
\textbf{Run details:}
 \penalty 100
\begin{itemize}

  \item Pure QCD events without underlying event.\end{itemize}

\noindent Multijet distributions corresponding to (4N-4) variables that span the N-body parameter space in inclusive N=3, 4, 5 jet events.

\clearpage


\clearpage

\typeout{Handling analysis CDF_1996_S3418421}
\subsection[CDF\_1996\_S3418421]{CDF\_1996\_S3418421\,\cite{Abe:1996mj}}
\textbf{Dijet angular distributions}\newline
\textbf{Experiment:} CDF (Tevatron Run 1) \newline
\textbf{Spires ID:} \href{http://www.slac.stanford.edu/spires/find/hep/www?rawcmd=key+3418421}{3418421}\newline
\textbf{Status:} \newline
\textbf{Authors:}
 \penalty 100
\begin{itemize}
  \item Frank Siegert $\langle\,$\href{mailto:frank.siegert@durham.ac.uk}{frank.siegert@durham.ac.uk}$\,\rangle$;
\end{itemize}
\textbf{References:}
 \penalty 100
\begin{itemize}
  \item Phys.Rev.Lett.77:5336-5341,1996
  \item DOI: \href{http://dx.doi.org/10.1103/PhysRevLett.77.5336}{10.1103/PhysRevLett.77.5336}
  \item arXiv: \href{http://arxiv.org/abs/hep-ex/9609011}{hep-ex/9609011}
\end{itemize}
\textbf{Run details:}
 \penalty 100
\begin{itemize}

  \item QCD dijet events at Tevatron $\sqrt{s}=1.8$ TeV without MPI.\end{itemize}

\noindent Measurement of jet angular distributions in events with two jets in the final state in 5 bins of dijet invariant mass. Based on $106 \mathrm{pb}^{-1}$

\clearpage


\clearpage

\typeout{Handling analysis CDF_1997_S3541940}
\subsection[CDF\_1997\_S3541940]{CDF\_1997\_S3541940\,\cite{Abe:1997yb}}
\textbf{Properties of six jet events with large six jet mass}\newline
\textbf{Experiment:} CDF (Tevatron Run 1) \newline
\textbf{Spires ID:} \href{http://www.slac.stanford.edu/spires/find/hep/www?rawcmd=key+3541940}{3541940}\newline
\textbf{Status:} \newline
\textbf{Authors:}
 \penalty 100
\begin{itemize}
  \item Frank Siegert $\langle\,$\href{mailto:frank.siegert@durham.ac.uk}{frank.siegert@durham.ac.uk}$\,\rangle$;
\end{itemize}
\textbf{References:}
 \penalty 100
\begin{itemize}
  \item Phys.Rev.D56:2532-2543,1997
  \item DOI: \href{http://dx.doi.org/10.1103/PhysRevD.56.2532}{10.1103/PhysRevD.56.2532}
  \item http://lss.fnal.gov/archive/1997/pub/Pub-97-093-E.pdf
\end{itemize}
\textbf{Run details:}
 \penalty 100
\begin{itemize}

  \item Pure QCD events without underlying event.\end{itemize}

\noindent Multijet distributions corresponding to 20 variables that span the 6-body parameter space in inclusive 6-jet events.

\clearpage


\clearpage

\typeout{Handling analysis CDF_1998_S3618439}
\subsection[CDF\_1998\_S3618439]{CDF\_1998\_S3618439\,\cite{Abe:1997eua}}
\textbf{Differential cross-section for events with large total transverse energy}\newline
\textbf{Experiment:} CDF (Tevatron Run 1) \newline
\textbf{Spires ID:} \href{http://www.slac.stanford.edu/spires/find/hep/www?rawcmd=key+3618439}{3618439}\newline
\textbf{Status:} \newline
\textbf{Authors:}
 \penalty 100
\begin{itemize}
  \item Frank Siegert $\langle\,$\href{mailto:frank.siegert@durham.ac.uk}{frank.siegert@durham.ac.uk}$\,\rangle$;
\end{itemize}
\textbf{References:}
 \penalty 100
\begin{itemize}
  \item Phys.Rev.Lett.80:3461-3466,1998
  \item 10.1103/PhysRevLett.80.3461
\end{itemize}
\textbf{Run details:}
 \penalty 100
\begin{itemize}

  \item QCD events at Tevatron with $\sqrt{s}=1.8$ TeV without MPI.\end{itemize}

\noindent Measurement of the differential cross section  $\mathrm{d}\sigma/\mathrm{d}E_\perp^j$ for the production of multijet events in $p\bar{p}$ collisions where the  sum is over all jets with transverse energy  $E_\perp^j > E_\perp^\mathrm{min}$.

\clearpage


\clearpage

\typeout{Handling analysis CDF_2000_S4155203}
\subsection[CDF\_2000\_S4155203]{CDF\_2000\_S4155203\,\cite{Affolder:1999jh}}
\textbf{Z \pT measurement in CDF Z \ensuremath{\to} $e^+e^-$ events}\newline
\textbf{Experiment:} CDF (Tevatron Run 1) \newline
\textbf{Spires ID:} \href{http://www.slac.stanford.edu/spires/find/hep/www?rawcmd=key+4155203}{4155203}\newline
\textbf{Status:} \newline
\textbf{Authors:}
 \penalty 100
\begin{itemize}
  \item Hendrik Hoeth $\langle\,$\href{mailto:hendrik.hoeth@cern.ch}{hendrik.hoeth@cern.ch}$\,\rangle$;
\end{itemize}
\textbf{References:}
 \penalty 100
\begin{itemize}
  \item Phys.Rev.Lett.84:845-850,2000
  \item arXiv: \href{http://arxiv.org/abs/hep-ex/0001021}{hep-ex/0001021}
  \item DOI: \href{http://dx.doi.org/10.1103/PhysRevLett.84.845}{10.1103/PhysRevLett.84.845}
\end{itemize}
\textbf{Run details:}
 \penalty 100
\begin{itemize}

  \item $p\bar{p}$ collisions at 1800 GeV. $Z/\gamma^*$ Drell-Yan events with $e^+e^-$ decay mode only.\end{itemize}

\noindent Measurement of transverse momentum and total cross section of $e^+e^-$ pairs in the Z-boson region of $66~\text{GeV}/c^2 < m_{ee} < 116~\text{GeV}/c^2$ from pbar-p collisions at \ensuremath{\sqrt{s}} = 1.8 TeV, with the Tevatron CDF detector.  The Z \pT, in a fully-factorised picture, is generated by the momentum balance against initial state radiation (ISR) and the primordial/intrinsic \pT of the Z's parent partons in the incoming hadrons. The Z \pT is important in generator tuning to fix the interplay of ISR and multi-parton interactions (MPI) ingenerating `underlying event' activity. 
This analysis is subject to ambiguities in the experimental Z \pT definition, since the Rivet implementation reconstructs the Z momentum from the dilepton pair with finite cones for QED bremstrahlung summation, rather than non-portable direct use of the (sometimes absent) Z in the event record.

\clearpage


\clearpage

\typeout{Handling analysis CDF_2000_S4266730}
\subsection{CDF\_2000\_S4266730}
\textbf{Differential Dijet Mass Cross Section}\newline
\textbf{Experiment:} CDF (Tevatron Run 1) \newline
\textbf{Spires ID:} \href{http://www.slac.stanford.edu/spires/find/hep/www?rawcmd=key+4266730}{4266730}\newline
\textbf{Status:} \newline
\textbf{Authors:}
 \penalty 100
\begin{itemize}
  \item Frank Siegert $\langle\,$\href{mailto:frank.siegert@durham.ac.uk}{frank.siegert@durham.ac.uk}$\,\rangle$;
\end{itemize}
\textbf{References:}
 \penalty 100
\begin{itemize}
  \item Phys.Rev.D61:091101,2000
  \item DOI: \href{http://dx.doi.org/10.1103/PhysRevD.61.091101}{10.1103/PhysRevD.61.091101}
  \item arXiv: \href{http://arxiv.org/abs/hep-ex/9912022}{hep-ex/9912022}
\end{itemize}
\textbf{Run details:}
 \penalty 100
\begin{itemize}

  \item Dijet events at Tevatron with $\sqrt{s}=1.8$ TeV\end{itemize}

\noindent Measurement of the cross section for production of two or more jets as a function of dijet mass in the range 180 to 1000 GeV. It is based on an integrated luminosity of $86 \mathrm{pb}^{-1}$.

\clearpage


\clearpage

\typeout{Handling analysis CDF_2001_S4517016}
\subsection[CDF\_2001\_S4517016]{CDF\_2001\_S4517016\,\cite{Affolder:2000ew}}
\textbf{Two jet triply-differential cross-section}\newline
\textbf{Experiment:} CDF (Tevatron Run 1) \newline
\textbf{Spires ID:} \href{http://www.slac.stanford.edu/spires/find/hep/www?rawcmd=key+4517016}{4517016}\newline
\textbf{Status:} \newline
\textbf{Authors:}
 \penalty 100
\begin{itemize}
  \item Frank Siegert $\langle\,$\href{mailto:frank.siegert@durham.ac.uk}{frank.siegert@durham.ac.uk}$\,\rangle$;
\end{itemize}
\textbf{References:}
 \penalty 100
\begin{itemize}
  \item Phys.Rev.D64:012001,2001
  \item DOI: \href{http://dx.doi.org/10.1103/PhysRevD.64.012001}{10.1103/PhysRevD.64.012001}
  \item arXiv: \href{http://arxiv.org/abs/hep-ex/0012013}{hep-ex/0012013}
\end{itemize}
\textbf{Run details:}
 \penalty 100
\begin{itemize}

  \item Dijet events at Tevatron with $\sqrt{s}=1.8$ TeV\end{itemize}

\noindent A measurement of the two-jet differential cross section,  $\mathrm{d}^3\sigma/\mathrm{d}E_T \, \mathrm{d}\eta_1 \, \mathrm{d}\eta_2$, based on an integrated luminosity of $86 \mathrm{pb}^{-1}$. The differential cross section is measured as a function of the transverse energy, $E_\perp$, of a jet in the pseudorapidity region $0.1 < |\eta_1| < 0.7$ for four different pseudorapidity bins of a second jet restricted to $0.1 < |\eta_2| < 3.0$.

\clearpage


\clearpage

\typeout{Handling analysis CDF_2001_S4563131}
\subsection[CDF\_2001\_S4563131]{CDF\_2001\_S4563131\,\cite{Affolder:2001fa}}
\textbf{Inclusive jet cross section}\newline
\textbf{Experiment:} CDF (Tevatron Run 1) \newline
\textbf{Spires ID:} \href{http://www.slac.stanford.edu/spires/find/hep/www?rawcmd=key+4563131}{4563131}\newline
\textbf{Status:} \newline
\textbf{Authors:}
 \penalty 100
\begin{itemize}
  \item Frank Siegert $\langle\,$\href{mailto:frank.siegert@durham.ac.uk}{frank.siegert@durham.ac.uk}$\,\rangle$;
\end{itemize}
\textbf{References:}
 \penalty 100
\begin{itemize}
  \item Phys.Rev.D64:032001,2001
  \item DOI: \href{http://dx.doi.org/10.1103/PhysRevD.64.032001}{10.1103/PhysRevD.64.032001}
  \item arXiv: \href{http://arxiv.org/abs/hep-ph/0102074}{hep-ph/0102074}
\end{itemize}
\textbf{Run details:}
 \penalty 100
\begin{itemize}

  \item Dijet events at Tevatron with $\sqrt{s}=1.8$ TeV\end{itemize}

\noindent Measurement of the inclusive jet cross section for jet transverse energies from 40 to 465 GeV in the pseudo-rapidity range $0.1<|\eta|<0.7$. The results are based on 87 $\mathrm{pb}^{-1}$ of data.

\clearpage


\clearpage

\typeout{Handling analysis CDF_2001_S4751469}
\subsection[CDF\_2001\_S4751469]{CDF\_2001\_S4751469\,\cite{Affolder:2001xt}}
\textbf{Field \& Stuart Run I underlying event analysis.}\newline
\textbf{Experiment:} CDF (Tevatron Run 1) \newline
\textbf{Spires ID:} \href{http://www.slac.stanford.edu/spires/find/hep/www?rawcmd=key+4751469}{4751469}\newline
\textbf{Status:} \newline
\textbf{Authors:}
 \penalty 100
\begin{itemize}
  \item Andy Buckley $\langle\,$\href{mailto:andy.buckley@durham.ac.uk}{andy.buckley@durham.ac.uk}$\,\rangle$;
\end{itemize}
\textbf{References:}
 \penalty 100
\begin{itemize}
  \item Phys.Rev.D65:092002,2002
  \item FNAL-PUB 01/211-E
\end{itemize}
\textbf{Run details:}
 \penalty 100
\begin{itemize}

  \item $p\bar{p}$ QCD interactions at 1800 GeV. The leading jet is binned from 0--49 GeV, and histos can usually can be filled with a single generator run without kinematic sub-samples.\end{itemize}

\noindent The original CDF underlying event analysis, based on decomposing each event into a transverse structure with ``toward'', ``away'' and ``transverse'' regions defined relative to the azimuthal direction of the leading jet in the event. Since the toward region is by definition dominated by the hard process, as is the away region by momentum balance in the matrix element, the transverse region is most sensitive to multi-parton interactions. The transverse regions occupy $|\phi| \in [60\degree, 120\degree]$ for $|\eta| < 1$. The \pT ranges for the leading jet are divided experimentally into the `min-bias' sample from 0--20 GeV, and the `JET20' sample from 18--49 GeV.

\clearpage


\clearpage

\typeout{Handling analysis CDF_2002_S4796047}
\subsection[CDF\_2002\_S4796047]{CDF\_2002\_S4796047\,\cite{Acosta:2001rm}}
\textbf{CDF Run 1 charged multiplicity measurement}\newline
\textbf{Experiment:} CDF (Tevatron Run 1) \newline
\textbf{Spires ID:} \href{http://www.slac.stanford.edu/spires/find/hep/www?rawcmd=key+4796047}{4796047}\newline
\textbf{Status:} \newline
\textbf{Authors:}
 \penalty 100
\begin{itemize}
  \item Hendrik Hoeth $\langle\,$\href{mailto:hendrik.hoeth@cern.ch}{hendrik.hoeth@cern.ch}$\,\rangle$;
\end{itemize}
\textbf{References:}
 \penalty 100
\begin{itemize}
  \item Phys.Rev.D65:072005,2002
  \item DOI: \href{http://dx.doi.org/10.1103/PhysRevD.65.072005}{10.1103/PhysRevD.65.072005}
\end{itemize}
\textbf{Run details:}
 \penalty 100
\begin{itemize}

  \item QCD events at \ensuremath{\sqrt{s}} = 630 and 1800 GeV.\end{itemize}

\noindent A study of $p\bar{p}$ collisions at \ensuremath{\sqrt{s}} = 1800 and 630 GeV collected using a minimum bias trigger in which the data set is divided into two classes corresponding to `soft' and `hard' interactions. For each subsample, the analysis includes measurements of the multiplicity, transverse momentum (\pT) spectra, and the average \pT and event-by-event \pT dispersion as a function of multiplicity. A comparison of results shows distinct differences in the behavior of the two samples as a function of the center of mass energy. The properties of the soft sample are invariant as a function of c.m. energy.

\clearpage


\clearpage

\typeout{Handling analysis CDF_2004_S5839831}
\subsection[CDF\_2004\_S5839831]{CDF\_2004\_S5839831\,\cite{Acosta:2004wqa}}
\textbf{Transverse cone and `Swiss cheese' underlying event studies}\newline
\textbf{Experiment:} CDF (Tevatron Run 2) \newline
\textbf{Spires ID:} \href{http://www.slac.stanford.edu/spires/find/hep/www?rawcmd=key+5839831}{5839831}\newline
\textbf{Status:} \newline
\textbf{Authors:}
 \penalty 100
\begin{itemize}
  \item Andy Buckley $\langle\,$\href{mailto:andy.buckley@durham.ac.uk}{andy.buckley@durham.ac.uk}$\,\rangle$;
\end{itemize}
\textbf{References:}
 \penalty 100
\begin{itemize}
  \item Phys. Rev. D70, 072002 (2004)
  \item arXiv: \href{http://arxiv.org/abs/hep-ex/0404004}{hep-ex/0404004}
\end{itemize}
\textbf{Run details:}
 \penalty 100
\begin{itemize}

  \item QCD events at \ensuremath{\sqrt{s}} = 630 \& 1800 GeV. Several \pTmin cutoffs are probably required to fill the profile histograms, e.g. 0 (min bias), 30, 90, 150 GeV at 1800 GeV, and 0 (min bias), 20, 90,  150 GeV at 630 GeV.\end{itemize}

\noindent This analysis studies the underlying event via transverse cones of  $R = 0.7$ at 90 degrees in \phi relative to the leading (highest E) jet, at \ensuremath{\sqrt{s}} = 630 and 1800 GeV. This is similar to the 2001 CDF UE analysis, except that cones, rather than the whole central \eta range are used. The transverse cones are categorised as TransMIN and TransMAX on an event-by-event basis, to give greater sensitivity to the UE component.
`Swiss Cheese' distributions, where cones around the leading $n$ jets are excluded from the distributions, are also included for $n = 2, 3$.  This analysis is useful for constraining the energy evolution of the underlying event, since it performs the same analyses at two distinct CoM energies.
WARNING! The \pT plots are normalised to raw number of events. The min bias data have not been reproduced by MC, and are not recommended for tuning.

\clearpage


\clearpage

\typeout{Handling analysis CDF_2005_S6080774}
\subsection{CDF\_2005\_S6080774}
\textbf{Differential cross sections for prompt diphoton production}\newline
\textbf{Experiment:} CDF (Tevatron Run 2) \newline
\textbf{Spires ID:} \href{http://www.slac.stanford.edu/spires/find/hep/www?rawcmd=key+6080774}{6080774}\newline
\textbf{Status:} \newline
\textbf{Authors:}
 \penalty 100
\begin{itemize}
  \item Frank Siegert $\langle\,$\href{mailto:frank.siegert@durham.ac.uk}{frank.siegert@durham.ac.uk}$\,\rangle$;
\end{itemize}
\textbf{References:}
 \penalty 100
\begin{itemize}
  \item Phys. Rev. Lett. 95, 022003
  \item DOI: \href{http://dx.doi.org/10.1103/PhysRevLett.95.022003}{10.1103/PhysRevLett.95.022003}
  \item arXiv: \href{http://arxiv.org/abs/hep-ex/0412050}{hep-ex/0412050}
\end{itemize}
\textbf{Run details:}
 \penalty 100
\begin{itemize}

  \item $p \bar{p} \to \gamma \gamma$ [+ jets] at 1960 GeV. The analysis uses photons with \pT larger then 13 GeV. To allow for shifts in the shower, the ME cut on the transverse photon momentum shouldn't be too hard, e.g. 5 GeV.\end{itemize}

\noindent Measurement of the cross section of prompt diphoton production in $p\bar{p}$ collisions at $\sqrt{s} = 1.96$ TeV using a data sample of 207~pb$^{-1}$ as a function of the diphoton mass, the transverse momentum of the diphoton system, and the azimuthal angle between the two photons.

\clearpage


\clearpage

\typeout{Handling analysis CDF_2005_S6217184}
\subsection{CDF\_2005\_S6217184}
\textbf{CDF Run II jet shape analysis}\newline
\textbf{Experiment:} CDF (Tevatron Run 2) \newline
\textbf{Spires ID:} \href{http://www.slac.stanford.edu/spires/find/hep/www?rawcmd=key+6217184}{6217184}\newline
\textbf{Status:} \newline
\textbf{Authors:}
 \penalty 100
\begin{itemize}
  \item Lars Sonnenschein $\langle\,$\href{mailto:Lars.Sonnenschein@cern.ch}{Lars.Sonnenschein@cern.ch}$\,\rangle$;
  \item Andy Buckley $\langle\,$\href{mailto:andy.buckley@cern.ch}{andy.buckley@cern.ch}$\,\rangle$;
\end{itemize}
\textbf{References:}
 \penalty 100
\begin{itemize}
  \item Phys.Rev.D71:112002,2005
  \item DOI: \href{http://dx.doi.org/10.1103/PhysRevD.71.112002}{10.1103/PhysRevD.71.112002}
  \item arXiv: \href{http://arxiv.org/abs/hep-ex/0505013}{hep-ex/0505013}
\end{itemize}
\textbf{Run details:}
 \penalty 100
\begin{itemize}

  \item QCD events at \ensuremath{\sqrt{s}} = 1960 GeV. Jet \pTmin in plots is 37 GeV/c --- choose generator min \pT somewhere  well below this.\end{itemize}

\noindent Measurement of jet shapes in inclusive jet production in p pbar collisions at center-of-mass energy \ensuremath{\sqrt{s}} = 1.96 TeV. The data cover jet transverse momenta from 37--380 GeV and absolute jet rapidities in the range 0.1--0.7.

\clearpage


\clearpage

\typeout{Handling analysis CDF_2006_S6450792}
\subsection[CDF\_2006\_S6450792]{CDF\_2006\_S6450792\,\cite{Abulencia:2005yg}}
\textbf{Inclusive jet cross section differential in \pT}\newline
\textbf{Experiment:} CDF (Tevatron Run 2) \newline
\textbf{Spires ID:} \href{http://www.slac.stanford.edu/spires/find/hep/www?rawcmd=key+6450792}{6450792}\newline
\textbf{Status:} \newline
\textbf{Authors:}
 \penalty 100
\begin{itemize}
  \item Frank Siegert $\langle\,$\href{mailto:frank.siegert@durham.ac.uk}{frank.siegert@durham.ac.uk}$\,\rangle$;
\end{itemize}
\textbf{References:}
 \penalty 100
\begin{itemize}
  \item Phys.Rev.D74:071103,2006
  \item DOI: \href{http://dx.doi.org/10.1103/PhysRevD.74.071103}{10.1103/PhysRevD.74.071103}
  \item arXiv: \href{http://arxiv.org/abs/hep-ex/0512020}{hep-ex/0512020}
\end{itemize}
\textbf{Run details:}
 \penalty 100
\begin{itemize}

  \item $p\bar{p}$ \ensuremath{\to} jets at 1960 GeV\end{itemize}

\noindent Measurement of the inclusive jet cross section in ppbar interactions at $\sqrt{s}=1.96$ TeV using 385 $\mathrm{pb}^{-1}$ of data. The data cover the jet transverse momentum range from 61 to 620 GeV/c in $0.1 < |y| < 0.7$. This analysis has been updated with more data in more rapidity bins in CDF_2008_S7828950.

\clearpage


\clearpage

\typeout{Handling analysis CDF_2006_S6653332}
\subsection[CDF\_2006\_S6653332]{CDF\_2006\_S6653332\,\cite{Abulencia:2006ce}}
\textbf{\pT and eta distributions of jets in Z + jet production}\newline
\textbf{Experiment:} CDF (Tevatron Run 2) \newline
\textbf{Spires ID:} \href{http://www.slac.stanford.edu/spires/find/hep/www?rawcmd=key+6653332}{6653332}\newline
\textbf{Status:} \newline
\textbf{Authors:}
 \penalty 100
\begin{itemize}
  \item Lars Sonnenschein $\langle\,$\href{mailto:Lars.Sonnenschein@cern.ch}{Lars.Sonnenschein@cern.ch}$\,\rangle$;
\end{itemize}
\textbf{References:}
 \penalty 100
\begin{itemize}
  \item Phys.Rev.D.74:032008,2006
  \item DOI: \href{http://dx.doi.org/10.1103/PhysRevD.74.032008}{10.1103/PhysRevD.74.032008}
  \item arXiv: \href{http://arxiv.org/abs/hep-ex/0605099v2}{hep-ex/0605099v2}
\end{itemize}
\textbf{Run details:}
 \penalty 100
\begin{itemize}

  \item Z + jets events at \ensuremath{\sqrt{s}} = 1960 GeV. Jets min \pT cut = 20~GeV, leptons min \pT cut = 10~GeV\end{itemize}

\noindent Measurement of the b jet cross section in events with Z boson in p pbar collisions at center-of-mass energy \ensuremath{\sqrt{s}} = 1.96 TeV. The data cover jet transverse momenta above 20 GeV and jet pseudorapidities in the range -1.5 to 1.5. Z bosons are identified in their electron and muon decay modes in an invariant dilepton mass range between 66 and 116 GeV.

\clearpage


\clearpage

\typeout{Handling analysis CDF_2007_S7057202}
\subsection[CDF\_2007\_S7057202]{CDF\_2007\_S7057202\,\cite{Abulencia:2007ez}}
\textbf{CDF Run II inclusive jet cross-section using the kT algorithm}\newline
\textbf{Experiment:} CDF (Tevatron Run 2) \newline
\textbf{Spires ID:} \href{http://www.slac.stanford.edu/spires/find/hep/www?rawcmd=key+7057202}{7057202}\newline
\textbf{Status:} \newline
\textbf{Authors:}
 \penalty 100
\begin{itemize}
  \item David Voong
  \item Frank Siegert $\langle\,$\href{mailto:frank.siegert@durham.ac.uk}{frank.siegert@durham.ac.uk}$\,\rangle$;
\end{itemize}
\textbf{References:}
 \penalty 100
\begin{itemize}
  \item Phys.Rev.D75:092006,2007
  \item Erratum-ibid.D75:119901,2007
  \item FNAL-PUB 07/026-E
  \item hep-ex/0701051
\end{itemize}
\textbf{Run details:}
 \penalty 100
\begin{itemize}

  \item p-pbar collisions at 1960~GeV. Jet \pT bins from 54~GeV to 700~GeV.  Jet rapidity $< |2.1|$.\end{itemize}

\noindent CDF Run II measurement of inclusive jet cross sections at a p-pbar collision energy of 1.96~TeV. Jets are reconstructed in the central part of the detector ($|y|<2.1$) using the kT algorithm with an $R$ parameter of 0.7. The minimum jet \pT considered is 54~GeV, with a maximum around 700~GeV.  The inclusive jet \pT is plotted in bins of rapidity $|y|<0.1$, $0.1<|y|<0.7$, $0.7<|y|<1.1$, $1.1<|y|<1.6$ and $1.6<|y|<2.1$.

\clearpage


\clearpage

\typeout{Handling analysis CDF_2008_LEADINGJETS}
\subsection{CDF\_2008\_LEADINGJETS}
\textbf{CDF Run 2 underlying event in leading jet events}\newline
\textbf{Experiment:} CDF (Tevatron Run 2) \newline
\textbf{Spires ID:} \href{http://www.slac.stanford.edu/spires/find/hep/www?rawcmd=key+NONE}{NONE}\newline
\textbf{Status:} \newline
\textbf{Authors:}
 \penalty 100
\begin{itemize}
  \item Hendrik Hoeth $\langle\,$\href{mailto:hendrik.hoeth@cern.ch}{hendrik.hoeth@cern.ch}$\,\rangle$;
\end{itemize}
\textbf{No references listed}\\ 
\textbf{Run details:}
 \penalty 100
\begin{itemize}

  \item $p\bar{p}$ QCD interactions at 1960~GeV. Particles with  $c \tau > {}$10 mm should be set stable. Several $p_\perp^\text{min}$  cutoffs are probably required to fill the profile histograms. $p_\perp^\text{min} = {}$ 0 (min bias), 10, 20, 50, 100, 150 GeV. The corresponding merging points are at $p_T = $ 0, 30, 50, 80,  130, 180 GeV\end{itemize}

\noindent Rick Field's measurement of the underlying event in leading jet events. If the leading jet of the event is within $|\eta| < 2$, the event is accepted and ``toward'', ``away'' and ``transverse'' regions are defined in the same way as in the original (2001) CDF underlying event analysis. The leading jet defines the $\phi$ direction of the toward region. The transverse regions are most sensitive to the underlying event.

\clearpage


\clearpage

\typeout{Handling analysis CDF_2008_NOTE_9351}
\subsection{CDF\_2008\_NOTE\_9351}
\textbf{CDF Run 2 underlying event in Drell-Yan}\newline
\textbf{Experiment:} CDF (Tevatron Run 2) \newline
\textbf{Spires ID:} \href{http://www.slac.stanford.edu/spires/find/hep/www?rawcmd=key+NONE}{NONE}\newline
\textbf{Status:} \newline
\textbf{Authors:}
 \penalty 100
\begin{itemize}
  \item Hendrik Hoeth $\langle\,$\href{mailto:hendrik.hoeth@cern.ch}{hendrik.hoeth@cern.ch}$\,\rangle$;
\end{itemize}
\textbf{References:}
 \penalty 100
\begin{itemize}
  \item CDF public note 9351
\end{itemize}
\textbf{Run details:}
 \penalty 100
\begin{itemize}

  \item ppbar collisions at 1960 GeV.
  \item Drell-Yan events with $Z/\gamma* \ensuremath{\to} e e$ and $Z/\gamma* \ensuremath{\to} \mu\mu$.
  \item A mass cut $m_{ll} > 70~\text{GeV}$ can be applied on generator level.
  \item Particles with $c \tau > 10~\text{mm}$ should be set stable.\end{itemize}

\noindent Deepak Kar and Rick Field's measurement of the underlying event in Drell-Yan events. $Z \ensuremath{\to} ee$ and $Z \ensuremath{\to} \mu\mu$ events are selected using a $Z$ mass window cut between 70 and 110~GeV. ``Toward'', ``away'' and ``transverse'' regions are defined in the same way as in the original (2001) CDF underlying event analysis. The reconstructed $Z$ defines the $\phi$ direction of the toward region. The leptons are ignored after the $Z$ has been reconstructed. Thus the region most sensitive to the underlying event is the toward region (the recoil jet is boosted into the away region).

\clearpage


\clearpage

\typeout{Handling analysis CDF_2008_S7540469}
\subsection[CDF\_2008\_S7540469]{CDF\_2008\_S7540469\,\cite{:2007cp}}
\textbf{Measurement of differential Z/$\gamma^*$ + jet + X cross sections}\newline
\textbf{Experiment:} CDF (Tevatron Run 2) \newline
\textbf{Spires ID:} \href{http://www.slac.stanford.edu/spires/find/hep/www?rawcmd=key+7540469}{7540469}\newline
\textbf{Status:} \newline
\textbf{Authors:}
 \penalty 100
\begin{itemize}
  \item Frank Siegert $\langle\,$\href{mailto:frank.siegert@durham.ac.uk}{frank.siegert@durham.ac.uk}$\,\rangle$;
\end{itemize}
\textbf{References:}
 \penalty 100
\begin{itemize}
  \item Phys.Rev.Lett.100:102001,2008
  \item arXiv: \href{http://arxiv.org/abs/0711.3717}{0711.3717}
\end{itemize}
\textbf{Run details:}
 \penalty 100
\begin{itemize}

  \item $p \bar{p} \to e^+ e^-$ + jets at 1960 GeV. Needs mass cut on lepton pair to  avoid photon singularity, looser than $66 < m_{ee} < 116$\end{itemize}

\noindent Cross sections as a function of jet transverse momentum in 1 and 2 jet events, and jet multiplicity in ppbar collisions at \ensuremath{\sqrt{s}} = 1.96 TeV, based on an integrated luminosity of $1.7~\text{fb}^{-1}$. The  measurements cover the rapidity region $|y_\text{jet}| < 2.1$ and  the transverse momentum range $\pT^\text{jet} > 30~\text{GeV}/c$.

\clearpage


\clearpage

\typeout{Handling analysis CDF_2008_S7541902}
\subsection[CDF\_2008\_S7541902]{CDF\_2008\_S7541902\,\cite{Aaltonen:2007ip}}
\textbf{Jet \pT distributions for 4 jet multiplicity bins as well as the jet multiplicity distribution in W + jets events.}\newline
\textbf{Experiment:} CDF (Tevatron Run 2) \newline
\textbf{Spires ID:} \href{http://www.slac.stanford.edu/spires/find/hep/www?rawcmd=key+7541902}{7541902}\newline
\textbf{Status:} \newline
\textbf{Authors:}
 \penalty 100
\begin{itemize}
  \item Ben Cooper $\langle\,$\href{mailto:b.d.cooper@qmul.ac.uk}{b.d.cooper@qmul.ac.uk}$\,\rangle$;
  \item Emily Nurse $\langle\,$\href{mailto:nurse@hep.ucl.ac.uk}{nurse@hep.ucl.ac.uk}$\,\rangle$;
\end{itemize}
\textbf{References:}
 \penalty 100
\begin{itemize}
  \item arXiv: \href{http://arxiv.org/abs/0711.4044}{0711.4044}
  \item Phys.Rev.D77:011108,2008
\end{itemize}
\textbf{Run details:}
 \penalty 100
\begin{itemize}

  \item Requires the process $p\bar{p} \rightarrow {W} \rightarrow{e}\nu$,  additional hard jets will also have to be included to get a good  description. The LO process in Herwig is set with IPROC=1451.\end{itemize}

\noindent Measurement of the cross section for W boson production in association with jets in $p\bar{p}$ collisions at $\sqrt{s}=1.96$ TeV. The analysis uses 320 pb$^{-1}$ of data collected with the CDF II detector. W bosons are identified in their $e\nu$ decay channel and jets are reconstructed using an $R < 0.4$ cone algorithm. For each $W + \geq$ n-jet sample (where n = 1--4) a measurement of d$\sigma({p}\bar{p} \rightarrow W + \geq$ n jet)/d$E_T(n^{th}$-jet) $\times$ BR($W \rightarrow{e}\nu$) is made, where d$E_T(n^{th}$-jet) is the Et of the n$^{th}$-highest Et jet above 20 GeV. A measurement of the total cross section, $\sigma(p\bar{p} \rightarrow W + \geq$ n-jet) $\times$ BR($W \rightarrow{e}\nu)$ with $E_T(n^{th}-jet) > 25$ GeV is also made. Both measurements are made for jets with $|\eta| < 2$ and for a limited region of the $W \rightarrow{e}\nu$ decay phase space; $|\eta^{e}| < 1.1$, $p_{T}^{e} > 20$ GeV, $p_{T}^{\nu} > 30$ GeV and $M_{T} > 20$ GeV. The cross sections are corrected for all detector effects and can be directly compared to particle level $W$ + jet(s) predictions. These measurements can be used to test and tune QCD predictions for the number of jets in and kinematics of $W$ + jets events.

\clearpage


\clearpage

\typeout{Handling analysis CDF_2008_S7782535}
\subsection[CDF\_2008\_S7782535]{CDF\_2008\_S7782535\,\cite{Aaltonen:2008de}}
\textbf{CDF Run II b-jet shape paper}\newline
\textbf{Experiment:} CDF (Tevatron Run 2) \newline
\textbf{Spires ID:} \href{http://www.slac.stanford.edu/spires/find/hep/www?rawcmd=key+7782535}{7782535}\newline
\textbf{Status:} \newline
\textbf{Authors:}
 \penalty 100
\begin{itemize}
  \item Alison Lister $\langle\,$\href{mailto:alister@fnal.gov}{alister@fnal.gov}$\,\rangle$;
  \item Emily Nurse $\langle\,$\href{mailto:nurse@hep.ucl.ac.uk}{nurse@hep.ucl.ac.uk}$\,\rangle$;
\end{itemize}
\textbf{References:}
 \penalty 100
\begin{itemize}
  \item arXiv: \href{http://arxiv.org/abs/0806.1699}{0806.1699}
  \item Phys.Rev.D78:072005,2008
\end{itemize}
\textbf{Run details:}
 \penalty 100
\begin{itemize}

  \item Requires  $2\rightarrow{2}$ QCD scattering processes. The minimum jet Et is 52 GeV, so a cut on kinematic \pTmin may be required for good statistics.\end{itemize}

\noindent A measurement of the shapes of b-jets using 300 pb$^{-1}$ of data obtained with CDF II in $p\bar{p}$ collisions at $\sqrt{s}=1.96$ TeV. The measured quantity is the average integrated jet shape, which is computed over an ensemble of jets. This quantity is expressed as $\Psi(r/R) = \langle\frac{p_T(0 \rightarrow r)}{p_T(0 \rightarrow R)}\rangle$, where \pT (0$\rightarrow$r) is the scalar sum of the transverse momenta of all objects inside a sub-cone of radius r around the jet axis. The integrated shapes are by definition normalized such that  $\Psi(r/R =1) = 1$.   The measurement is done in bins of jet \pT in the range 52 to 300 GeV/c. The jets have $|\eta| < 0.7$. The b-jets are expected to be broader than inclusive jets.Moreover, b-jets containing a single b-quark are expected to be narrower than those containing a b bbar pair from gluon splitting.

\clearpage


\clearpage

\typeout{Handling analysis CDF_2008_S7828950}
\subsection[CDF\_2008\_S7828950]{CDF\_2008\_S7828950\,\cite{Aaltonen:2008eq}}
\textbf{CDF Run II inclusive jet cross-section using the Midpoint algorithm}\newline
\textbf{Experiment:} CDF (Tevatron Run 2) \newline
\textbf{Spires ID:} \href{http://www.slac.stanford.edu/spires/find/hep/www?rawcmd=key+7828950}{7828950}\newline
\textbf{Status:} \newline
\textbf{Authors:}
 \penalty 100
\begin{itemize}
  \item Craig Group $\langle\,$\href{mailto:group@fnal.gov}{group@fnal.gov}$\,\rangle$;
  \item Frank Siegert $\langle\,$\href{mailto:frank.siegert@durham.ac.uk}{frank.siegert@durham.ac.uk}$\,\rangle$;
\end{itemize}
\textbf{References:}
 \penalty 100
\begin{itemize}
  \item arXiv: \href{http://arxiv.org/abs/0807.2204}{0807.2204}
  \item Phys.Rev.D78:052006,2008
\end{itemize}
\textbf{Run details:}
 \penalty 100
\begin{itemize}

  \item Requires $2\rightarrow{2}$ QCD scattering processes. The minimum jet $E_\perp$ is 62 GeV, so a cut on kinematic \pTmin may be required for good statistics.\end{itemize}

\noindent Measurement of the inclusive jet cross section in $p\bar{p}$ collisions at $\sqrt{s}=1.96$ TeV as a function of jet $E_\perp$, for $E_\perp >$ 62 GeV. The data is collected by the CDF II detector and has an integrated luminosity of 1.13 fb$^{-1}$. The measurement was made using the cone-based Midpoint jet clustering algorithm in rapidity bins within $|y|<2.1$. This measurement can be used to provide increased precision in PDFs at high parton momentum fraction $x$.

\clearpage


\clearpage

\typeout{Handling analysis CDF_2008_S8093652}
\subsection[CDF\_2008\_S8093652]{CDF\_2008\_S8093652\,\cite{Aaltonen:2008dn}}
\textbf{Dijet mass spectrum}\newline
\textbf{Experiment:} CDF (Tevatron Run 2) \newline
\textbf{Spires ID:} \href{http://www.slac.stanford.edu/spires/find/hep/www?rawcmd=key+8093652}{8093652}\newline
\textbf{Status:} \newline
\textbf{Authors:}
 \penalty 100
\begin{itemize}
  \item Frank Siegert $\langle\,$\href{mailto:frank.siegert@durham.ac.uk}{frank.siegert@durham.ac.uk}$\,\rangle$;
\end{itemize}
\textbf{References:}
 \penalty 100
\begin{itemize}
  \item arXiv: \href{http://arxiv.org/abs/0812.4036}{0812.4036}
\end{itemize}
\textbf{Run details:}
 \penalty 100
\begin{itemize}

  \item $p \bar{p} \to$ jets at 1960 GeV\end{itemize}

\noindent Dijet mass spectrum  from 0.2 TeV to 1.4 TeV in $p \bar{p}$ collisions at $\sqrt{s} = 1.96$ TeV, based on an integrated luminosity of 1.13 fb$^{-1}$.

\clearpage


\clearpage

\typeout{Handling analysis CDF_2008_S8095620}
\subsection[CDF\_2008\_S8095620]{CDF\_2008\_S8095620\,\cite{Aaltonen:2008mt}}
\textbf{CDF Run II Z+b-jet cross section paper, 2 fb-1}\newline
\textbf{Experiment:} CDF (Tevatron Run 2) \newline
\textbf{Spires ID:} \href{http://www.slac.stanford.edu/spires/find/hep/www?rawcmd=key+8095620}{8095620}\newline
\textbf{Status:} \newline
\textbf{Authors:}
 \penalty 100
\begin{itemize}
  \item Emily Nurse $\langle\,$\href{mailto:nurse@hep.ucl.ac.uk}{nurse@hep.ucl.ac.uk}$\,\rangle$;
\end{itemize}
\textbf{References:}
 \penalty 100
\begin{itemize}
  \item arXiv: \href{http://arxiv.org/abs/0812.4458}{0812.4458}
\end{itemize}
\textbf{Run details:}
 \penalty 100
\begin{itemize}

  \item Requires the process $p\bar{p} \rightarrow {Z} \rightarrow{\ell}\ell$, where $\ell$ is $e$ or $\mu$. Additional hard jets will also have to be included to get a good description.\end{itemize}

\noindent Measurement of the b-jet production cross section for events containing a $Z$ boson produced in $p\bar{p}$ collisions at $\sqrt{s}=1.96$ TeV, using data corresponding to an integrated luminosity of 2 fb$^{-1}$ collected by the CDF II detector at the Tevatron. $Z$ bosons are selected in the electron and muon decay modes. Jets are considered with transverse energy $E_T>20$ GeV and pseudorapidity $|\eta|<1.5$. The ratio of the integrated $Z$ + b-jet cross section to the inclusive $Z$ production cross section is measured differentially in jet $E_T$, jet $\eta$, $Z$-boson transverse momentum, number of jets, and number of b-jets. The first two measurements have an entry for each b-jet in the event, the last three measurements have one entry per event.

\clearpage


\clearpage

\typeout{Handling analysis CDF_2009_S8233977}
\subsection[CDF\_2009\_S8233977]{CDF\_2009\_S8233977\,\cite{Aaltonen:2009ne}}
\textbf{CDF Run 2 min bias cross-section analysis}\newline
\textbf{Experiment:} CDF (Tevatron Run 2) \newline
\textbf{Spires ID:} \href{http://www.slac.stanford.edu/spires/find/hep/www?rawcmd=key+8233977}{8233977}\newline
\textbf{Status:} \newline
\textbf{Authors:}
 \penalty 100
\begin{itemize}
  \item Hendrik Hoeth $\langle\,$\href{mailto:hendrik.hoeth@cern.ch}{hendrik.hoeth@cern.ch}$\,\rangle$;
  \item Niccolo' Moggi $\langle\,$\href{mailto:niccolo.moggi@bo.infn.it}{niccolo.moggi@bo.infn.it}$\,\rangle$;
\end{itemize}
\textbf{References:}
 \penalty 100
\begin{itemize}
  \item Phys.Rev.D79:112005,2009
  \item DOI: \href{http://dx.doi.org/10.1103/PhysRevD.79.112005}{10.1103/PhysRevD.79.112005}
  \item arXiv: \href{http://arxiv.org/abs/0904.1098}{0904.1098}
\end{itemize}
\textbf{Run details:}
 \penalty 100
\begin{itemize}

  \item $p\bar{p}$ QCD interactions at 1960~GeV. Particles with $c \\tau > {}$10 mm  should be set stable.\end{itemize}

\noindent Niccolo Moggi's min bias analysis. Minimum bias events are used to measure the average track \pT vs. charged multiplicity, a track \pT distribution and an inclusive $\sum E_T$ distribution.

\clearpage


\clearpage

\typeout{Handling analysis CDF_2009_S8383952}
\subsection[CDF\_2009\_S8383952]{CDF\_2009\_S8383952\,\cite{Aaltonen:2009pc}}
\textbf{Z rapidity measurement}\newline
\textbf{Experiment:} CDF (Tevatron Run 2) \newline
\textbf{Spires ID:} \href{http://www.slac.stanford.edu/spires/find/hep/www?rawcmd=key+8383952}{8383952}\newline
\textbf{Status:} \newline
\textbf{Authors:}
 \penalty 100
\begin{itemize}
  \item Frank Siegert $\langle\,$\href{mailto:frank.siegert@durham.ac.uk}{frank.siegert@durham.ac.uk}$\,\rangle$;
\end{itemize}
\textbf{References:}
 \penalty 100
\begin{itemize}
  \item arXiv: \href{http://arxiv.org/abs/0908.3914}{0908.3914}
\end{itemize}
\textbf{Run details:}
 \penalty 100
\begin{itemize}

  \item $p \bar{p} \to e^+ e^-$ + jets at 1960 GeV. Needs mass cut on lepton pair to  avoid photon singularity, looser than $66 < m_{ee} < 116$ GeV\end{itemize}

\noindent CDF measurement of the total cross section and rapidity distribution, $\mathrm{d}\sigma/\mathrm{d}y$, for $q\bar{q}\to \gamma^{*}/Z\to e^{+}e^{-}$ events in the $Z$ boson mass region ($66<M_{ee}<116$ GeV/c$^2$) produced in $p\bar{p}$ collisions at $\sqrt{s}=1.96$ TeV with 2.1 fb$^{-1}$ of integrated luminosity.

\clearpage


\clearpage

\typeout{Handling analysis CDF_2009_S8436959}
\subsection[CDF\_2009\_S8436959]{CDF\_2009\_S8436959\,\cite{Aaltonen:2009ty}}
\textbf{Measurement of the inclusive isolated prompt photon cross-section}\newline
\textbf{Experiment:} CDF (Tevatron Run 2) \newline
\textbf{Spires ID:} \href{http://www.slac.stanford.edu/spires/find/hep/www?rawcmd=key+8436959}{8436959}\newline
\textbf{Status:} \newline
\textbf{Authors:}
 \penalty 100
\begin{itemize}
  \item Frank Siegert $\langle\,$\href{mailto:frank.siegert@durham.ac.uk}{frank.siegert@durham.ac.uk}$\,\rangle$;
\end{itemize}
\textbf{References:}
 \penalty 100
\begin{itemize}
  \item arXiv: \href{http://arxiv.org/abs/0910.3623}{0910.3623}
\end{itemize}
\textbf{Run details:}
 \penalty 100
\begin{itemize}

  \item $\gamma$ + jet processes in ppbar collisions at $\sqrt{s} = 1960$~GeV. Minimum \pT cut on the photon in the analysis is 30~GeV.\end{itemize}

\noindent A measurement of the cross section for the inclusive production of isolated photons. The measurement covers the pseudorapidity region $|\eta^\gamma|<1.0$ and the transverse energy range $E_T^\gamma>30$~GeV and is based on 2.5~fb$^{-1}$ of integrated luminosity. The cross section is measured differential in $E_\perp(\gamma)$.

\clearpage


\clearpage

\typeout{Handling analysis D0_1996_S3214044}
\subsection[D0\_1996\_S3214044]{D0\_1996\_S3214044\,\cite{Abachi:1995zv}}
\textbf{Topological distributions of inclusive three- and four-jet events}\newline
\textbf{Experiment:} D0 (Tevatron Run 1) \newline
\textbf{Spires ID:} \href{http://www.slac.stanford.edu/spires/find/hep/www?rawcmd=key+3214044}{3214044}\newline
\textbf{Status:} \newline
\textbf{Authors:}
 \penalty 100
\begin{itemize}
  \item Frank Siegert $\langle\,$\href{mailto:frank.siegert@durham.ac.uk}{frank.siegert@durham.ac.uk}$\,\rangle$;
\end{itemize}
\textbf{References:}
 \penalty 100
\begin{itemize}
  \item Phys.Rev.D53:6000-6016,1996
  \item DOI: \href{http://dx.doi.org/10.1103/PhysRevD.53.6000}{10.1103/PhysRevD.53.6000}
  \item arXiv: \href{http://arxiv.org/abs/hep-ex/9509005}{hep-ex/9509005}
\end{itemize}
\textbf{Run details:}
 \penalty 100
\begin{itemize}

  \item $p \bar{p} \to$ jets at 1800 GeV with minimum jet \pT in analysis = 20 GeV\end{itemize}

\noindent The global topologies of inclusive three- and four-jet events produced in pbar p interactions are described. The three- and four-jet events are selected from data recorded by the D0 detector at the Fermilab Tevatron Collider operating at a center-of-mass energy of $\sqrt{s}$=1800 GeV. The studies also show that the topological distributions of the different subprocesses involving different numbers of quarks are very similar and reproduce the measured distributions well. The parton-shower Monte Carlo generators provide a less satisfactory description of the topologies of the three- and four-jet events.

\clearpage


\clearpage

\typeout{Handling analysis D0_1996_S3324664}
\subsection[D0\_1996\_S3324664]{D0\_1996\_S3324664\,\cite{Abachi:1996et}}
\textbf{Azimuthal decorrelation of jets widely separated in rapidity}\newline
\textbf{Experiment:} D0 (Tevatron Run 1) \newline
\textbf{Spires ID:} \href{http://www.slac.stanford.edu/spires/find/hep/www?rawcmd=key+3324664}{3324664}\newline
\textbf{Status:} \newline
\textbf{Authors:}
 \penalty 100
\begin{itemize}
  \item Frank Siegert $\langle\,$\href{mailto:frank.siegert@durham.ac.uk}{frank.siegert@durham.ac.uk}$\,\rangle$;
\end{itemize}
\textbf{References:}
 \penalty 100
\begin{itemize}
  \item Phys.Rev.Lett.77:595-600,1996
  \item DOI: \href{http://dx.doi.org/10.1103/PhysRevLett.77.595}{10.1103/PhysRevLett.77.595}
  \item arXiv: \href{http://arxiv.org/abs/hep-ex/9603010}{hep-ex/9603010}
\end{itemize}
\textbf{Run details:}
 \penalty 100
\begin{itemize}

  \item $p \bar{p} \to jets$ at 1800 GeV\end{itemize}

\noindent First measurement of the azimuthal decorrelation between jets with pseudorapidity separation up to five units. The data were accumulated using the D0 detector during Tevatron Run 1 at $\sqrt{s}=1.8$ TeV. So far, this analysis is using the wrong jet algorithm, namely the D0 Run II Improved Legacy cone.

\clearpage


\clearpage

\typeout{Handling analysis D0_1998_S3711838}
\subsection[D0\_1998\_S3711838]{D0\_1998\_S3711838\,\cite{Abbott:1998jy}}
\textbf{W-boson \pT measurement in $p\bar{p}$ collisions at $\sqrt{s}=1.8~\TeV$}\newline
\textbf{Experiment:} D0 (Tevatron) \newline
\textbf{Spires ID:} \href{http://www.slac.stanford.edu/spires/find/hep/www?rawcmd=key+3711838}{3711838}\newline
\textbf{Status:} \newline
\textbf{Authors:}
 \penalty 100
\begin{itemize}
  \item Holger Schulz $\langle\,$\href{mailto:hschulz@physik.hu-berlin.de}{hschulz@physik.hu-berlin.de}$\,\rangle$;
\end{itemize}
\textbf{References:}
 \penalty 100
\begin{itemize}
  \item Phys. Rev. Lett. 80, 5498–5503
\end{itemize}
\textbf{Run details:}
 \penalty 100
\begin{itemize}

  \item QCD events with W+- production and electronic decays\end{itemize}

\noindent This is a D0 analysis from run 1, where the distribution of the transverse momentum of W candidates that decay  electronically, is measured. The electron is required to be within $\left|\eta\right| < 1.1$ and to have a transverse energy of $E_\perp > 25~\GeV$.  The neutrino is required to produce a missing energy of $E_{\perp, \text{ miss}}>25~\GeV$. The analysed data sample is three times as large as the similar measurement performed at CDF in 1991.

\clearpage


\clearpage

\typeout{Handling analysis D0_2001_S4674421}
\subsection[D0\_2001\_S4674421]{D0\_2001\_S4674421\,\cite{Abazov:2001nta}}
\textbf{Tevatron Run I differential W/Z boson cross-section analysis}\newline
\textbf{Experiment:} D0 (Tevatron Run 1) \newline
\textbf{Spires ID:} \href{http://www.slac.stanford.edu/spires/find/hep/www?rawcmd=key+4674421}{4674421}\newline
\textbf{Status:} \newline
\textbf{Authors:}
 \penalty 100
\begin{itemize}
  \item Lars Sonnenschein $\langle\,$\href{mailto:Lars.Sonnenschein@cern.ch}{Lars.Sonnenschein@cern.ch}$\,\rangle$;
\end{itemize}
\textbf{References:}
 \penalty 100
\begin{itemize}
  \item Phys.Lett.B517:299-308,2001
  \item DOI: \href{http://dx.doi.org/10.1016/S0370-2693(01)01020-6}{10.1016/S0370-2693(01)01020-6}
  \item arXiv: \href{http://arxiv.org/abs/hep-ex/0107012v2}{hep-ex/0107012v2}
\end{itemize}
\textbf{Run details:}
 \penalty 100
\begin{itemize}

  \item W/Z events with decays to first generation leptons, in ppbar collisions at \ensuremath{\sqrt{s}} = 1800~GeV\end{itemize}

\noindent Measurement of differential W/Z boson cross section and ratio in $p \bar{p}$ collisions at center-of-mass energy \ensuremath{\sqrt{s}} = 1.8 TeV. The data cover electrons and neutrinos in a pseudo-rapidity range of -2.5 to 2.5.

\clearpage


\clearpage

\typeout{Handling analysis D0_2004_S5992206}
\subsection[D0\_2004\_S5992206]{D0\_2004\_S5992206\,\cite{Abazov:2004hm}}
\textbf{Run II jet azimuthal decorrelation analysis}\newline
\textbf{Experiment:} D0 (Tevatron Run 2) \newline
\textbf{Spires ID:} \href{http://www.slac.stanford.edu/spires/find/hep/www?rawcmd=key+5992206}{5992206}\newline
\textbf{Status:} \newline
\textbf{Authors:}
 \penalty 100
\begin{itemize}
  \item Lars Sonnenschein $\langle\,$\href{mailto:lars.sonnenschein@cern.ch}{lars.sonnenschein@cern.ch}$\,\rangle$;
\end{itemize}
\textbf{References:}
 \penalty 100
\begin{itemize}
  \item Phys. Rev. Lett., 94, 221801 (2005)
  \item arXiv: \href{http://arxiv.org/abs/hep-ex/0409040}{hep-ex/0409040}
\end{itemize}
\textbf{Run details:}
 \penalty 100
\begin{itemize}

  \item QCD events in ppbar interactions at \ensuremath{\sqrt{s}} = 1960 GeV.\end{itemize}

\noindent Correlations in the azimuthal angle between the two largest \pT jets have been measured using the D0 detector in ppbar collisions at 1960 GeV. The analysis is based on an inclusive dijet event sample in the central rapidity region. The correlations are determined for four different \pT intervals.

\clearpage


\clearpage

\typeout{Handling analysis D0_2006_S6438750}
\subsection[D0\_2006\_S6438750]{D0\_2006\_S6438750\,\cite{Abazov:2005wc}}
\textbf{Inclusive isolated photon cross-section, differential in \pT(gamma)}\newline
\textbf{Experiment:} D0 (Tevatron Run 2) \newline
\textbf{Spires ID:} \href{http://www.slac.stanford.edu/spires/find/hep/www?rawcmd=key+6438750}{6438750}\newline
\textbf{Status:} \newline
\textbf{Authors:}
 \penalty 100
\begin{itemize}
  \item Andy Buckley $\langle\,$\href{mailto:andy.buckley@durham.ac.uk}{andy.buckley@durham.ac.uk}$\,\rangle$;
  \item Gavin Hesketh $\langle\,$\href{mailto:gavin.hesketh@cern.ch}{gavin.hesketh@cern.ch}$\,\rangle$;
\end{itemize}
\textbf{References:}
 \penalty 100
\begin{itemize}
  \item Phys.Lett.B639:151-158,2006, Erratum-ibid.B658:285-289,2008
  \item DOI: \href{http://dx.doi.org/10.1016/j.physletb.2006.04.048}{10.1016/j.physletb.2006.04.048}
  \item arXiv: \href{http://arxiv.org/abs/hep-ex/0511054}{hep-ex/0511054}
\end{itemize}
\textbf{Run details:}
 \penalty 100
\begin{itemize}

  \item ppbar collisions at \ensuremath{\sqrt{s}} = 1960 GeV. Requires gamma + jet (q,qbar,g)  hard processes, which for Pythia 6 means MSEL=10 for with MSUB indices  14, 18, 29, 114, 115 enabled.\end{itemize}

\noindent Measurement of differential cross section for inclusive production of isolated photons in p pbar collisions at \ensuremath{\sqrt{s}} = 1.96 TeV with the D\O detector at the Fermilab Tevatron collider. The photons span transverse momenta 23--300 GeV and have pseudorapidity $|\eta| < 0.9$. Isolated direct photons are probes of pQCD via the annihilation ($q \bar{q} \ensuremath{\to} \gamma g$) and quark-gluon Compton scattering ($q g \ensuremath{\to} \gamma q$) processes, the latter of which is also sensitive to the gluon PDF. The initial state radiation / resummation formalisms are sensitive to the resulting photon \pT spectrum

\clearpage


\clearpage

\typeout{Handling analysis D0_2007_S7075677}
\subsection[D0\_2007\_S7075677]{D0\_2007\_S7075677\,\cite{Abazov:2007jy}}
\textbf{$Z/\gamma^* + X$ cross-section shape, differential in $y(Z)$}\newline
\textbf{Experiment:} D0 (Tevatron Run 2) \newline
\textbf{Spires ID:} \href{http://www.slac.stanford.edu/spires/find/hep/www?rawcmd=key+7075677}{7075677}\newline
\textbf{Status:} \newline
\textbf{Authors:}
 \penalty 100
\begin{itemize}
  \item Andy Buckley $\langle\,$\href{mailto:andy.buckley@durham.ac.uk}{andy.buckley@durham.ac.uk}$\,\rangle$;
  \item Gavin Hesketh $\langle\,$\href{mailto:ghesketh@fnal.gov}{ghesketh@fnal.gov}$\,\rangle$;
  \item Frank Siegert $\langle\,$\href{mailto:frank.siegert@durham.ac.uk}{frank.siegert@durham.ac.uk}$\,\rangle$;
\end{itemize}
\textbf{References:}
 \penalty 100
\begin{itemize}
  \item Phys.Rev.D76:012003,2007
  \item arXiv: \href{http://arxiv.org/abs/hep-ex/0702025}{hep-ex/0702025}
\end{itemize}
\textbf{Run details:}
 \penalty 100
\begin{itemize}

  \item Drell-Yan $p \bar{p} \to Z/\gamma^*$ + jets events at $\sqrt{s}$ = 1960 GeV. Needs mass cut on lepton pair to avoid photon singularity, looser than  $71 < m_{ee} < 111$ GeV\end{itemize}

\noindent Cross sections as a function of boson rapidity in $p \bar{p}$  collisions at $\sqrt{s}$ = 1.96 TeV, based on an integrated luminosity  of $0.4~\text{fb}^{-1}$.

\clearpage


\clearpage

\typeout{Handling analysis D0_2008_S6879055}
\subsection[D0\_2008\_S6879055]{D0\_2008\_S6879055\,\cite{Abazov:2006gs}}
\textbf{Measurement of the ratio sigma($Z/\gamma^*$ + $n$ jets)/sigma($Z/\gamma^*$)}\newline
\textbf{Experiment:} D0 (Tevatron Run 2) \newline
\textbf{Spires ID:} \href{http://www.slac.stanford.edu/spires/find/hep/www?rawcmd=key+6879055}{6879055}\newline
\textbf{Status:} \newline
\textbf{Authors:}
 \penalty 100
\begin{itemize}
  \item Giulio Lenzi
  \item Frank Siegert $\langle\,$\href{mailto:frank.siegert@durham.ac.uk}{frank.siegert@durham.ac.uk}$\,\rangle$;
\end{itemize}
\textbf{References:}
 \penalty 100
\begin{itemize}
  \item hep-ex/0608052
\end{itemize}
\textbf{Run details:}
 \penalty 100
\begin{itemize}

  \item $p \bar{p} \to e^+ e^-$ + jets at 1960~GeV. Needs mass cut on lepton pair to avoid photon singularity, looser than $75 < m_{ee} < 105$ GeV.\end{itemize}

\noindent Cross sections as a function of \pT of the three leading jets and $n$-jet cross section ratios in $p \bar{p}$ collisions at $\sqrt{s}$ = 1.96 TeV, based on an integrated luminosity of $0.4~\text{fb}^{-1}$.

\clearpage


\clearpage

\typeout{Handling analysis D0_2008_S7554427}
\subsection[D0\_2008\_S7554427]{D0\_2008\_S7554427\,\cite{:2007nt}}
\textbf{$Z/\gamma^* + X$ cross-section shape, differential in $\pT(Z)$}\newline
\textbf{Experiment:} D0 (Tevatron Run 2) \newline
\textbf{Spires ID:} \href{http://www.slac.stanford.edu/spires/find/hep/www?rawcmd=key+7554427}{7554427}\newline
\textbf{Status:} \newline
\textbf{Authors:}
 \penalty 100
\begin{itemize}
  \item Andy Buckley $\langle\,$\href{mailto:andy.buckley@durham.ac.uk}{andy.buckley@durham.ac.uk}$\,\rangle$;
  \item Frank Siegert $\langle\,$\href{mailto:frank.siegert@durham.ac.uk}{frank.siegert@durham.ac.uk}$\,\rangle$;
\end{itemize}
\textbf{References:}
 \penalty 100
\begin{itemize}
  \item arXiv: \href{http://arxiv.org/abs/0712.0803}{0712.0803}
\end{itemize}
\textbf{Run details:}
 \penalty 100
\begin{itemize}

  \item * $p \bar{p} \to e^+ e^-$ + jets at 1960~GeV.
  \item Needs mass cut on lepton pair to avoid photon singularity, looser than $40 < m_{ee} < 200$ GeV.\end{itemize}

\noindent Cross sections as a function of \pT of the vector boson inclusive and in forward region ($|y| > 2$, $\pT<30$ GeV) in $p \bar{p}$ collisions at $\sqrt{s}$ = 1.96 TeV, based on an integrated luminosity of 0.98~fb$^{-1}$.

\clearpage


\clearpage

\typeout{Handling analysis D0_2008_S7662670}
\subsection[D0\_2008\_S7662670]{D0\_2008\_S7662670\,\cite{:2008hua}}
\textbf{Measurement of D0 Run II differential jet cross sections}\newline
\textbf{Experiment:} D0 (Tevatron Run 2) \newline
\textbf{Spires ID:} \href{http://www.slac.stanford.edu/spires/find/hep/www?rawcmd=key+7662670}{7662670}\newline
\textbf{Status:} \newline
\textbf{Authors:}
 \penalty 100
\begin{itemize}
  \item Andy Buckley $\langle\,$\href{mailto:andy.buckley@durham.ac.uk}{andy.buckley@durham.ac.uk}$\,\rangle$;
  \item Gavin Hesketh $\langle\,$\href{mailto:gavin.hesketh@cern.ch}{gavin.hesketh@cern.ch}$\,\rangle$;
\end{itemize}
\textbf{References:}
 \penalty 100
\begin{itemize}
  \item Phys.Rev.Lett.101:062001,2008
  \item DOI: \href{http://dx.doi.org/10.1103/PhysRevLett.101.062001}{10.1103/PhysRevLett.101.062001}
  \item arXiv: \href{http://arxiv.org/abs/0802.2400v3}{0802.2400v3}
\end{itemize}
\textbf{Run details:}
 \penalty 100
\begin{itemize}

  \item QCD events at \ensuremath{\sqrt{s}} = 1960 GeV. A \pTmin cut is probably necessary since  the lowest jet \pT bin is at 50 GeV\end{itemize}

\noindent Measurement of the inclusive jet cross section in $p \bar{p}$ collisions at center-of-mass energy \ensuremath{\sqrt{s}} = 1.96 TeV. The data cover jet transverse momenta from 50--600 GeV and jet rapidities in the range -2.4 to 2.4.

\clearpage


\clearpage

\typeout{Handling analysis D0_2008_S7719523}
\subsection[D0\_2008\_S7719523]{D0\_2008\_S7719523\,\cite{Abazov:2008er}}
\textbf{Isolated $\gamma$ + jet cross-sections, differential in \pT($\gamma$) for various $y$ bins}\newline
\textbf{Experiment:} D0 (Tevatron Run 2) \newline
\textbf{Spires ID:} \href{http://www.slac.stanford.edu/spires/find/hep/www?rawcmd=key+7719523}{7719523}\newline
\textbf{Status:} \newline
\textbf{Authors:}
 \penalty 100
\begin{itemize}
  \item Andy Buckley $\langle\,$\href{mailto:andy.buckley@durham.ac.uk}{andy.buckley@durham.ac.uk}$\,\rangle$;
  \item Gavin Hesketh $\langle\,$\href{mailto:gavin.hesketh@cern.ch}{gavin.hesketh@cern.ch}$\,\rangle$;
\end{itemize}
\textbf{References:}
 \penalty 100
\begin{itemize}
  \item Phys.Lett.B666:435-445,2008
  \item DOI: \href{http://dx.doi.org/10.1016/j.physletb.2008.06.076}{10.1016/j.physletb.2008.06.076}
  \item arXiv: \href{http://arxiv.org/abs/0804.1107v2}{0804.1107v2}
\end{itemize}
\textbf{Run details:}
 \penalty 100
\begin{itemize}

  \item Produce only gamma + jet ($q,\bar{q},g$) hard processes (for Pythia 6, this means MSEL=10  and MSUB indices 14, 29 \& 115 enabled). The lowest bin edge is at 30 GeV, so a kinematic  \pTmin cut is probably required to fill the histograms.\end{itemize}

\noindent The process $p \bar{p}$ \ensuremath{\to} photon + jet + X as studied by the D0 detector at the Fermilab Tevatron collider at center-of-mass energy \ensuremath{\sqrt{s}} = 1.96 TeV. Photons are reconstructed in the central rapidity region $|y_\gamma| < 1.0$ with transverse momenta in the range 30--400 GeV, while jets are reconstructed in either the central $|y_\text{jet}| < 0.8$ or forward $1.5 < |y_\text{jet}| < 2.5$ rapidity intervals with $\pT^\text{jet} > 15~\text{GeV}$. The differential cross section $\mathrm{d}^3 \sigma / \mathrm{d}{\pT^\gamma} \mathrm{d}{y_\gamma} \mathrm{d}{y_\text{jet}}$ is measured as a function of $\pT^\gamma$ in four regions, differing by the relative orientations of the photon and the jet.  MC predictions have trouble with simultaneously describing the measured normalization and $\pT^\gamma$ dependence of the cross section in any of the four measured regions.

\clearpage


\clearpage

\typeout{Handling analysis D0_2008_S7837160}
\subsection[D0\_2008\_S7837160]{D0\_2008\_S7837160\,\cite{Abazov:2008qv}}
\textbf{Measurement of W charge asymmetry from D0 Run II}\newline
\textbf{Experiment:} D0 (Tevatron Run 2) \newline
\textbf{Spires ID:} \href{http://www.slac.stanford.edu/spires/find/hep/www?rawcmd=key+7837160}{7837160}\newline
\textbf{Status:} \newline
\textbf{Authors:}
 \penalty 100
\begin{itemize}
  \item Andy Buckley $\langle\,$\href{mailto:andy.buckley@durham.ac.uk}{andy.buckley@durham.ac.uk}$\,\rangle$;
  \item Gavin Hesketh $\langle\,$\href{mailto:gavin.hesketh@cern.ch}{gavin.hesketh@cern.ch}$\,\rangle$;
\end{itemize}
\textbf{References:}
 \penalty 100
\begin{itemize}
  \item Phys.Rev.Lett.101:211801,2008
  \item DOI: \href{http://dx.doi.org/10.1103/PhysRevLett.101.211801}{10.1103/PhysRevLett.101.211801}
  \item arXiv: \href{http://arxiv.org/abs/0807.3367v1}{0807.3367v1}
\end{itemize}
\textbf{Run details:}
 \penalty 100
\begin{itemize}

  \item * Event type: W production with decay to $e \, \nu_e$ only  
  \item for Pythia 6: MSEL = 12, MDME(206,1) = 1
  \item Energy: 1.96 TeV\end{itemize}

\noindent Measurement of the electron charge asymmetry in $p \bar p \to W + X \to e \nu_e + X$ events at a center of mass energy of 1.96 TeV. The asymmetry is measured as a function of the electron transverse momentum and pseudorapidity in the interval (-3.2, 3.2).  This data is sensitive to proton parton distribution functions due to the valence asymmetry in the incoming quarks which produce the W. Initial state radiation should also affect the \pT distribution.

\clearpage


\clearpage

\typeout{Handling analysis D0_2008_S7863608}
\subsection[D0\_2008\_S7863608]{D0\_2008\_S7863608\,\cite{Abazov:2008ez}}
\textbf{Measurement of differential $Z/\gamma^*$ + jet + $X$ cross sections}\newline
\textbf{Experiment:} D0 (Tevatron Run 2) \newline
\textbf{Spires ID:} \href{http://www.slac.stanford.edu/spires/find/hep/www?rawcmd=key+7863608}{7863608}\newline
\textbf{Status:} \newline
\textbf{Authors:}
 \penalty 100
\begin{itemize}
  \item Andy Buckley $\langle\,$\href{mailto:andy.buckley@durham.ac.uk}{andy.buckley@durham.ac.uk}$\,\rangle$;
  \item Gavin Hesketh $\langle\,$\href{mailto:gavin.hesketh@fnal.gov}{gavin.hesketh@fnal.gov}$\,\rangle$;
  \item Frank Siegert $\langle\,$\href{mailto:frank.siegert@durham.ac.uk}{frank.siegert@durham.ac.uk}$\,\rangle$;
\end{itemize}
\textbf{References:}
 \penalty 100
\begin{itemize}
  \item arXiv: \href{http://arxiv.org/abs/0808.1296}{0808.1296}
\end{itemize}
\textbf{Run details:}
 \penalty 100
\begin{itemize}

  \item $p \bar{p} \to \mu^+ \mu^-$ + jets at 1960~GeV. Needs mass cut on lepton pair to avoid photon singularity, looser than $65 < m_{ee} < 115$ GeV.\end{itemize}

\noindent Cross sections as a function of \pT and rapidity of the boson and \pT and rapidity of the leading jet in $p \bar{p}$ collisions at $\sqrt{s}$ = 1.96 TeV, based on an integrated luminosity of 1.0 fb$^{-1}$.

\clearpage


\clearpage

\typeout{Handling analysis D0_2009_S8202443}
\subsection[D0\_2009\_S8202443]{D0\_2009\_S8202443\,\cite{Abazov:2009av}}
\textbf{$Z/\gamma^*$ + jet + $X$ cross sections differential in \pT(jet 1,2,3)}\newline
\textbf{Experiment:} D0 (Tevatron Run 2) \newline
\textbf{Spires ID:} \href{http://www.slac.stanford.edu/spires/find/hep/www?rawcmd=key+8202443}{8202443}\newline
\textbf{Status:} \newline
\textbf{Authors:}
 \penalty 100
\begin{itemize}
  \item Frank Siegert $\langle\,$\href{mailto:frank.siegert@durham.ac.uk}{frank.siegert@durham.ac.uk}$\,\rangle$;
\end{itemize}
\textbf{References:}
 \penalty 100
\begin{itemize}
  \item arXiv: \href{http://arxiv.org/abs/0903.1748}{0903.1748}
\end{itemize}
\textbf{Run details:}
 \penalty 100
\begin{itemize}

  \item $p \bar{p} \to e^+ e^-$ + jets at 1960~GeV. Needs mass cut on lepton pair to avoid photon singularity, looser than $65 < m_{ee} < 115$ GeV.\end{itemize}

\noindent Cross sections as a function of \pT of the three leading jets in $Z/\gamma^{*} (\to e^{+} e^{-})$ + jet + X production in $p \bar{p}$ collisions at $\sqrt{s} = 1.96$ TeV, based on an integrated luminosity of 1.0 fb$^{-1}$.

\clearpage


\clearpage

\typeout{Handling analysis D0_2009_S8320160}
\subsection[D0\_2009\_S8320160]{D0\_2009\_S8320160\,\cite{:2009mh}}
\textbf{Dijet angular distributions}\newline
\textbf{Experiment:} D0 (Tevatron Run 2) \newline
\textbf{Spires ID:} \href{http://www.slac.stanford.edu/spires/find/hep/www?rawcmd=key+8320160}{8320160}\newline
\textbf{Status:} \newline
\textbf{Authors:}
 \penalty 100
\begin{itemize}
  \item Frank Siegert $\langle\,$\href{mailto:frank.siegert@durham.ac.uk}{frank.siegert@durham.ac.uk}$\,\rangle$;
\end{itemize}
\textbf{References:}
 \penalty 100
\begin{itemize}
  \item arXiv: \href{http://arxiv.org/abs/0906.4819}{0906.4819}
\end{itemize}
\textbf{Run details:}
 \penalty 100
\begin{itemize}

  \item $p \bar{p} \to$ jets at 1960 GeV\end{itemize}

\noindent Dijet angular distributions in different bins of dijet mass from 0.25 TeV to above 1.1 TeV in $p \bar{p}$ collisions at $\sqrt{s} = 1.96$ TeV, based on an integrated luminosity of 0.7 fb$^{-1}$.

\clearpage


\clearpage

\typeout{Handling analysis D0_2009_S8349509}
\subsection[D0\_2009\_S8349509]{D0\_2009\_S8349509\,\cite{Abazov:2009pp}}
\textbf{Z+jets angular distributions}\newline
\textbf{Experiment:} D0 (Tevatron Run 2) \newline
\textbf{Spires ID:} \href{http://www.slac.stanford.edu/spires/find/hep/www?rawcmd=key+8349509}{8349509}\newline
\textbf{Status:} \newline
\textbf{Authors:}
 \penalty 100
\begin{itemize}
  \item Frank Siegert $\langle\,$\href{mailto:frank.siegert@durham.ac.uk}{frank.siegert@durham.ac.uk}$\,\rangle$;
\end{itemize}
\textbf{References:}
 \penalty 100
\begin{itemize}
  \item arXiv: \href{http://arxiv.org/abs/0907.4286}{0907.4286}
\end{itemize}
\textbf{Run details:}
 \penalty 100
\begin{itemize}

  \item $p \bar{p} \to \mu^+ \mu^-$ + jets at 1960~GeV. Needs mass cut on lepton pair to avoid photon singularity, looser than $65 < m_{ee} < 115$ GeV.\end{itemize}

\noindent First measurements at a hadron collider of differential cross sections for $Z$+jet+X production in $\Delta\phi(Z, j)$, $|\Delta y(Z, j)|$ and $|y_\mathrm{boost}(Z, j)|$. Vector boson production in association with jets is an excellent probe of QCD and constitutes the main background to many small cross section processes, such as associated Higgs production. These measurements are crucial tests of the predictions of perturbative QCD and current event generators, which have varied success in describing the data. Using these measurements as inputs in tuning event generators will increase the experimental sensitivity to rare signals.

\clearpage


\clearpage

\typeout{Handling analysis D0_2010_S8566488}
\subsection[D0\_2010\_S8566488]{D0\_2010\_S8566488\,\cite{Abazov:2010fr}}
\textbf{Dijet invariant mass}\newline
\textbf{Experiment:} D0 (Tevatron Run 2) \newline
\textbf{Spires ID:} \href{http://www.slac.stanford.edu/spires/find/hep/www?rawcmd=key+8566488}{8566488}\newline
\textbf{Status:} \newline
\textbf{Authors:}
 \penalty 100
\begin{itemize}
  \item Frank Siegert $\langle\,$\href{mailto:frank.siegert@durham.ac.uk}{frank.siegert@durham.ac.uk}$\,\rangle$;
\end{itemize}
\textbf{References:}
 \penalty 100
\begin{itemize}
  \item arXiv: \href{http://arxiv.org/abs/1002.4594}{1002.4594}
\end{itemize}
\textbf{Run details:}
 \penalty 100
\begin{itemize}

  \item $p \bar{p} \to$ jets at 1960 GeV. Analysis needs two hard jets above 40 GeV.\end{itemize}

\noindent The inclusive dijet production double differential cross section as a function of the dijet invariant mass and of the largest absolute rapidity ($|y|_\text{max}$) of the two jets with the largest transverse momentum in an event is measured using 0.7 fb$^{-1}$ of data. The measurement is performed in six rapidity regions up to $|y|_\text{max}=2.4$.

\clearpage


\clearpage

\typeout{Handling analysis D0_2010_S8570965}
\subsection[D0\_2010\_S8570965]{D0\_2010\_S8570965\,\cite{Abazov:2010ah}}
\textbf{Direct photon pair production}\newline
\textbf{Experiment:} CDF (Tevatron Run 2) \newline
\textbf{Spires ID:} \href{http://www.slac.stanford.edu/spires/find/hep/www?rawcmd=key+8570965}{8570965}\newline
\textbf{Status:} \newline
\textbf{Authors:}
 \penalty 100
\begin{itemize}
  \item Frank Siegert $\langle\,$\href{mailto:frank.siegert@durham.ac.uk}{frank.siegert@durham.ac.uk}$\,\rangle$;
\end{itemize}
\textbf{References:}
 \penalty 100
\begin{itemize}
  \item arXiv: \href{http://arxiv.org/abs/1002.4917}{1002.4917}
\end{itemize}
\textbf{Run details:}
 \penalty 100
\begin{itemize}

  \item All processes that can produce prompt photon pairs, e.g. $jj \to jj$, $jj \to j\gamma$ and $jj \to \gamma \gamma$. Non-prompt photons from hadron decays like $\pi$ and $\eta$ have been corrected for.\end{itemize}

\noindent Direct photon pair production cross sections are measured using 4.2 fb$^{-1}$ of data. They are binned in diphoton mass, the transverse momentum of the diphoton system, the azimuthal angle between the photons, and the polar scattering angle of the photons. Also available are double differential cross sections considering the last three kinematic variables in three diphoton mass bins.

\clearpage


\clearpage

\typeout{Handling analysis E735_1998_S3905616}
\subsection[E735\_1998\_S3905616]{E735\_1998\_S3905616\,\cite{Alexopoulos:1998bi}}
\textbf{Charged particle multiplicity in ppbar collisions at \ensuremath{\sqrt{s}} = 1.8 TeV}\newline
\textbf{Experiment:} E735 (Tevatron) \newline
\textbf{Spires ID:} \href{http://www.slac.stanford.edu/spires/find/hep/www?rawcmd=key+3905616}{3905616}\newline
\textbf{Status:} \newline
\textbf{Authors:}
 \penalty 100
\begin{itemize}
  \item Holger Schulz $\langle\,$\href{mailto:holger.schulz@physik.hu-berlin.de}{holger.schulz@physik.hu-berlin.de}$\,\rangle$;
  \item Andy Buckley $\langle\,$\href{mailto:andy.buckley@cern.ch}{andy.buckley@cern.ch}$\,\rangle$;
\end{itemize}
\textbf{References:}
 \penalty 100
\begin{itemize}
  \item Phys.Lett.B435:453-457,1998
\end{itemize}
\textbf{Run details:}
 \penalty 100
\begin{itemize}

  \item QCD events, diffractive processes need to be switched on in order to fill the low multiplicity regions. The measurement was done in $\left|\eta\right| ~< ~3.25$ and was extrapolated to full phase space. However, the method of extrapolation remains unclear.\end{itemize}

\noindent A measurement of the charged multiplicity distribution at \ensuremath{\sqrt{s}} = 1.8 TeV.

\clearpage


\section{LHC analyses}\typeout{Handling analysis ATLAS_2010_S8591806}
\subsection[ATLAS\_2010\_S8591806]{ATLAS\_2010\_S8591806\,\cite{Aad:2010rd}}
\textbf{Charged particles at 900 GeV in ATLAS}\newline
\textbf{Experiment:} ATLAS (LHC 900GeV) \newline
\textbf{Spires ID:} \href{http://www.slac.stanford.edu/spires/find/hep/www?rawcmd=key+8591806}{8591806}\newline
\textbf{Status:} \newline
\textbf{Authors:}
 \penalty 100
\begin{itemize}
  \item Frank Siegert $\langle\,$\href{mailto:frank.siegert@durham.ac.uk}{frank.siegert@durham.ac.uk}$\,\rangle$;
\end{itemize}
\textbf{References:}
 \penalty 100
\begin{itemize}
  \item arXiv: \href{http://arxiv.org/abs/1003.3124}{1003.3124}
\end{itemize}
\textbf{Run details:}
 \penalty 100
\begin{itemize}

  \item pp QCD interactions at 900 GeV including diffractive events.\end{itemize}

\noindent The first measurements with the ATLAS detector at the LHC. Data were collected using a minimum-bias trigger in December 2009 during proton-proton collisions at a centre of mass energy of 900 GeV. The charged- particle density, its dependence on transverse momentum and pseudorapid- ity, and the relationship between transverse momentum and charged-particle multiplicity are measured for events with at least one charged particle in the kinematic range $|\eta| < 2.5$ and $p_\perp > 500$ MeV. All data is corrected to the particle level.

\clearpage


\section{SPS analyses}\typeout{Handling analysis UA1_1990_S2044935}
\subsection[UA1\_1990\_S2044935]{UA1\_1990\_S2044935\,\cite{Albajar:1989an}}
\textbf{UA1 multiplicities, transverse momenta and transverse energy distributions.}\newline
\textbf{Experiment:} UA1 (SPS) \newline
\textbf{Spires ID:} \href{http://www.slac.stanford.edu/spires/find/hep/www?rawcmd=key+2044935}{2044935}\newline
\textbf{Status:} \newline
\textbf{Authors:}
 \penalty 100
\begin{itemize}
  \item Andy Buckley $\langle\,$\href{mailto:andy.buckley@cern.ch}{andy.buckley@cern.ch}$\,\rangle$;
  \item Christophe Vaillant $\langle\,$\href{mailto:c.l.j.j.vaillant@durham.ac.uk}{c.l.j.j.vaillant@durham.ac.uk}$\,\rangle$;
\end{itemize}
\textbf{References:}
 \penalty 100
\begin{itemize}
  \item Nucl.Phys.B353:261,1990
\end{itemize}
\textbf{Run details:}
 \penalty 100
\begin{itemize}

  \item QCD min bias events at sqrtS = 63, 200, 500 and 900 GeV.\end{itemize}

\noindent Particle multiplicities, transverse momenta and transverse energy distributions at the UA1 experiment, at energies of 200, 500 and 900 GeV (with one plot at 63 GeV for comparison).

\clearpage


\clearpage

\typeout{Handling analysis UA5_1982_S875503}
\subsection[UA5\_1982\_S875503]{UA5\_1982\_S875503\,\cite{Alpgard:1982zx}}
\textbf{UA5 multiplicity and pseudorapidity distributions for $pp$ and $p\bar{p}$.}\newline
\textbf{Experiment:} UA5 (SPS) \newline
\textbf{Spires ID:} \href{http://www.slac.stanford.edu/spires/find/hep/www?rawcmd=key+875503}{875503}\newline
\textbf{Status:} \newline
\textbf{Authors:}
 \penalty 100
\begin{itemize}
  \item Andy Buckley $\langle\,$\href{mailto:andy.buckley@cern.ch}{andy.buckley@cern.ch}$\,\rangle$;
  \item Christophe Vaillant $\langle\,$\href{mailto:c.l.j.j.vaillant@durham.ac.uk}{c.l.j.j.vaillant@durham.ac.uk}$\,\rangle$;
\end{itemize}
\textbf{References:}
 \penalty 100
\begin{itemize}
  \item Phys.Lett.112B:183,1982
\end{itemize}
\textbf{Run details:}
 \penalty 100
\begin{itemize}

  \item Min bias QCD events at \ensuremath{\sqrt{s}} = 53~GeV. Run with both $pp$ and $p\bar{p}$ beams.\end{itemize}

\noindent Comparisons of multiplicity and pseudorapidity distributions for $pp$ and  $p\bar{p}$ collisions at 53 GeV, based on the UA5 53~GeV runs in 1982. Data confirms the lack of significant difference between the two beams.

\clearpage


\clearpage

\typeout{Handling analysis UA5_1986_S1583476}
\subsection[UA5\_1986\_S1583476]{UA5\_1986\_S1583476\,\cite{Alner:1986xu}}
\textbf{Pseudorapidity distributions in $p\bar{p}$ (NSD, NSD+SD) events at \ensuremath{\sqrt{s}} = 200 and 900 GeV}\newline
\textbf{Experiment:} UA5 (CERN SPS) \newline
\textbf{Spires ID:} \href{http://www.slac.stanford.edu/spires/find/hep/www?rawcmd=key+1583476}{1583476}\newline
\textbf{Status:} \newline
\textbf{Authors:}
 \penalty 100
\begin{itemize}
  \item Andy Buckley $\langle\,$\href{mailto:andy.buckley@cern.ch}{andy.buckley@cern.ch}$\,\rangle$;
  \item Holger Schulz $\langle\,$\href{mailto:holger.schulz@physik.hu-berlin.de}{holger.schulz@physik.hu-berlin.de}$\,\rangle$;
  \item Christophe Vaillant $\langle\,$\href{mailto:c.l.j.j.vaillant@durham.ac.uk}{c.l.j.j.vaillant@durham.ac.uk}$\,\rangle$;
\end{itemize}
\textbf{References:}
 \penalty 100
\begin{itemize}
  \item Eur. Phys. J. C33, 1, 1986
\end{itemize}
\textbf{Run details:}
 \penalty 100
\begin{itemize}

  \item * Single- and double-diffractive, plus non-diffractive inelastic, events.
  \item $p\bar{p}$ collider, \ensuremath{\sqrt{s}} = 200 or 900 GeV.
  \item The trigger implementation for NSD events is the same as in, e.g.,  the UA5_1989 analysis. No further cuts are needed.\end{itemize}

\noindent This study comprises measurements of pseudorapidity distributions measured with the UA5 detector at 200 and 900 GeV center of momentum energy. There are distributions for non-single diffractive (NSD) events and also for the combination of single- and double-diffractive events. The NSD distributions are further studied for certain ranges of the events charged multiplicity.

\clearpage


\clearpage

\typeout{Handling analysis UA5_1988_S1867512}
\subsection[UA5\_1988\_S1867512]{UA5\_1988\_S1867512\,\cite{Ansorge:1988fg}}
\textbf{Charged particle correlations in ppbar non-single-diffractive events of the UA5 detector at \ensuremath{\sqrt{s}} = 200, 546 and 900 GeV.}\newline
\textbf{Experiment:} UA5 (CERN SPS) \newline
\textbf{Spires ID:} \href{http://www.slac.stanford.edu/spires/find/hep/www?rawcmd=key+1867512}{1867512}\newline
\textbf{Status:} \newline
\textbf{Authors:}
 \penalty 100
\begin{itemize}
  \item Holger Schulz $\langle\,$\href{mailto:holger.schulz@physik.hu-berlin.de}{holger.schulz@physik.hu-berlin.de}$\,\rangle$;
\end{itemize}
\textbf{References:}
 \penalty 100
\begin{itemize}
  \item Z.Phys.C37:191-213,1988
\end{itemize}
\textbf{Run details:}
 \penalty 100
\begin{itemize}

  \item ppbar events. Non-single diffractive events need to be switched on. The trigger implementation is the same as in UA5_1989_S1926373.\end{itemize}

\noindent Data on two-particle pseudorapidity and multiplicity correlations  of charged particles for non single-diffractive $\bar{p}p$ collisions at  c.m. energies of 200, 546 and 900 GeV. Pseudorapidity correlations interpreted  in terms of a cluster model, which has been motivated by this and other  experiments, require on average about two charged particles per cluster.  The decay width of the clusters in pseudorapidity is approximately independent  of multiplicity and of c.m. energy. The investigations of correlations in terms  of pseudorapidity gaps confirm the picture of cluster production. The strength  of forward-backward multiplicity correlations increases linearly with ins and  depends strongly on position and size of the pseudorapidity gap separating  the forward and backward interval. All our correlation studies can be understood  in terms of a cluster model in which clusters contain on average about two  charged particles, i.e. are of similar magnitude to earlier estimates from the ISR.

\clearpage


\clearpage

\typeout{Handling analysis UA5_1989_S1926373}
\subsection[UA5\_1989\_S1926373]{UA5\_1989\_S1926373\,\cite{Ansorge:1988kn}}
\textbf{UA5 charged multiplicity measurements}\newline
\textbf{Experiment:} UA5 (CERN SPS) \newline
\textbf{Spires ID:} \href{http://www.slac.stanford.edu/spires/find/hep/www?rawcmd=key+1926373}{1926373}\newline
\textbf{Status:} \newline
\textbf{Authors:}
 \penalty 100
\begin{itemize}
  \item Holger Schulz $\langle\,$\href{mailto:holger.schulz@physik.hu-berlin.de}{holger.schulz@physik.hu-berlin.de}$\,\rangle$;
  \item Christophe L. J. Vaillant $\langle\,$\href{mailto:c.l.j.j.vaillant@durham.ac.uk}{c.l.j.j.vaillant@durham.ac.uk}$\,\rangle$;
  \item Andy Buckley $\langle\,$\href{mailto:andy.buckley@cern.ch}{andy.buckley@cern.ch}$\,\rangle$;
\end{itemize}
\textbf{References:}
 \penalty 100
\begin{itemize}
  \item Z. Phys. C - Particles and Fields 43, 357-374 (1989)
  \item DOI: \href{http://dx.doi.org/10.1007/BF01506531}{10.1007/BF01506531}
\end{itemize}
\textbf{Run details:}
 \penalty 100
\begin{itemize}

  \item MinBias events at \ensuremath{\sqrt{s}} = 200 and 900 GeV. Enable single and double diffractive events in addition to minimum bias and non-diffractive processes.\end{itemize}

\noindent Multiplicity distributions of charged particles produced in non-single-diffractive  collisions between protons and antiprotons at centre-of-mass energies of 200 and  900 GeV. The data were recorded in the UA5 streamer chambers at the CERN collider,  which was operated in a pulsed mode between the two energies. This analysis confirms the violation of KNO scaling in full phase space found by the UA5 group  at an energy of 546 GeV, with similar measurements at 200 and 900 GeV.

\clearpage


\section{HERA analyses}\typeout{Handling analysis H1_1994_S2919893}
\subsection[H1\_1994\_S2919893]{H1\_1994\_S2919893\,\cite{Abt:1994ye}}
\textbf{H1 energy flow and charged particle spectra in DIS}\newline
\textbf{Experiment:} H1 (HERA) \newline
\textbf{Spires ID:} \href{http://www.slac.stanford.edu/spires/find/hep/www?rawcmd=key+2919893}{2919893}\newline
\textbf{Status:} \newline
\textbf{Authors:}
 \penalty 100
\begin{itemize}
  \item Peter Richardson $\langle\,$\href{mailto:peter.richardson@durham.ac.uk}{peter.richardson@durham.ac.uk}$\,\rangle$;
\end{itemize}
\textbf{References:}
 \penalty 100
\begin{itemize}
  \item Z.Phys.C63:377-390,1994
  \item DOI: \href{http://dx.doi.org/10.1007/BF01580319}{10.1007/BF01580319}
\end{itemize}
\textbf{Run details:}
 \penalty 100
\begin{itemize}

  \item $e^- p$ / $e^+ p$ deep inelastic scattering, 820~GeV protons colliding with 26.7~GeV electrons\end{itemize}

\noindent Global properties of the hadronic final state in deep inelastic scattering events at HERA are investigated. The data are corrected for detector effects. Energy flows in both the laboratory frame and the hadronic centre of mass system, and energy-energy correlations in the laboratory frame are presented.  Historically, the Ariadne colour dipole model provided the only satisfactory description of this data, hence making it a useful 'target' analysis for MC shower models.

\clearpage


\clearpage

\typeout{Handling analysis H1_1995_S3167097}
\subsection[H1\_1995\_S3167097]{H1\_1995\_S3167097\,\cite{Aid:1995we}}
\textbf{Transverse energy and forward jet production in the low x regime at H1}\newline
\textbf{Experiment:} H1 (HERA Run I) \newline
\textbf{Spires ID:} \href{http://www.slac.stanford.edu/spires/find/hep/www?rawcmd=key+3167097}{3167097}\newline
\textbf{Status:} \newline
\textbf{Authors:}
 \penalty 100
\begin{itemize}
  \item Leif Lonnblad $\langle\,$\href{mailto:leif.lonnblad@thep.lu.se}{leif.lonnblad@thep.lu.se}$\,\rangle$;
\end{itemize}
\textbf{References:}
 \penalty 100
\begin{itemize}
  \item Phys.Lett.B356:118,1995
  \item hep-ex/9506012
\end{itemize}
\textbf{Run details:}
 \penalty 100
\begin{itemize}

  \item 820~GeV protons colliding with 26.7~GeV electrons. DIS events with an outgoing electron energy $> 12~\text{GeV}$. $5~\text{GeV}^2 < Q^2 < 100~\text{GeV}^2$, $10^{-4} < x < 10^{-2}$.\end{itemize}

\noindent DIS events at low x may be sensitive to new QCD dynamics such as BFKL or CCFM radiation. In particular, BFKL is expected to produce more radiation at high transverse energy  in the rapidity span between the proton remnant and the struck quark jet. Performing a transverse energy sum in bins of x and $\eta$ may distinguish between DGLAP and BFKL evolution.

\clearpage


\clearpage

\typeout{Handling analysis H1_2000_S4129130}
\subsection[H1\_2000\_S4129130]{H1\_2000\_S4129130\,\cite{Adloff:1999ws}}
\textbf{H1 energy flow in DIS}\newline
\textbf{Experiment:} H1 (HERA) \newline
\textbf{Spires ID:} \href{http://www.slac.stanford.edu/spires/find/hep/www?rawcmd=key+4129130}{4129130}\newline
\textbf{Status:} \newline
\textbf{Authors:}
 \penalty 100
\begin{itemize}
  \item Peter Richardson $\langle\,$\href{mailto:peter.richardson@durham.ac.uk}{peter.richardson@durham.ac.uk}$\,\rangle$;
\end{itemize}
\textbf{References:}
 \penalty 100
\begin{itemize}
  \item Eur.Phys.J.C12:595-607,2000
  \item DOI: \href{http://dx.doi.org/10.1007/s100520000287}{10.1007/s100520000287}
  \item arXiv: \href{http://arxiv.org/abs/hep-ex/9907027v1}{hep-ex/9907027v1}
\end{itemize}
\textbf{Run details:}
 \penalty 100
\begin{itemize}

  \item $e^+ p$ deep inelastic scattering with $p$ at 820 GeV, $e^+$ at 27.5 GeV \ensuremath{\to} \ensuremath{\sqrt{s}} = 300 GeV\end{itemize}

\noindent Measurements of transverse energy flow for neutral current deep- inelastic scattering events produced in positron-proton collisions at HERA. The kinematic range covers squared momentum transfers $Q^2$ from 3.2 to 2200 GeV$^2$; the Bjorken scaling variable $x$ from $8 \times 10^{-5}$ to 0.11 and the hadronic mass $W$ from 66 to 233 GeV. The transverse energy flow is measured in the hadronic centre of mass frame and is studied as a function of $Q^2$, $x$, $W$ and pseudorapidity. The behaviour of the mean transverse energy in the central pseudorapidity region and an interval corresponding to the photon fragmentation region are analysed as a function of $Q^2$ and $W$.  This analysis is useful for exploring the effect of photon PDFs and for tuning models of parton evolution and treatment of fragmentation and the proton remnant in DIS.

\clearpage


\clearpage

\typeout{Handling analysis ZEUS_2001_S4815815}
\subsection[ZEUS\_2001\_S4815815]{ZEUS\_2001\_S4815815\,\cite{Chekanov:2001bw}}
\textbf{Dijet photoproduction analysis}\newline
\textbf{Experiment:} ZEUS (HERA Run I) \newline
\textbf{Spires ID:} \href{http://www.slac.stanford.edu/spires/find/hep/www?rawcmd=key+4815815}{4815815}\newline
\textbf{Status:} \newline
\textbf{Authors:}
 \penalty 100
\begin{itemize}
  \item Jon Butterworth $\langle\,$\href{mailto:jmb@hep.ucl.ac.uk}{jmb@hep.ucl.ac.uk}$\,\rangle$;
\end{itemize}
\textbf{References:}
 \penalty 100
\begin{itemize}
  \item Eur.Phys.J.C23:615,2002
  \item DESY 01/220
  \item hep-ex/0112029
\end{itemize}
\textbf{Run details:}
 \penalty 100
\begin{itemize}

  \item 820 GeV protons colliding with 27.5 GeV positrons; Direct and resolved photoproduction of dijets; Leading jet \pT $>$ 14 GeV, second jet \pT $>$ 11 GeV; Jet pseudorapidity $-1 < \eta < 2.4$\end{itemize}

\noindent ZEUS photoproduction of jets from proton-positron collisions at beam energies of 820~GeV on 27.5~GeV. Photoproduction can either be direct, in which case the photon interacts directly with the parton, or resolved, in which case the photon acts as a source of quarks and gluons. A photon-proton centre of mass energy of between 134~GeV and 227~GeV is probed, with values of xP, the fractional momentum of the partons inside the proton, predominantly in the region between 0.01 and 0.1. The fractional momentum of the partons from the photon, $x\gamma$, is in the region 0.1 to 1. Jets are reconstructed in the range $-1<|\eta|<2.4$ using the kT algorithm with an R parameter of 1.0. The minimum \pT of the leading jet should be greater then 14~GeV, and at least one other jet must have \pT$>$11~GeV.

\clearpage


\section{RHIC analyses}\typeout{Handling analysis STAR_2006_S6500200}
\subsection[STAR\_2006\_S6500200]{STAR\_2006\_S6500200\,\cite{Adams:2006nd}}
\textbf{Identified hadron spectra in pp at 200 GeV}\newline
\textbf{Experiment:} STAR (RHIC pp 200 GeV) \newline
\textbf{Spires ID:} \href{http://www.slac.stanford.edu/spires/find/hep/www?rawcmd=key+6500200}{6500200}\newline
\textbf{Status:} \newline
\textbf{Authors:}
 \penalty 100
\begin{itemize}
  \item Bedanga Mohanty $\langle\,$\href{mailto:bedanga@rcf.bnl.gov}{bedanga@rcf.bnl.gov}$\,\rangle$;
  \item Hendrik Hoeth $\langle\,$\href{mailto:hendrik.hoeth@cern.ch}{hendrik.hoeth@cern.ch}$\,\rangle$;
\end{itemize}
\textbf{References:}
 \penalty 100
\begin{itemize}
  \item Phys. Lett. B637, 161
  \item nucl-ex/0601033
\end{itemize}
\textbf{Run details:}
 \penalty 100
\begin{itemize}

  \item pp at 200 GeV\end{itemize}

\noindent \pT distributions of charged pions and (anti)protons in pp collisions  at $\sqrt{s} = 200$ GeV, measured by the STAR experiment at RHIC in non-single-diffractive minbias events.

\clearpage


\clearpage

\typeout{Handling analysis STAR_2006_S6860818}
\subsection[STAR\_2006\_S6860818]{STAR\_2006\_S6860818\,\cite{Abelev:2006cs}}
\textbf{Strange particle production in pp at 200 GeV}\newline
\textbf{Experiment:} STAR (RHIC pp 200 GeV) \newline
\textbf{Spires ID:} \href{http://www.slac.stanford.edu/spires/find/hep/www?rawcmd=key+6860818}{6860818}\newline
\textbf{Status:} \newline
\textbf{Authors:}
 \penalty 100
\begin{itemize}
  \item Hendrik Hoeth $\langle\,$\href{mailto:hendrik.hoeth@cern.ch}{hendrik.hoeth@cern.ch}$\,\rangle$;
\end{itemize}
\textbf{References:}
 \penalty 100
\begin{itemize}
  \item Phys. Rev. C75, 064901
  \item nucl-ex/0607033
\end{itemize}
\textbf{Run details:}
 \penalty 100
\begin{itemize}

  \item pp at 200 GeV\end{itemize}

\noindent \pT distributions of identified strange particles in pp collisions  at $\sqrt{s} = 200$ GeV, measured by the STAR experiment at RHIC in non-single-diffractive minbias events. WARNING The $\langle \pT \rangle$ vs. particle mass plot is not validated yet and might be wrong.

\clearpage


\clearpage

\typeout{Handling analysis STAR_2006_S6870392}
\subsection[STAR\_2006\_S6870392]{STAR\_2006\_S6870392\,\cite{Abelev:2006uq}}
\textbf{Inclusive jet cross-section in pp at 200 GeV}\newline
\textbf{Experiment:} STAR (RHIC pp 200 GeV) \newline
\textbf{Spires ID:} \href{http://www.slac.stanford.edu/spires/find/hep/www?rawcmd=key+6870392}{6870392}\newline
\textbf{Status:} \newline
\textbf{Authors:}
 \penalty 100
\begin{itemize}
  \item Hendrik Hoeth $\langle\,$\href{mailto:hendrik.hoeth@cern.ch}{hendrik.hoeth@cern.ch}$\,\rangle$;
\end{itemize}
\textbf{References:}
 \penalty 100
\begin{itemize}
  \item Phys. Rev. Lett. 97, 252001
  \item hep-ex/0608030
\end{itemize}
\textbf{Run details:}
 \penalty 100
\begin{itemize}

  \item pp at 200 GeV\end{itemize}

\noindent Inclusive jet cross section as a function of \pT in pp collisions  at $\sqrt{s} = 200$ GeV, measured by the STAR experiment at RHIC.

\clearpage


\clearpage

\typeout{Handling analysis STAR_2008_S7993412}
\subsection[STAR\_2008\_S7993412]{STAR\_2008\_S7993412\,\cite{Nattrass:2008tw}}
\textbf{Di-hadron correlations in d-Au at 200 GeV}\newline
\textbf{Experiment:} STAR (RHIC d-Au 200 GeV) \newline
\textbf{Spires ID:} \href{http://www.slac.stanford.edu/spires/find/hep/www?rawcmd=key+7993412}{7993412}\newline
\textbf{Status:} \newline
\textbf{Authors:}
 \penalty 100
\begin{itemize}
  \item Christine Nattrass $\langle\,$\href{mailto:christine.nattrass@yale.edu}{christine.nattrass@yale.edu}$\,\rangle$;
  \item Hendrik Hoeth $\langle\,$\href{mailto:hendrik.hoeth@cern.ch}{hendrik.hoeth@cern.ch}$\,\rangle$;
\end{itemize}
\textbf{References:}
 \penalty 100
\begin{itemize}
  \item arXiv: \href{http://arxiv.org/abs/0809.5261}{0809.5261}
\end{itemize}
\textbf{Run details:}
 \penalty 100
\begin{itemize}

  \item d-Au at 200 GeV (use pp Monte Carlo! See description)\end{itemize}

\noindent Correlation in $\eta$ and $\phi$ between the charged hadron with the highest \pT (``trigger particle'') and the other charged hadrons in the event (``associated particles''). The data was collected in d-Au collisions at 200 GeV. Nevertheless, it is very proton-proton like and can therefore be compared to pp Monte Carlo (not for tuning, but for qualitative studies.)

\clearpage


\clearpage

\typeout{Handling analysis STAR_2009_UE_HELEN}
\subsection{STAR\_2009\_UE\_HELEN}
\textbf{UE measurement in pp at 200 GeV}\newline
\textbf{Experiment:} STAR (RHIC pp 200 GeV) \newline
\textbf{Spires ID:} \href{http://www.slac.stanford.edu/spires/find/hep/www?rawcmd=key+None}{None}\newline
\textbf{Status:} \newline
\textbf{Authors:}
 \penalty 100
\begin{itemize}
  \item Helen Caines $\langle\,$\href{mailto:helen.caines@yale.edu}{helen.caines@yale.edu}$\,\rangle$;
  \item Hendrik Hoeth $\langle\,$\href{mailto:hendrik.hoeth@cern.ch}{hendrik.hoeth@cern.ch}$\,\rangle$;
\end{itemize}
\textbf{References:}
 \penalty 100
\begin{itemize}
  \item arXiv: \href{http://arxiv.org/abs/0910.5203}{0910.5203}
  \item arXiv: \href{http://arxiv.org/abs/0907.3460}{0907.3460}
  \item WARNING! Mark as "STAR preliminary" and contact authors when using it!
\end{itemize}
\textbf{Run details:}
 \penalty 100
\begin{itemize}

  \item pp at 200 GeV\end{itemize}

\noindent UE analysis similar to Rick Field's leading jet analysis. SIScone with radius/resolution parameter R=0.7 is used. Particles with $\pT > 0.2~\text{GeV}$ and $|\eta| < 1$ are included in the analysis. All particles are assumed to have zero mass. Only jets with neutral energy $< 0.7$ are included. For the transMIN and transMAX $\Delta(\phi)$ is between $\pi/3$ and $2\pi/3$, and $\Delta(\eta) < 2.0$. For the jet region the area of the jet is used for the normalization, i.e. the scaling factor is $\pi R^2$ and not  $\mathrm{d}\phi\mathrm{d}\eta$ (this is different from what Rick Field does!).  The tracking efficiency is $\sim 0.8$, but that is an approximation,  as below $\pT \sim 0.6~\text{GeV}$ it is falling quite steeply.

\clearpage


\section{Monte Carlo analyses}\typeout{Handling analysis MC_DIJET}
\subsection{MC\_DIJET}
\textbf{Analysis of dijet events at the LHC.}\newline
\textbf{Status:} \newline
\textbf{No authors listed}\\ 
\textbf{No references listed}\\ 
\textbf{Run details:}
 \penalty 100
\begin{itemize}

  \item Generic QCD events at any energy.\end{itemize}

\noindent Analysis of dijet events for the upcoming runs at the LHC, specifically studying azimuthal angle, transverse momentum distributions (including for leading jet and secondary jet), as well as charged particle multiplicities and transverse momenta.

\clearpage


\clearpage

\typeout{Handling analysis MC_DIPHOTON}
\subsection{MC\_DIPHOTON}
\textbf{Monte Carlo validation observables for diphoton production at LHC}\newline
\textbf{Status:} \newline
\textbf{Authors:}
 \penalty 100
\begin{itemize}
  \item Frank Siegert $\langle\,$\href{mailto:frank.siegert@durham.ac.uk}{frank.siegert@durham.ac.uk}$\,\rangle$;
\end{itemize}
\textbf{No references listed}\\ 
\textbf{Run details:}
 \penalty 100
\begin{itemize}

  \item LHC pp \ensuremath{\to} jet+jet, photon+jet, photon+photon, all with EW+QCD shower\end{itemize}

\noindent Different observables related to the two photons

\clearpage


\clearpage

\typeout{Handling analysis MC_JETS}
\subsection{MC\_JETS}
\textbf{Monte Carlo validation observables for jet production}\newline
\textbf{Status:} \newline
\textbf{Authors:}
 \penalty 100
\begin{itemize}
  \item Frank Siegert $\langle\,$\href{mailto:frank.siegert@durham.ac.uk}{frank.siegert@durham.ac.uk}$\,\rangle$;
\end{itemize}
\textbf{No references listed}\\ 
\textbf{Run details:}
 \penalty 100
\begin{itemize}

  \item Pure QCD jet production events at an arbitrary collider.\end{itemize}

\noindent Jets with $p_\perp>20$ GeV are constructed with a $k_\perp$ jet finder with $D=0.7$ and projected onto many different observables.

\clearpage


\clearpage

\typeout{Handling analysis MC_LEADINGJETS}
\subsection{MC\_LEADINGJETS}
\textbf{Underlying event in leading jet events, extended to LHC}\newline
\textbf{Status:} \newline
\textbf{Authors:}
 \penalty 100
\begin{itemize}
  \item Andy Buckley $\langle\,$\href{mailto:andy.buckley@cern.ch}{andy.buckley@cern.ch}$\,\rangle$;
\end{itemize}
\textbf{No references listed}\\ 
\textbf{Run details:}
 \penalty 100
\begin{itemize}

  \item LHC pp QCD interactions at 0.9, 10 or 14 TeV. Particles with  $c \tau > 10$ mm should be set stable. Several $p_\perp^\text{min}$  cutoffs are probably required to fill the profile histograms.\end{itemize}

\noindent Rick Field's measurement of the underlying event in leading jet events, extended to the LHC. As usual, the leading jet of the defines an azimuthal toward/transverse/away decomposition, in this case the event is accepted within $|\eta| < 2$, as in the CDF 2008 version of the analysis. Since this isn't the Tevatron, I've chosen to use $k_\perp$ rather than midpoint jets.

\clearpage


\clearpage

\typeout{Handling analysis MC_PHOTONJETS}
\subsection{MC\_PHOTONJETS}
\textbf{Monte Carlo validation observables for photon + jets production}\newline
\textbf{Status:} \newline
\textbf{Authors:}
 \penalty 100
\begin{itemize}
  \item Frank Siegert $\langle\,$\href{mailto:frank.siegert@durham.ac.uk}{frank.siegert@durham.ac.uk}$\,\rangle$;
\end{itemize}
\textbf{No references listed}\\ 
\textbf{Run details:}
 \penalty 100
\begin{itemize}

  \item Tevatron Run II ppbar \ensuremath{\to} gamma + jets.\end{itemize}

\noindent Different observables related to the photon and extra jets.

\clearpage


\clearpage

\typeout{Handling analysis MC_PHOTONJETUE}
\subsection{MC\_PHOTONJETUE}
\textbf{Study the usual underlying event observables in photon + jet events}\newline
\textbf{Status:} \newline
\textbf{Authors:}
 \penalty 100
\begin{itemize}
  \item Andy Buckley $\langle\,$\href{mailto:andy.buckley@cern.ch}{andy.buckley@cern.ch}$\,\rangle$;
\end{itemize}
\textbf{No references listed}\\ 
\textbf{Run details:}
 \penalty 100
\begin{itemize}

  \item Photon + jet events at any energy. \pT cutoff at 10 GeV advised.\end{itemize}

\noindent Modification of the MC leading jets underlying event analysis to study the UE in hard photon+jet events. This may be of interest, because the leading QCD dipole structure is different from that in either dijet or Drell-Yan hard processes. Observables are also extended to include the variation of transverse activity as a function of jet-photon balance, and using the photon rather than the jet to define the event alignment.

\clearpage


\clearpage

\typeout{Handling analysis MC_SUSY}
\subsection{MC\_SUSY}
\textbf{Validate generic SUSY events, including various lepton invariant mass}\newline
\textbf{Status:} \newline
\textbf{Authors:}
 \penalty 100
\begin{itemize}
  \item Andy Buckley $\langle\,$\href{mailto:andy.buckley@cern.ch}{andy.buckley@cern.ch}$\,\rangle$;
\end{itemize}
\textbf{No references listed}\\ 
\textbf{Run details:}
 \penalty 100
\begin{itemize}

  \item SUSY events at any energy. \pT cutoff at 10 GeV may be advised.\end{itemize}

\noindent Analysis of generic SUSY events at the LHC, based on Atlas Herwig++  validation analysis contents. Plotted are eta, phi and \pT observables  for charged tracks, photons, isolated photons, electrons, muons, and jets,  as well as various dilepton mass `edge' plots for different event selection  criteria.

\clearpage


\clearpage

\typeout{Handling analysis MC_TTBAR}
\subsection{MC\_TTBAR}
\textbf{MC analysis for ttbar studies}\newline
\textbf{Status:} \newline
\textbf{Authors:}
 \penalty 100
\begin{itemize}
  \item Holger Schulz $\langle\,$\href{mailto:hschulz@physik.hu-berlin.de}{hschulz@physik.hu-berlin.de}$\,\rangle$;
  \item Andy Buckley $\langle\,$\href{mailto:andy.buckley@cern.ch}{andy.buckley@cern.ch}$\,\rangle$;
\end{itemize}
\textbf{No references listed}\\ 
\textbf{Run details:}
 \penalty 100
\begin{itemize}

  \item * For Pythia6, set MSEL=6.
  \item For Fortran Herwig/Jimmy select IPROC=1706.\end{itemize}

\noindent This is a pure Monte Carlo study for t-tbar production.

\clearpage


\clearpage

\typeout{Handling analysis MC_WJETS}
\subsection{MC\_WJETS}
\textbf{Monte Carlo validation observables for $W[e \, \nu]$ + jets production}\newline
\textbf{Status:} \newline
\textbf{Authors:}
 \penalty 100
\begin{itemize}
  \item Frank Siegert $\langle\,$\href{mailto:frank.siegert@durham.ac.uk}{frank.siegert@durham.ac.uk}$\,\rangle$;
\end{itemize}
\textbf{No references listed}\\ 
\textbf{Run details:}
 \penalty 100
\begin{itemize}

  \item $e \, \nu$ + jets analysis.\end{itemize}

\noindent Available observables are W mass, \pT of jets 1-4, jet multiplicity, $\Delta\eta(W, \text{jet1})$, $\Delta R(\text{jet2}, \text{jet3})$,  differential jet rates 0\ensuremath{\to}1, 1\ensuremath{\to}2, 2\ensuremath{\to}3, 3\ensuremath{\to}4, integrated 0--4 jet  rates.

\clearpage


\clearpage

\typeout{Handling analysis MC_ZJETS}
\subsection{MC\_ZJETS}
\textbf{Monte Carlo validation observables for $Z[e^+ \, e^-]$ + jets production}\newline
\textbf{Status:} \newline
\textbf{Authors:}
 \penalty 100
\begin{itemize}
  \item Frank Siegert $\langle\,$\href{mailto:frank.siegert@durham.ac.uk}{frank.siegert@durham.ac.uk}$\,\rangle$;
\end{itemize}
\textbf{No references listed}\\ 
\textbf{Run details:}
 \penalty 100
\begin{itemize}

  \item $e^+ e^-$ + jets analysis. Needs mass cut on lepton pair to avoid  photon singularity, e.g. a min range of $66 < m_{ee} < 116$ GeV\end{itemize}

\noindent Available observables are Z mass, \pT of jets 1-4, jet multiplicity, $\Delta\eta(Z, \text{jet1})$, $\Delta R(\text{jet2}, \text{jet3})$,  differential jet rates 0\ensuremath{\to}1, 1\ensuremath{\to}2, 2\ensuremath{\to}3, 3\ensuremath{\to}4, integrated 0--4 jet  rates.

\clearpage


\section{Example analyses}\typeout{Handling analysis EXAMPLE}
\subsection{EXAMPLE}
\textbf{A demo to show aspects of writing a Rivet analysis}\newline
\textbf{Status:} \newline
\textbf{Authors:}
 \penalty 100
\begin{itemize}
  \item Andy Buckley $\langle\,$\href{mailto:andy.buckley@durham.ac.uk}{andy.buckley@durham.ac.uk}$\,\rangle$;
\end{itemize}
\textbf{No references listed}\\ 
\textbf{Run details:}
 \penalty 100
\begin{itemize}

  \item All event types will be accepted.\end{itemize}

\noindent This analysis is a demonstration of the Rivet analysis structure and functionality: booking histograms; the initialisation, analysis and finalisation phases; and a simple loop over event particles. It has no physical meaning, but can be used as a simple pedagogical template for writing real analyses.

\clearpage


\section{Misc. analyses}\typeout{Handling analysis BELLE_2006_S6265367}
\subsection{BELLE\_2006\_S6265367}
\textbf{Charm hadrons from fragmentation and B decays on the $\Upsilon(4S)$}\newline
\textbf{Status:} \newline
\textbf{Authors:}
 \penalty 100
\begin{itemize}
  \item Jan Eike von Seggern $\langle\,$\href{mailto:jan.eike.von.seggern@physik.hu-berlin.de}{jan.eike.von.seggern@physik.hu-berlin.de}$\,\rangle$;
\end{itemize}
\textbf{References:}
 \penalty 100
\begin{itemize}
  \item Phys.Rev.D73:032002,2006.
  \item arXiv: \href{http://arxiv.org/abs/hep-ex/0506068}{hep-ex/0506068}
  \item DOI: \href{http://dx.doi.org/10.1103/PhysRevD.73.032002}{10.1103/PhysRevD.73.032002}
\end{itemize}
\textbf{Run details:}
 \penalty 100
\begin{itemize}

  \item $e^+ e^-$ analysis on the $\Upsilon(4S)$ resonance, with CoM boost -- 8.0~GeV~($e^−$) and 3.5~GeV~($e^+$)\end{itemize}

\noindent Analysis of charm quark fragmentation at 10.6 GeV, based on a data sample of 103 fb collected by the Belle detector at the KEKB accelerator.  Fragmentation into charm is studied for the main charmed hadron ground states,  namely $D^0$, $D^+$, $D^+_s$ and $\Lambda_c^+$, as well as the excited states  $D^{*0}$ and $D^{*+}$. This analysis can be used to constrain charm fragmentation  in Monte Carlo generators. Additionally, we determine the average number of these charmed hadrons produced per B decay at the $\Upsilon(4S)$ resonance and measure the distribution of their production angle in $e^+ e^-$ annihilation  events and in B decays.

\clearpage


\clearpage

\typeout{Handling analysis PDG_HADRON_MULTIPLICITIES}
\subsection[PDG\_HADRON\_MULTIPLICITIES]{PDG\_HADRON\_MULTIPLICITIES\,\cite{Amsler:2008zzb}}
\textbf{Hadron multiplicities in hadronic $e^+e^-$ events}\newline
\textbf{Experiment:} PDG (Various) \newline
\textbf{Spires ID:} \href{http://www.slac.stanford.edu/spires/find/hep/www?rawcmd=key+7857373}{7857373}\newline
\textbf{Status:} \newline
\textbf{Authors:}
 \penalty 100
\begin{itemize}
  \item Hendrik Hoeth $\langle\,$\href{mailto:hendrik.hoeth@cern.ch}{hendrik.hoeth@cern.ch}$\,\rangle$;
\end{itemize}
\textbf{References:}
 \penalty 100
\begin{itemize}
  \item Phys. Lett. B, 667, 1 (2008)
\end{itemize}
\textbf{Run details:}
 \penalty 100
\begin{itemize}

  \item Hadronic events in $e+e-$ collisions\end{itemize}

\noindent Hadron multiplicities in hadronic $e^+e^-$ events, taken from Review of Particle Properties 2008, table 40.1, page 355.   Average hadron multiplicities per hadronic $e^+e^-$ annihilation event at $\sqrt{s} \approx {}$ 10, 29--35, 91, and 130--200 GeV. The numbers are averages from various experiments. Correlations of the systematic uncertainties were considered for the calculation of the averages.

\clearpage


\clearpage

\typeout{Handling analysis PDG_HADRON_MULTIPLICITIES_RATIOS}
\subsection[PDG\_HADRON\_MULTIPLICITIES\_RATIOS]{PDG\_HADRON\_MULTIPLICITIES\_RATIOS\,\cite{Amsler:2008zzb}}
\textbf{Ratios (w.r.t. $\pi^+/\pi^-$) of hadron multiplicities in hadronic $e^+e^-$ events}\newline
\textbf{Experiment:} PDG (Various) \newline
\textbf{Spires ID:} \href{http://www.slac.stanford.edu/spires/find/hep/www?rawcmd=key+7857373}{7857373}\newline
\textbf{Status:} \newline
\textbf{Authors:}
 \penalty 100
\begin{itemize}
  \item Holger Schulz $\langle\,$\href{mailto:holger.schulz@physik.hu-berlin.de}{holger.schulz@physik.hu-berlin.de}$\,\rangle$;
\end{itemize}
\textbf{References:}
 \penalty 100
\begin{itemize}
  \item Phys. Lett. B, 667, 1 (2008)
\end{itemize}
\textbf{Run details:}
 \penalty 100
\begin{itemize}

  \item Hadronic events in $e^+ e^-$ collisions\end{itemize}

\noindent Ratios (w.r.t. $\pi^+/\pi^-$) of hadron multiplicities in hadronic $e^+ e^-$ events, taken from Review of Particle Properties 2008, table 40.1, page 355.  Average hadron multiplicities per hadronic $e^+ e^-$ annihilation event at $\sqrt{s} \approx$ 10, 29--35, 91, and 130--200 GeV, normalised to the pion multiplicity. The numbers are averages from various experiments. Correlations of the systematic uncertainties were considered for the calculation of the averages.

\clearpage


\clearpage

\typeout{Handling analysis SFM_1984_S1178091}
\subsection[SFM\_1984\_S1178091]{SFM\_1984\_S1178091\,\cite{Breakstone:1983ns}}
\textbf{Charged multiplicity distribution in pp interactions at CERN ISR energies}\newline
\textbf{Experiment:} SFM (CERN ISR) \newline
\textbf{Spires ID:} \href{http://www.slac.stanford.edu/spires/find/hep/www?rawcmd=key+1178091}{1178091}\newline
\textbf{Status:} \newline
\textbf{Authors:}
 \penalty 100
\begin{itemize}
  \item Holger Schulz $\langle\,$\href{mailto:holger.schulz@physik.hu-berlin.de}{holger.schulz@physik.hu-berlin.de}$\,\rangle$;
  \item Andy Buckley $\langle\,$\href{mailto:andy.buckley@cern.ch}{andy.buckley@cern.ch}$\,\rangle$;
\end{itemize}
\textbf{References:}
 \penalty 100
\begin{itemize}
  \item Phys.Rev.D30:528,1984
\end{itemize}
\textbf{Run details:}
 \penalty 100
\begin{itemize}

  \item QCD events, double-diffractive events should be turned on as well.\end{itemize}

\noindent Charged multiplicities are measured at \ensuremath{\sqrt{s}} = 30.4, 44.5, 52.2  and 62.2 GeV using a minimum-bias trigger. The data is sub-divided  into inleastic as well as non-single-diffractive events. However,  the implementation of the diffractive events will require some work.

\clearpage


