As with many things, Rivet may be meaningfully approached at several distinct
levels of detail:

\begin{itemize}
\item The simplest, and we hope the most common, is to use the analyses which
  are already in the library to study events from a variety of generators and
  tunes: this is enormously valuable in itself and we encourage all manner of
  experimentalists and phenomenologists alike to use Rivet in this mode.
\item A more involved level of usage is to write your own Rivet analyses ---
  this may be done without affecting the installed standard analyses by use of a
  ``plugin'' system (although we encourage users who develop analyses to submit
  them to the Rivet developers for inclusion into a future release of the main
  package). This approach requires some understanding of programming within
  Rivet but you don't \emph{need} to know about exactly what the system is doing
  with the objects that you have defined.
\item Finally, Rivet developers and people who want to do non-standard things
  with their analyses will need to know something about the messy details of
  what Rivet's infrastructure is doing behind the scenes. But you'd probably
  rather be doing some physics!
\end{itemize}

The current part of this manual is for the first sort of user, who wants to get
on with studying some observables with a generator or tune, or comparing several
such models. Since everyone will fall into this category at some point, our
present interest is to get you to that all-important ``physics plots'' stage as
quickly as possible. Analysis authors and Rivet service-mechanics will find the
more detailed information that they crave in Part~\ref{part:writinganalyses}.


\section{Quickstart}

The point of this section is to get you up and running with Rivet as soon as
possible. Doing this by hand may be rather frustrating, as Rivet depends on
several external libraries --- you'll get bored downloading and building them by
hand in the right order. Here we recommend two much simpler ways --- for the
full details of how to build Rivet by hand, please consult the Rivet Web page.


\paragraph{Ubuntu/Debian package archive}

A selection of HEP packages, including Rivet, are maintained as Debian/Ubuntu
Linux packages on the Launchpad PPA system:
\url{https://launchpad.net/~hep/+archive}. This is the nicest option for
Debian/Ubuntu, since not only will it work more easily than anything else, but
you will also automatically benefit from bug fixes and version upgrades as they
appear.

The PPA packages have been built as binaries for a variety of architectures, and
the package interdependencies are automatically known and used: all you need to
do on a Debian-type Linux system (Ubuntu included) is to add the Launchpad
archive address to your APT sources list and then request installation of the
\kbd{rivet} package in the usual way. See the Launchpad and system documentation
for all the details.


\paragraph{Bootstrap script}

For those not using Debian/Ubuntu systems, we have written a bootstrapping
script which will download tarballs of Rivet, AGILe and the other required
libraries, expand them and build them in the right order with the correct build
flags. This is generally nicer than doing it all by hand, and virtually
essential if you want to use the existing versions of FastJet, HepMC, generator
libraries, and so on from CERN AFS: there are issues with these versions which
the script works around, which you won't find easy to do yourself.

To run the script, we recommend that you choose a personal installation
directory. Personally, I make a \kbd{\home/local} directory for this purpose, to
avoid polluting my home directory with a lot of files. If you already use a
directory of the same name, you might want to use a separate one, say
\kbd{\home/rivetlocal}, such that if you need to delete everything in the
installation area you can do so without difficulties.

Now, change directory to your build area (you may also want to make this,
e.g. \kbd{\home/build}), and download the script:\\
\inp{wget \url{http://svn.hepforge.org/rivet/bootstrap/rivet-bootstrap}}\\
\inp{chmod +x rivet-bootstrap}\\
Now run it to get some help:
\inp{./rivet-bootstrap --help}\\
Now to actually do the install: for example, to bootstrap Rivet and AGILe
to the install area specified as the prefix argument, run this:\\
\inp{./rivet-bootstrap --install-agile --prefix=\val{localdir}}

If you are running on a system where the CERN AFS area is mounted as
\path{/afs/cern.ch}, then the bootstrap script will attempt to use the pre-built
HepMC\cite{Dobbs:2001ck}, LHAPDF\cite{Whalley:2005nh},
FastJet\cite{Cacciari:2005hq,fastjetweb} and GSL libraries from the LCG software area.
Either way, finally the bootstrap script will write out a file containing the
environment settings which will make the system useable. You can source this
file, e.g. \kbd{source rivetenv.sh} to make your current shell ready-to-go
for a Rivet run (use \kbd{rivetenv.csh} if you are a C shell user).

You now have a working, installed copy of the Rivet and AGILe libraries, and the
\kbd{rivet} and \kbd{agile-runmc} executables: respectively these are the
command-line frontend to the Rivet analysis library, and a convenient steering
command for generators which do not provide their own main program with HepMC
output. To test that they work as expected, source the setup scripts as above,
if you've not already done so, and run this:\\
\inp{rivet --help}\\
%
This should print a quick-reference user guide for the \kbd{rivet} command to
the terminal. Similarly, for \kbd{agile-runmc},\\
\inp{agile-runmc --help}\\
\inp{agile-runmc --list-gens}\\
\inp{agile-runmc --beams=pp:14000 Pythia6:425}\\
which should respectively print the help, list the available generators and make
10 LHC-type events using the Fortran Pythia\cite{Sjostrand:2006za} 6.423 generator. You're on your
way! If no generators are listed, you probaby need to install a local
Genser-type generator repository: see \SectionRef{sec:genser}.

In this manual, because of its convenience, we will use \kbd{agile-runmc} as our
canonical way of producing a stream of HepMC event data; if your interest is in
running a generator like Sherpa\cite{Gleisberg:2008ta},
Pythia~8\cite{Sjostrand:2007gs,Sjostrand:2008vc}, or Herwig++\cite{Bahr:2008pv}
which provides their own native way to make HepMC output, or a generator like
PHOJET which is not currently supported by AGILe, then substitute the
appropriate command in what follows.  We'll discuss using these commands in
detail in \SectionRef{sec:agile-runmc}.


\subsection{Getting generators for AGILe}
\label{sec:genser}

One last thing before continuing, though: the generators themselves. Again, if
you're running on a system with the CERN LCG AFS area mounted, then
\kbd{agile-runmc} will attempt to automatically use the generators packaged by the
LCG Genser team.

Otherwise, you'll have to build your own mirror of the LCG generators. This
process is evolving with time, and so, rather than provide information in this
manual which will be outdated by the time you read it, we simply refer you to
the relevant page on the Rivet wiki:
\url{http://rivet.hepforge.org/trac/wiki/GenserMirror}.

If you are interested in using a generator not currently supported by AGILe,
which does not output HepMC events in its native state, then please contact the
authors (via the Rivet developer contact email address) and hopefully we can
help.


\subsection{Command completion}

A final installation point worth considering is using the supplied bash-shell
programmable completion setup for the \kbd{rivet} and \kbd{agile-runmc}
commands. Despite being cosmetic and semi-trivial, programmable completion makes
using \kbd{rivet} positively pleasant, especially since you no longer need to
remember the somewhat cryptic analysis names\footnote{Standard Rivet analyses
  have names which, as well as the publication date and experiment name,
  incorporate the 8-digit Spires ID code.}!

To use programmable completion, source the appropriate files from the install
location:\\
\inp{. \val{localdir}/share/Rivet/rivet-completion}\\
\inp{. \val{localdir}/share/AGILe/agile-completion}\\
(if you are using the setup script \kbd{rivetenv.sh} this is automatically
done for you).
If there is already a \kbd{\val{localdir}/etc/bash_completion.d} directory in
your install path, Rivet and AGILe's installation scripts will install extra
copies into that location, since automatically sourcing all completion files in
such a path is quite standard.

Apologies to \{C,k,z,\dots\}-shell users, but this feature is currently only
available for the \kbd{bash} shell. Anyone who feels like supplying fixes or
additions for their favourite shell is very welcome to get in touch with the
developers.



\section{Running Rivet analyses}
\label{sec:agile-runmc}

The \kbd{rivet} executable is the easiest way to use Rivet, and will be our
example throughout this manual. This command reads HepMC events in the standard
ASCII format, either from file or from a text stream.

\subsection{The FIFO idiom}
\label{sec:fifo-idiom}

Since you rarely want to store simulated HepMC events and they are
computationally cheap to produce (at least when compared to the remainder of
experiment simulation chains), we recommend using a Unix \emph{named pipe} (or
``FIFO'' --- first-in, first-out) to stream the events. While this may seem
unusual at first, it is just a nice way of ``pretending'' that we are writing to
and reading from a file, without actually involving any slow disk access or
building of huge files: a 1M event LHC run would occupy $\sim 60 GB$ on disk,
and typically it takes twice as long to make and analyse the events when the
filesystem is involved! Here is an example:\\
\inp{mkfifo fifo.hepmc}\\
\inp{agile-runmc Pythia6:425 -o fifo.hepmc \&}\\
\inp{rivet -a EXAMPLE fifo.hepmc}\\
%
Note that the generator process (\kbd{agile-runmc} in this case) is
\emph{backgrounded} before \kbd{rivet} is run.

Notably, \kbd{mkfifo} will not work if applied to a directory mounted via the
AFS distributed filesystem, as widely used in HEP. This is not a big problem:
just make your FIFO object somewhere not mounted via AFS, e.g. \kbd{/tmp}. There
is no performance penalty, as the filesystem object is not written to during the
streaming process.

In the following command examples, we will assume that a generator has been set
up to write to the \kbd{fifo.hepmc} FIFO, and just list the \kbd{rivet} command
that reads from that location. Some typical \kbd{agile-runmc} commands are
listed in \AppendixRef{app:agilerunmc}.


\subsection{Analysis status}

The standard Rivet analyses are divided into four status classes: validated,
preliminary, obsolete, and unvalidated (in roughly decreasing order of academic
acceptability).

The Rivet ``validation procedure'' is not (as of February 2011) formally
defined, but generally implies that an analysis has been checked to ensure
reproduction of MC points shown in the paper where possible, and is believed to
have no outstanding issues with analysis procedure or cuts.  Additionally,
analyses marked as ``validated'' and distributed with Rivet should normally have
been code-checked by an experienced developer to ensure that the code is a good
example of Rivet usage and is not more complex than required or otherwise
difficult to read or maintain. Such analyses are regarded as fully ready for use
in any MC validation or tuning studies.

Validated analyses which implement an unfinished piece of experimental work are
considered to be trustworthy in their implementation of a conference note or
similar ``informal'' publication, but do not have the magic stamp of approval
that comes from a journal publication. This remains the standard mark of
experimental respectability and accordingly we do not include such analyses in
the Rivet standard analysis libraries, but in a special ``preliminary''
library. While preliminary analyses may be used for physics studies, please be
aware of the incomplete status of the corresponding experimental study, and also
be aware that the histograms in such analyses may be renamed or removed
entirely, as may the analysis itself.

Preliminary analyses will not have a SPIRES preprint number, and hence on their
move into the standard Rivet analysis library they will normally undergo a name
change: please ensure when you upgrade between Rivet versions that any scripts
or programs which were using preliminary analyses are not broken by the
disappearance or change of that analysis in the newer version. The minor perils
of using preliminary analyses can be avoided by the cautious by building Rivet
with the \kbd{-{}-disable-preliminary} configuration flag, in which case their
temptation will not even be offered.

To make transitions between Rivet versions more smooth and predictable for users
of preliminary analyses, preliminary analyses which are superseded by a
validated version will be reclassified as obsolete and will be retained for one
major version of Rivet with a status of "obsolete" before being removed, to give
users time to migrate their run scripts, i.e. if an analysis is marked as
obsolete in version 1.4.2, it will remain in Rivet's distribution until version
1.5.0.  Obsolete analyses may have different reference histograms from the final
version and will not be maintained. Obsolete analyses will not be built if
either the \kbd{-{}-disable-obsolete} configuration flag is specified at build
time: for convenience, the default value of this flag is the value of the
\kbd{-{}-disable-preliminary} flag.

Finally, unvalidated analyses are those whose implementation is incomplete,
flawed or just troubled by doubts. Running such analyses is not a good idea if
you aren't trying to fix them, and Rivet's command line tools will print copious
warning messages if you do. Unvalidated analyses in the Rivet distribution are
not built by default, as they are only of interest to developers and would be
distracting clutter for the majority of users: if you \emph{really} need them,
building Rivet with the \kbd{-{}-enable-unvalidated} configuration flag will
slake your thirst for danger.


\subsection{Example \kbd{rivet} commands}

\begin{itemize}

\item \textbf{Getting help: }{\kbd{rivet --help} will print a (hopefully)
    helpful list of options which may be used with the \kbd{rivet} command, as
    well as other information such as environment variables which may affect the
    run.}

\item \textbf{Choosing analyses: }{\kbd{rivet --list-analyses} will list the
    available analyses, including both those in the Rivet distribution and any
    plugins which are found at runtime. \kbd{rivet --show-analysis \val{patt}}
    will show a lot of details about any analyses whose name match the
    \val{patt} regular expression pattern --- simple bits of analysis name are a
    perfectly valid subset of this. For example, \kbd{rivet --show-analysis
      CDF_200} exploits the standard Rivet analysis naming scheme to show
    details of all available CDF experiment analyses published in the
    ``noughties.''}

\item \textbf{Running analyses: }{\kbd{rivet -a~DELPHI_1996_S3430090
      fifo.hepmc} will run the Rivet
    \kbd{DELPHI_1996_S3430090}\cite{Abreu:1996na} analysis on the events in the
    \kbd{fifo.hepmc} file (which, from the name, is probably a filesystem named
    pipe rather than a normal \emph{file}). This analysis is the one originally
    used for the \Delphi ``\textsc{Professor}'' generator tuning. If the first
    event in the data file does not have appropriate beam particles, the
    analysis will be disabled; since there is only one analysis in this case,
    the command will exit immediately with a warning if the first event is not
    an $\Ppositron\Pelectron$ event.}

% \item \paragraph{Using all analyses:}{\kbd{rivet -n~50000 -A -} will read up to
%     50k events from standard input (specified by the special ``-'' input
%     filename) and analyse them with \emph{all} the Rivet library analyses. As
%     above, incompatible analyses will be removed before the main analysis run
%     begins.}

\item \textbf{Histogramming: }{\kbd{rivet fifo.hepmc -H~foo.aida} will read all the
    events in the \kbd{fifo.hepmc} file. The \kbd{-H} switch is used to
    specify that the output histogram file will be named \kbd{foo.aida}. By
    default the output file is called \kbd{Rivet.aida}.}

\item \textbf{Fine-grained logging: }{\kbd{rivet fifo.hepmc -A
      -l~Rivet.Analysis=DEBUG~\cmdbreak -l~Rivet.Projection=DEBUG
      -l~Rivet.Projection.FinalState=TRACE~\cmdbreak -l~NEvt=WARN}
    will analyse events as before, but will print different status
    information as the run progresses. Hierarchical logging control is possible
    down to the level of individual analyses and projections as shown above;
    this is useful for debugging without getting overloaded with debug
    information from \emph{all} the components at once. The default level is
    ``\textsc{info}'', which lies between ``\textsc{debug}'' and
    ``\textsc{warning}''; the ``\textsc{trace}'' level is for very low level
    information, and probably isn't needed by normal users.}

\end{itemize}



\section{Using analysis data}

In this section, we summarise how to use the data files which Rivet produces for
plotting, validation and tuning.

\subsection{Histogram formats}

Rivet currently produces output histogram data in the AIDA XML format. Most
people aren't familiar with AIDA (and we recommend that you remain that way!),
and it will disappear entirely from Rivet in version 2.0. You will probably
wish to cast the AIDA files to a different format for plotting, and for this we
supply several scripts.

\paragraph{Conversion to ROOT}

Your knee-jerk reaction is probably to want to know how to plot your Rivet
histograms in ROOT\cite{Antcheva:2009zz}. Don't worry: a few months of therapy
can work wonders. For unrepentant ROOT junkies, Rivet installs an
\kbd{aida2root} script, which converts the AIDA records to a \kbd{.root} file
full of ROOT \texttt{TGraph}s. One word of warning: a bug in ROOT means that
\texttt{TGraph}s do not render properly from file because the axis is not drawn by
default. To display the plots correctly in ROOT you will need to pass the
\kbd{"AP"} drawing option string to either the \kbd{TGraph::Draw()} method, or
in the options box in the \kbd{TBrowser} GUI interface.

\paragraph{Conversion to ``flat format''}

Most of our histogramming is based around a ``flat'' plain text format,
which can easily be read (and written) by hand. We provide a script called
\kbd{aida2flat} to do this conversion. Run \kbd{aida2flat -h} to get usage
instructions; in particular the Gnuplot and ``split output'' options are useful
for further visualisation. Aside from anything else, this is useful for simply
checking the contents of an AIDA file, with \kbd{aida2flat Rivet.aida | less}.

\vspace{1.8em}

\begin{detail}
  We get asked a lot about why we don't use ROOT internally: aside from a
  general unhappiness about the design and quality of the data objects in ROOT,
  the monolithic nature of the system makes it a big dependency for a system as
  small as Rivet. While not an issue for experimentalists, most theorists and
  generator developers do not use ROOT and we preferred to embed the AIDA
  system, which in its LWH implementation requires no external package. The
  replacement for AIDA will be another lightweight system rather than ROOT, with
  an emphasis on friendly, intuitive data object design, and correct handling of
  sample merging statistics for all data objects.
\end{detail}

\subsection{Chopping histograms}
\newcommand{\chophisto}{\kbd{rivet-chopbins}\xspace} In some cases you don't
want to keep the complete histograms produced by Rivet.  For generator tuning
purposes, for example, you want to get rid of the bins you already know your
generator is incapable of describing. You can use the script \chophisto to
specify those bin-ranges you want to keep individually for each histogram in a
Rivet output-file. The bin-ranges have to be specified using the corresponding
x-values of that histogram.  The usage is very simple. You can specify bin
ranges of histograms to keep on the command-line via the \kbd{-b}
switch, which can be given multiple times, e.g.\\
\kbd{\chophisto -b /CDF\_2001\_S4751469/d03-x01-y01:5:13 Rivet.aida}\\
%
will chop all bins with $x<5$ and $x>13$ from the histogram
\kbd{/CDF\_2001\_S4751469/d03\-x01\-y01} in the file \kbd{Rivet.aida}. (In
this particular case, $x$ would be a leading jet \pT.)

\subsection{Normalising histograms}
\newcommand{\normhisto}{\kbd{rivet-rescale }} Sometimes you want to
use histograms normalised to, e.g., the generator cross-section or the area of
a reference-data histogram. The script \normhisto was designed for these
purposes. The usage is the following:\\
\kbd{\normhisto -O observables -r RIVETDATA -o normalised Rivet.aida}\\
%
By default, the normalised histograms are written to file in the AIDA-XML
format. You can also give the \kbd{-f} switch on the command line to produce
flat histograms.

\paragraph{Normalising to reference data} You will need an output-file of
Rivet, \kbd{Rivet.aida}, a folder that contains the reference-data histograms
(e.g. \kbd{rivet-config --datadir}) and optionally, a text-file,
\kbd{observables} that contains the names of the histograms you would like to
normalise - those not given in the file will remain un-normalised. These
are examples of how your \kbd{observables} file might look like:
%
\begin{snippet}
/CDF_2000_S4155203/d01-x01-y01
\end{snippet}

If a histogram \kbd{/CDF\_2000\_S4155203/d01-x01-y01} is found in one of the
reference-data files in the folder specified via the \kbd{-r} switch, then this
will result in a histogram \kbd{/CDF\_2000\_S4155203/d01-x01-y01} being
normalised to the area of the corresponding reference-data histogram.  You can
further specify a certain range of bins to normalise:
%
\begin{snippet}
/CDF_2000_S4155203/d01-x01-y01:2:35
\end{snippet}
%
\noindent will chop off the bins
with $x<2$ and $x>35$ of both, the histogram in your \kbd{Rivet.aida} and the
reference-data histogram. The remaining MC histogram is then normalised to the
remaining area of the reference-data histogram.

\paragraph{Normalising to arbitrary areas}%
In the file \kbd{observables} you
can further specify an arbitrary number, e.g. a generator cross-section, as
follows:
%
\begin{snippet}
/CDF_2000_S4155203/d01-x01-y01  1.0
\end{snippet}
\noindent will result in the histogram \kbd{/CDF\_2000\_S4155203/d01-x01-y01} being
normalised to 1.0, and
%
\begin{snippet}
/CDF_2000_S4155203/d01-x01-y01:2:35  1.0
\end{snippet}
%
\noindent will chop off the bins with $x<2$ and $x>35$ of the histogram\\
\kbd{/CDF\_2000\_S4155203/d01-x01-y01} first and normalise the remaining
histogram to one.


\subsection{Plotting and comparing data}
Rivet comes with three commands --- \kbd{rivet-mkhtml}, \kbd{compare-histos} and
\kbd{make-plots} --- for comparing and plotting data files. These commands
produce nice comparison plots of publication quality from the AIDA format text
files.

The high level program \kbd{rivet-mkhtml} will automatically create a plot
webpage from the given AIDA files. It searches for reference data automatically
and uses the other two commands internally. Example:\\
\inp{rivet-mkhtml withUE.aida:'Title=With UE' withoutUE.aida:'LineColor=blue'}\\
Run \kbd{rivet-mkhtml --help} to find out about all features and options.

You can also run the other two commands separately:
%
\begin{itemize}
\item \kbd{compare-histos} will accept a number of AIDA files as input (ending in
\kbd{.aida}), identify which plots are available in them, and combine the MC
and reference plots appropriately into a set of plot data files ending with
\kbd{.dat}. More options are described by running \kbd{compare-histos --help}.

Incidentally, the reference files for each Rivet analysis are to be found in the
installed Rivet shared data directory, \kbd{\val{installdir}/share/Rivet}. You
can find the location of this by using the \kbd{rivet-config} command:\\
\inp{rivet-config --datadir}

\item You can plot the created data files using the \kbd{make-plots} command:\\
\inp{make-plots --pdf *.dat}\\
The \kbd{--pdf} flag makes the output plots in PDF format: by default the output
is in PostScript (\kbd{.ps}), and flags for conversion to EPS and PNG are also
available.
\end{itemize}
