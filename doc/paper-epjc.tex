\RequirePackage{fix-cm}
%\documentclass{svjour3}                     % onecolumn (standard format)
%\documentclass[smallcondensed]{svjour3}     % onecolumn (ditto)
\documentclass[epjc3]{svjour3}       % onecolumn (second format)
%\documentclass[twocolumn]{svjour3}          % twocolumn
%\documentclass[preprint,12pt]{elsarticle}

\input{preamble}

\def\part#1{\section*{\large #1}}

\RequirePackage[T1]{fontenc}

\smartqed  % flush right qed marks, e.g. at end of proof

\RequirePackage{graphicx}
\RequirePackage{mathptmx}      % use Times fonts if available on your TeX system
\RequirePackage{flushend}
\RequirePackage[numbers,sort&compress]{natbib}
\RequirePackage[colorlinks,citecolor=blue,urlcolor=blue,linkcolor=blue]{hyperref}
\journalname{Eur. Phys. J. C}

\begin{document}

  
  \title{Rivet user manual}

  \author{Andy~Buckley\thanksref{inst:a}
    \and
    Jonathan~Butterworth\thanksref{inst:b}
    \and
    David~Grellscheid\thanksref{inst:c}
    \and
    Hendrik~Hoeth\thanksref{inst:c}
    \and
    Leif~L\"onnblad\thanksref{inst:d}
    \and
    James~Monk\thanksref{inst:b}
    \and
    Holger~Schulz\thanksref{inst:e}
    \and
    Frank~Siegert\thanksref{inst:f,corrauthor}}


\institute{PPE Group, School of Physics, University of Edinburgh, UK\label{inst:a}
    \and
  HEP Group, Dept. of Physics and Astronomy, UCL, London, UK\label{inst:b}
    \and
  IPPP, Durham University, UK\label{inst:c}
    \and
  Theoretical Physics, Lund University, Sweden.\label{inst:d}
    \and
  Institut f\"ur Physik, Berlin Humboldt University, Germany.\label{inst:e}
    \and
  Physikalisches Institut, Freiburg University, Germany.\label{inst:f}
  \and
  corresponding author\label{corrauthor}}

%  \cortext[author]{Corresponding author.\\\textit{E-mail address:} frank.siegert@cern.ch}


\maketitle

  \begin{abstract}
    This is the manual and user guide for the Rivet system for the
    validation and tuning of Monte Carlo event generators. As well as the core
    Rivet library, this manual describes the usage of the \kbd{rivet} program and
    the AGILe generator interface library. The depth and level of description is
    chosen for users of the system, starting with the basics of using validation
    code written by others, and then covering sufficient details to write new
    Rivet analyses and calculational components.
  \end{abstract}
  \keywords{Event generator; simulation; validation; tuning; QCD}

\section{Introduction}
\label{sec:intro}
\input{intro}

\cleardoublepage

\part{Part I: Getting started with Rivet}
\label{part:gettingstarted}
\input{gettingstarted}

\cleardoublepage

\part{Part II: How Rivet works}
\label{part:writinganalyses}
\input{writinganalyses}

\cleardoublepage

\part{Part III: Appendices}
\appendix

\section{Typical \kbd{agile-runmc} commands}
\label{app:agilerunmc}
\input{agilerunmc}

\section{Acknowledgements}
\label{app:acknowledgements}
\input{acknowledgements}


\cleardoublepage

\bibliographystyle{h-physrev3}
\bibliography{refs}

\end{document}
