\documentclass{JHEP3}
%\JHEP{00(2007)000}

\usepackage{xspace,graphicx,mparhack,amsmath,amssymb,url,underscore,fancyvrb}
\usepackage{hepnicenames,hepunits}
%\usepackage{dropping}

\newenvironment{snippet}{\Verbatim}{\endVerbatim}

\newcommand{\kbd}[1]{\texttt{#1}\xspace}
\newcommand{\inp}[1]{\textsf{\textdollar}\hspace{1mm}\texttt{#1}\xspace}
\newcommand{\outp}[1]{\textsf{#1}\xspace}
\newcommand{\code}[1]{\texttt{#1}\xspace}
\newcommand{\var}[1]{\texttt{\textdollar{}#1}\xspace}
\newcommand{\val}[1]{\textit{\ensuremath{\langle\text{\textrm{#1}\/}\rangle}}\xspace}
\newcommand{\home}{\texttt{\ensuremath{\sim}}\xspace}

\newcommand{\RGnegate}{\texttt{\ensuremath{\sim}}}

\newcommand{\cmdbreak}{\textbackslash\newline}

\newcommand{\SectionRef}[1]{section~\ref{#1}}
\newcommand{\SubsectionRef}[1]{sub-section~\ref{#1}}

\let\oldmarginpar\marginpar
\renewcommand\marginpar[1]{\-\oldmarginpar{\footnotesize \textit{#1}}}
%\renewcommand\marginpar[1]{\-\oldmarginpar[\raggedleft\footnotesize \textit{#1}]%
%{\raggedright\footnotesize \textit{#1}}}

\newcommand{\bendimg}{\includegraphics[height=1cm]{bend}}
\newcommand{\dblbendimg}{\bendimg\hspace{0.5mm}\bendimg}
\newenvironment{bend}{\bendimg}{\ignorespacesafterend}
\newenvironment{dblbend}{\dblbendimg}{\ignorespaceafterend}

\title{Rivet and AGILe manual}

\author{Andy Buckley\\ IPPP, Durham University, UK.\\ E-mail: \email{andy.buckley@durham.ac.uk}}
\author{James Monk\\ HEP Group, Dept. of Physics and Astronomy, UCL, London, UK.\\ E-mail: \email{jmonk@fnal.gov}}
\author{Lars Sonnenschein\\ CERN, Gen\`eve 1206, Switzerland.\\ E-mail: \email{sonne@cern.ch}}
\author{Jon Butterworth\\ HEP Group, Dept. of Physics and Astronomy, UCL, London, UK.\\ E-mail: \email{jmb@hep.ucl.ac.uk}}
\author{Leif L\"onnblad\\ Theoretical Physics, Lund University, Sweden.\\ E-mail: \email{lonnblad@thep.lu.se}}

\preprint{}
%\preprint{\hepth{9912999}}

\abstract{This document is the manual and user guide for the Rivet system for
  the validation and tuning of Monte Carlo event generators. As well as the core
  Rivet library, this manual describes the usage of the \kbd{rivetgun} program
  and the AGILe generator interface library. The level of description is
  generally intended for users of the system, starting with the basics of using
  validation code written by others, and then covering all the details you need
  to know to write your own Rivet components.}

\keywords{Event generator, simulation, validation, tuning, QCD}


\begin{document} 


\section{Introduction}
This manual is a users' guide to using the Rivet generator validation
system. Rivet itself is a C++ class library, with classes to represent
observable calculations, analyses and their data objects, and a collection of
infrastructure/management objects which you hopefully won't have to worry about
very much. The simplest way to use Rivet is via the \kbd{rivetgun} command line
tool, which can run a generator using the AGILe interface library, analyse the
events as they are generated, and produce output distributions in your format of
choice. For those who wish to embed their analyses in some larger framework,
Rivet can also be run programmatically on HepMC event objects with no special
executable being required.

Before we get started, a declaration of intent: this manual is intended to be a
guide to using Rivet, \emph{not} a comprehensive and painstakingly maintained
reference to the application programming interface (API) of the Rivet and AGILe
libraries. For that purpose, you will hopefully find the online generated
documentation at \url{http://projects.hepforge.org/rivet} and
\url{http://projects.hepforge.org/agile} to be ideal.

\subsection{Typographic conventions}
As is normal in computer user manuals, the typography in this manual is used to
indicate whether we are describing source code elements, commands to be run in a
terminal, the output of a command etc.

The main such clue will be the use of \kbd{typewriter-style} text: this
indicates the name of a command or code element --- class names, function names
etc. Typewriter font is also used for commands to be run in a terminal, but in
this case it will be prefixed by a dollar sign, as in \inp{echo ''Hello'' |
  cat}.  The output of such a command on the terminal will be typeset in
\outp{sans-serif} font. When we are documenting a code feature in detail (which
is not the main point of this manual), we will use square brackets to indicate
optional arguments, and italic font between angle brackets to represent an
argument name which should be replaced by a value,
e.g. \code{Event::applyProjection(\val{proj})}.

Following the example of Donald Knuth in his books on \TeX{}, in this document
we will indicate paragraphs of particular technicality or esoteric nature with a
``dangerous bend''\marginpar{\bendimg\\Dangerous bend} sign. These will typically be about the
internals of Rivet which most people will be lucky enough never to need know
about, but for detail obsessives, the inordinately curious and Rivet hackers
they may be useful. You can almost certainly skip them on a first
reading. Similarly, you may see double bend signs
\marginpar{\dblbendimg\\Double bend} --- the same rules apply for
these, but even more strongly.


\section{Quickstart}
The point of this section is to get you up and running with Rivet as soon as
possible. Doing this by hand may be rather frustrating, as Rivet depends on
several external libraries --- you'll get bored downloading and building them by
hand in the right order. To make life more pleasant, we have written a
bootstrapping script which will download tarballs of Rivet, AGILe and the other
required libraries, expand them and build them in the right order. You can get
this script from the following Web address:
\url{http://www.hepforge.org/downloads/rivet/rivet-bootstrap}

To run the script, we recommend that you choose a personal installation
directory. Personally, I make a \kbd{\home/local} directory for this purpose, to
avoid polluting my home directory with a lot of files. If you already use a
directory of the same name, you might want to use a separate one, say
\kbd{\home/rivetlocal}. You'll need to add \kbd{\val{localdir}/bin} to your
\var{PATH} environment variable and \kbd{\val{localdir}/lib} to your
\var{LD_LIBRARY_PATH}.

Now, change directory to your build area (you may also want to make this,
e.g. \kbd{\home/build}), and download the script:\\
\inp{wget \url{http://www.hepforge.org/downloads/rivet/rivet-bootstrap}}\\
Now run it, specifying the install area as the argument:\\
\inp{chmod a+x rivet-bootstrap}\\
\inp{./rivet-bootstrap \val{localdir}}\\

If you are running on a system where the CERN AFS area is mounted as
\path{/afs/cern.ch}, then the bootstrap script will attempt to use the
pre-built HepMC, LHAPDF, FastJet and GSL libraries from the LCG software area.

You should now have a working, installed copy of the Rivet and AGILe libraries,
and the \kbd{rivetgun} and \kbd{rivet-config} executables. \kbd{rivet-config} is
a simple script which can be used to determine useful paths when compiling
against Rivet. \kbd{rivetgun} is the main executable used to run generators and
process their output with Rivet analyses. We'll discuss it in detail
in \SectionRef{sec:rivetgun}.

One last thing before continuing, though: the generators themselves. Again, if
you're running on a system with the CERN LCG AFS area mounted, then
\kbd{rivetgun} will attempt to automatically use the generators packaged by the
LCG Genser team. Otherwise, you'll have to build your own mirror of the LCG
generators. This process is not standardised at the moment (this will hopefully
change), so we've provided a script, \kbd{mkGenserArea}, in the \kbd{bin}
directory of the AGILe source distribution (this will have been downloaded by
the \kbd{rivet-bootstrap} script. Make yourself a Genser installation directory,
e.g. \kbd{\var{HOME}/genser}, and \kbd{cd} into it. Then run the
\kbd{mkGenserArea} script, and wait for it all to build. Finally, set the
\var{AGILE_GEN_PATH} path variable to contain the \kbd{\val{genserDir}/liblinks}
directory: you should now have a few Fortran generators to play with.


\section{Running \kbd{rivetgun}}
\label{sec:rivetgun}
The \kbd{rivetgun} executable is the easiest way to use Rivet, and will be our
example throughout this manual. To get started, just run the command
\inp{rivetgun} with no arguments. You should get a short usage message, looking
something like this:
%
\begin{snippet}
buckley@d65:~/proj/cedar/agile\$ rivetgun
PARSE ERROR:  
             One or more required arguments missing!

Brief USAGE: 
   rivetgun  {-g <FHerwig:6510|FPythia:6413>|-G <genfile>} [-R] 
             [-o <filename>] [-n <num>] [--nocolor] [-l <name=level>] 
             [--histotype <AIDA|FLAT|ROOT>] [-H <name>]
             [-A] [-a <ALEPH_1991_S2435284|~ALEPH_1991_S2435284
             |CDF_1994_S2952106|~CDF_1994_S2952106|CDF_2001_S4751469...]
             [-P <paramfile>] ...  [-p <param=value>] ...  [--mom2 <mom>]
             [--mom1 <mom>] [--beam2 <ANTIPROTON|PIMINUS|NU_TAUBAR|...]
             [--beam1 <ANTIPROTON|PIMINUS|NU_TAUBAR|...]
             [-i <filename>] [-s <seed>] [--] [--version] [-h]

For complete USAGE and HELP type: 
   rivetgun --help
\end{snippet}

The message obtained with \kbd{rivetgun --help} is much more explicit about the
meanings of these options, but we'll survey a few of them here with concrete examples.

\begin{itemize}
\item \paragraph{Simple run:}{\kbd{rivetgun -g~FHerwig:6510 -P~lep1.params
      -n~1000} will use the Fortran Herwig 6.5.10 generator (the \kbd{-g} option
    switch) to generate 1000 events (the \kbd{-n} switch) in LEP1 mode,
    i.e. $\Ppositron\Pelectron$ collisions at $\sqrt{s} = \unit{90}{\GeV}$. The
    \kbd{-P} switch is actually the way of specifying a \kbd{rivetgun}
    parameters file: in this case, the file \kbd{lep1.params} is loaded from the
    \kbd{\val{installdir}/share/RivetGun} directory, if it isn't first found in
    the current directory.}

\item \paragraph{Parameter changes:}{\kbd{rivetgun -g~FPythia:6413
      -P~lep1.params -n~1000 \cmdbreak -p~"PARJ(82)=5.27"} will generate 1000
    events using the Fortran Pythia 6.4.13 generator, again in LEP1 mode. The
    difference this time is that we've used the \kbd{-p} (lower-case) switch to
    change a named generator parameter, here Pythia's PARJ(82), which sets the
    parton shower cutoff scale. Being able to change parameters on the command
    line is useful for scanning parameter ranges from a shell loop, or rapid
    testing of parameter values without needing to write a parameters file for
    use with~\kbd{-P}.}

\item \paragraph{Writing out HepMC events:}{\kbd{rivetgun -g~FPythia:6413
      -P~lhc.params -n~50 -o~out.hepmc -R} will generate 50 LHC events with
    Pythia. The~\kbd{-o} switch is being used here to tell \kbd{rivetgun} to
    write the generated events to the \kbd{out.hepmc} file. This file will be a
    plain text dump of the HepMC event records in the ``IO_GenEvent''
    format. The~\kbd{-R} flag is used to disable running Rivet at all: in this
    instance, \kbd{rivetgun} will produce no histogram output file and is
    basically just a flexible steering program for several generators.}

\item \paragraph{Choosing analyses:}{\kbd{rivetgun -g~FHerwig:6510
      -P~lep1.params \cmdbreak -a~DELPHI_1996_S3430090} will again generate LEP1
    events using Pythia. This time, since the \kbd{-n} switch is not present,
    \kbd{rivetgun} will fall back to its default run of 10 events. We're here
    running a Rivet analysis for the first time --- in the above runs, even
    when~\kbd{-R} wasn't specified, we weren't actually running any analyses. In
    this case, we're running the DELPHI analysis most used for Monte Carlo
    tuning.}

\item \paragraph{Using all analyses:}{\kbd{rivetgun -g~FHerwig:6510
      -P~tevatron1800.params \cmdbreak -n~ 50000 -A
      -a~\RGnegate{}CDF_1994_S2952106} will generate 50k Tevatron events at
    $\sqrt{s}_{\Pproton\Pproton} = \unit{1.8}{\TeV}$. The~\kbd{-A} switch says
    that \kbd{rivetgun} should try to run all the analyses that it can: it will
    automatically disable any analyses not compatible with the Tevatron beam
    particles. The~\kbd{-a} switch is being used here for the opposite of its
    usual reason: prefixing the analysis name with a tilde (\kbd{\RGnegate})
    \emph{removes} that analysis. Here, all Tevatron analyses \emph{other than}
    CDF_1994_S2952106 will run.}

\item \paragraph{Histogramming:}{\kbd{rivetgun -g~FPythia:6413
      -P~tevatron1800.params \cmdbreak -H~foo --histotype~ROOT} will make 10
    Tevatron events. The new features are the \kbd{-H} and \kbd{--histotype}
    switches, which influence the output data file. The \kbd{-H} chooses the
    base name of the output file, by default ``Rivet'' and \kbd{--histotype}
    chooses the format, here set to ROOT. So the output file, which in the above
    examples was \kbd{Rivet.aida}, will here be \kbd{foo.root}.}

\item \paragraph{Log level:}{\kbd{rivetgun -g~FPythia:6413 -P~lep1.params
      -l~Rivet.Analysis=DEBUG~\cmdbreak -l~Rivet.Projection=TRACE -l~RivetGun=WARNING}
    will again generate LEP1 events using Pythia. The~\kbd{-l} flags are being
    used to change the level of the logging coming from the various analysis
    classes. The first tells all Rivet analyses and associated handlers to show
    any messages at log level ``debug'' --- more than usual --- and above. The
    second tells Rivet's projections to log at ``trace'' level --- very detailed
    debug information --- and above. The last controls the log levels of the
    \kbd{rivetgun} executable itself: this can be used to suppress or enhance
    information like the event number. The default level is ``info'', which lies
    between ``debug'' and ``warning''.}

\item \paragraph{More on logging:}{\kbd{rivetgun -g~FPythia:6413 -P~lep1.params
      \cmdbreak -l~Rivet.Analysis.Handler=WARNING -l~Rivet.Analysis.Test=TRACE}
    is another demo of Rivet's logging capabilities. The logging system is
    hierarchical, so the examples above applied to \emph{all} analyses,
    \emph{all} projections, and so on. It is possible, indeed desirable, to be
    more specific, as in this example, where the log levels of most classes in
    the \kbd{Rivet.Analysis} log hierarchy is the default ``info'' level, but
    \kbd{Analysis.Handler} and \kbd{Analysis.Test} are set to different levels.}
\end{itemize}


\section{The projection/analysis model}


\section{Projections}
\subsection{Projection caching}
\begin{bend}
How projection caching works (skippable, but useful as a reference)
This section is super-hardcore
\end{bend}
\subsection{Standard projection summary}
\subsection{Example projection}


\section{Analyses}
\subsection{Analysis histo autobinning}
\subsection{Pluggable analyses}
\subsection{Example analysis}

\section{Viewing output data files}

\section{Comparing with reference data}


% \begin{thebibliography}{999}
% \end{thebibliography}

\end{document}
