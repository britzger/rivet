\documentclass{JHEP3}
%\JHEP{00(2007)000}

\usepackage{xspace}
\newcommand{\kbd}[1]{\texttt{#1}\xspace}


\title{Rivet and AGILe manual}

\author{
Andy Buckley\\ IPPP, Durham University, UK.\\ E-mail: \email{andy.buckley@durham.ac.uk}\\
James Monk\\ HEP Group, Dept. of Physics and Astronomy, UCL, London, UK.\\ E-mail: \email{jmonk@fnal.gov}\\
Lars Sonnenschein\\ CERN, Gen\`eve 1206, Switzerland.\\ E-mail: \email{sonne@cern.ch}\\
Jon Butterworth\\ HEP Group, Dept. of Physics and Astronomy, UCL, London, UK.\\ E-mail: \email{jmb@hep.ucl.ac.uk}\\
Leif L\"onnblad\\ Theoretical Physics, Lund University, Sweden.\\ E-mail: \email{lonnblad@thep.lu.se}%
}

%\preprint{\hepth{9912999}}

\abstract{This document is the manual and user guide for the Rivet system for
  the validation and tuning of Monte Carlo event generators. As well as the core
  Rivet library, this manual describes the usage of the \kbd{rivetgun} program
  and the AGILe generator interface library. The level of description is
  generally intended for users of the system, starting with the basics of using
  validation code written by others, and then covering all the details you need
  to know to write your own Rivet components.}

\keywords{TeX, LaTeX, JHEP}

%\dedicated{Dedicated to\ldots\\if you want.}


\begin{document} 


\section{Introduction}

Following the innovation of Donald Knuth in his books on \TeX{}, in this
document we will indicate paragraphs of particular technicality or esoteric
nature with a ``dangerous bend'' sign {IMAGE IN MARGIN}. These will typically be
about the internals of Rivet which most people will be lucky enough never to
need know about, but for detail obsessives, the inordinately curious and Rivet
hackers they may be useful. You can almost certianly skip them on a first
reading of this manual. Similarly, you may see double bend signs {IMAGE IN
  MARGIN} --- the same rules apply for these, but even more strongly.

OTHER DOCUMENT CONVENTIONS: kbd, italics, replacements


\section{Running \kbd{rivetgun}}

\section{The projection/analysis model}
\subsection{How projection caching works (skippable, but useful as a reference)}
\section{Analysis histo autobinning}
\section{Pluggable analyses}
\section{Standard projection summary}
\section{Example projection}
\section{Example analysis}
\section{Viewing output data files}
\section{Comparing with reference data (qualitative)}


% \begin{thebibliography}{999}
% \end{thebibliography}

\end{document}
