\documentclass{JHEP3}
%\JHEP{00(2007)000}

\include{preamble}

\title{Rivet user manual\\ {\smaller \textsc{version \RivetVersion}}}

\author{Andy Buckley\\ IPPP, Durham University, UK.\\ E-mail: \email{andy.buckley@durham.ac.uk}}
\author{Jonathan Butterworth\\ HEP Group, Dept. of Physics and Astronomy, UCL, London, UK.\\ E-mail: \email{J.Butterworth@ucl.ac.uk}}
\author{Leif L\"onnblad\\ Theoretical Physics, Lund University, Sweden.\\ E-mail: \email{lonnblad@thep.lu.se}}
\author{Hendrik Hoeth\\ Theoretical Physics, Lund University, Sweden.\\ E-mail: \email{hendrik.hoeth@cern.ch}}
\author{James Monk\\ HEP Group, Dept. of Physics and Astronomy, UCL, London, UK.\\ E-mail: \email{jmonk@hep.ucl.ac.uk}}
\author{Frank Siegert\\ IPPP, Durham University, UK.\\ E-mail: \email{frank.siegert@durham.ac.uk}}
\author{Lars Sonnenschein\\ CERN, Gen\`eve 1206, Switzerland.\\ E-mail: \email{sonne@cern.ch}}

\preprint{}
%\preprint{\hepth{9912999}}

\abstract{This is the manual and user guide for the Rivet system for the
  validation and tuning of Monte Carlo event generators. As well as the core
  Rivet library, this manual describes the usage of the \kbd{rivet} program and
  the AGILe generator interface library. The depth and level of description is
  chosen for users of the system, starting with the basics of using validation
  code written by others, and then covering sufficient details to write new
  Rivet analyses and calculational components.}

\keywords{Event generator, simulation, validation, tuning, QCD}


\begin{document} 


\section{Introduction}

This manual is a users' guide to using the Rivet generator validation
system. Rivet is a C++ class library, which provides the infrastructure and
calculational tools for simulation-level analyses, enabling physicists to
validate event generator models and tunings with minimal effort and maximum
portability. Rivet is designed to scale effectively to large numbers of analyses
for truly global validation, by transparent use of an automated result caching
system.

The Rivet ethos, if it may be expressed succinctly, is that user analysis code
should be extremely clean and easy to write --- ideally it should be
sufficiently self-explanatory to in itself be a reference to the experimental
analysis algorithm --- without sacrificing power or extensibility. The machinery
to make this possible is intentionally hidden from the view of all but the most
prying users. Generator independence is explicitly required by virtue of all
analyses operating on the generic ``HepMC'' event record.

The simplest way to use Rivet is via the \kbd{rivet} command line tool, which
analyses textual HepMC event records as they are generated and produces output
distributions in a structured textual format. The input events are generated
using the generator's own steering program, if one is provided; for generators
which provide no default way to produce HepMC output, the AGILe generator
interface library, and in particular the \kbd{agile-runmc} command which it
provides, may be useful. For those who wish to embed their analyses in some
larger framework, Rivet can also be run programmatically on HepMC event objects
with no special executable being required.

Before we get started, a declaration of intent: this manual is intended to be a
guide to using Rivet, rather than a comprehensive and painstakingly maintained
reference to the application programming interface (API) of the Rivet
library. For that purpose, you will hopefully find the online generated
documentation at \url{http://projects.hepforge.org/rivet} to be
sufficient. Similar API documentation is maintained for AGILe at
\url{http://projects.hepforge.org/agile}.


\subsection{Typographic conventions}
As is normal in computer user manuals, the typography in this manual is used to
indicate whether we are describing source code elements, commands to be run in a
terminal, the output of a command etc.

The main such clue will be the use of \kbd{typewriter-style} text: this
indicates the name of a command or code element --- class names, function names
etc. Typewriter font is also used for commands to be run in a terminal, but in
this case it will be prefixed by a dollar sign, as in \inp{echo ''Hello'' |
  cat}.  The output of such a command on the terminal will be typeset in
\outp{sans-serif} font. When we are documenting a code feature in detail (which
is not the main point of this manual), we will use square brackets to indicate
optional arguments, and italic font between angle brackets to represent an
argument name which should be replaced by a value,
e.g. \code{Event::applyProjection(\val{proj})}.

Following the example of Donald Knuth in his books on \TeX{}, in this document
we will indicate paragraphs of particular technicality or esoteric nature with a
``dangerous bend''\marginpar{\bendimg\\Dangerous bend} sign. These will
typically describe internals of Rivet of which most people will be fortunate
enough to remain happily ignorant without adverse effects. However they may be
of interest to detail obsessives, the inordinately curious and Rivet
hackers. You can certainly skip them on a first reading. Similarly, you may see
double bend signs \marginpar{\dblbendimg\\Double bend} --- the same rules apply
for these, but even more strongly.


\cleardoublepage
\part{Getting started with Rivet}
\label{part:gettingstarted}

As with many things, Rivet may be meaningfully approached at several distinct
levels of detail:

\begin{itemize}
\item The simplest, and we hope the most common, is to use the analyses which
  are already in the library to study events from a variety of generators and
  tunes: this is enormously valuable in itself and we encourage all manner of
  experimentalists and phenomenologists alike to use Rivet in this mode.
\item A more involved level of usage is to write your own Rivet analyses ---
  this may be done without affecting the installed standard analyses by use of a
  ``plugin'' system (although we encourage users who develop analyses to submit
  them to the Rivet developers for inclusion into a future release of the main
  package). This approach requires some understanding of programming within
  Rivet but you don't \emph{need} to know about exactly what the system is doing
  with the objects that you have defined.
\item Finally, Rivet developers and people who want to do non-standard things
  with their analyses will need to know something about the messy details of
  what Rivet's infrastructure is doing behind the scenes. But you'd probably
  rather be doing some physics!
\end{itemize}

The current part of this manual is for the first sort of user, who wants to get
on with studying some observables with a generator or tune, or comparing several
such models. Since everyone will fall into this category at some point, our
preent interest is to get you to that all-important ``physics plots'' stage as
quickly as possible. Analysis authors and Rivet service-mechanics will find the
more detailed information that they crave in Part~\ref{part:writinganalyses}.


\section{Quickstart}

The point of this section is to get you up and running with Rivet as soon as
possible. Doing this by hand may be rather frustrating, as Rivet depends on
several external libraries --- you'll get bored downloading and building them by
hand in the right order. Here we recommend two much simpler ways --- for the
full details of how to build Rivet by hand, please consult the Rivet Web page.


\paragraph{Ubuntu/Debian package archive}

A selection of HEP packages, including Rivet, are maintained as Debian/Ubuntu
Linux packages on the Launchpad PPA system:
\url{https://launchpad.net/~hep/+archive}. This is the nicest option for
Debian/Ubuntu, since not only will it work more easily than anything else, but
you will also automatically benefit from bug fixes and version upgrades as they
appear.

The PPA packages have been built as binaries for a variety of architectures, and
the package interdependencies are automatically known and used: all you need to
do on a Debian-type Linux system (Ubuntu included) is to add the Launchpad
archive address to your APT sources list and then request installation of the
\kbd{rivet} package in the usual way. See the Launchpad and system documentation
for all the details.


\paragraph{Bootstrap script}

For those not using Debian/Ubuntu systems, we have written a bootstrapping
script which will download tarballs of Rivet, AGILe and the other required
libraries, expand them and build them in the right order with the correct build
flags. This is generally nicer than doing it all by hand, and virtually
essential if you want to use the existing versions of FastJet, HepMC, generator
libraries, and so on from CERN AFS: there are issues with these versions which
the script works around, which you won't find easy to do yourself.

You can get the bootstrap script from the following Web address:
\url{http://svn.hepforge.org/rivet/bootstrap/rivet-bootstrap}

To run the script, we recommend that you choose a personal installation
directory. Personally, I make a \kbd{\home/local} directory for this purpose, to
avoid polluting my home directory with a lot of files. If you already use a
directory of the same name, you might want to use a separate one, say
\kbd{\home/rivetlocal}, such that if you need to delete everything in the
installation area you can do so without difficulties. You'll need to add
\kbd{\val{localdir}/bin} to your \var{PATH} environment variable and
\kbd{\val{localdir}/lib} to your \var{LD_LIBRARY_PATH}.

Now, change directory to your build area (you may also want to make this,
e.g. \kbd{\home/build}), and download the script:\\
\inp{wget \url{http://svn.hepforge.org/rivet/bootstrap/rivet-bootstrap}}\\
Now run it, specifying the install area as the argument:\\
\inp{chmod +x rivet-bootstrap}\\
\inp{./rivet-bootstrap \val{localdir}}

If you are running on a system where the CERN AFS area is mounted as
\path{/afs/cern.ch}, then the bootstrap script will attempt to use the pre-built
HepMC, LHAPDF, FastJet and GSL libraries from the LCG software area. Either way,
you'll see a large amount of build output, and finally a message telling you
what changes to your environment variables will make the system useable. 

You now have a working, installed copy of the Rivet and AGILe libraries, and the
\kbd{rivet} and \kbd{agile-runmc} executables: respectively these are the
command-line frontend to the Rivet analysis library, and a convenient steering
command for generators which do not provide their own main program with HepMC
output. To test that they work as expected, set the environment variables as
instructed, if you've not already done so, run this:\\
\inp{rivet --help}\\
%
This should print a quick-reference user guide for the \kbd{rivet} command to
the terminal. Similarly, for \kbd{agile-runmc},\\
\inp{agile-runmc --help}\\
\inp{agile-runmc --list-gens}\\
\inp{agile-runmc --beams=pp:14TeV FPythia:6413}\\
which should respectively print the help, list the available generators and make
10 LHC-type events using the Fortran Pythia 6.4.13 generator. You're on your
way! If no generators are listed, you probaby need to install a local
Genser-type generator repository: see \SectionRef{sec:genser}.

In this manual, because of its convenience, we will use \kbd{agile-runmc} as our
canonical way of producing a stream of HepMC event data; if your interest is in
running a generator like Sherpa or Herwig++ which provides its own native way to
make HepMC output, or a generator like Cascade or PHOJET which is not currently
supported by AGILe, then substitute the appropriate command in what follows.
We'll discuss using these commands in detail in \SectionRef{sec:rivetgun}.


\subsection{Getting generators for AGILe}
\label{sec:genser}

One last thing before continuing, though: the generators themselves. Again, if
you're running on a system with the CERN LCG AFS area mounted, then
\kbd{rivetgun} will attempt to automatically use the generators packaged by the
LCG Genser team.

Otherwise, you'll have to build your own mirror of the LCG generators. This
process is not standardised at the moment (this will hopefully
change), so we've provided a script, \kbd{agile-genser-bootstrap}:\\
\inp{wget \url{http://svn.hepforge.org/agile/genser/agile-genser-bootstrap}}

Now make yourself a Genser installation directory, e.g. \kbd{\var{HOME}/genser},
and \kbd{cd} into it. Then run the \kbd{agile-genser-bootstrap} script, and wait
for it all to build. Finally, set the \var{AGILE_GEN_PATH} path variable to
contain the \kbd{\val{genserDir}} directory: you should now have a few
generators to play with.

If you are interested in using a generator not currently supported by AGILe,
which does not output HepMC events in its native state, then please contact the
authors and hopefully we can help.


\subsection{Command completion}

A final installation point worth considering is using the supplied bash-shell
programmable completion setup for the \kbd{rivet} and \kbd{agile-runmc}
commands. Despite being cosmetic and semi-trivial, programmable completion makes
using \kbd{rivet} positively pleasant, especially since you no longer need to
remember the somewhat cryptic analysis names\footnote{Standard Rivet analyses
  have names which, as well as the publication date and experiment name,
  incorporate the 8-digit Spires ID code.}!

To use programmable completion, source the appropriate files from the install
location:\\
\inp{. \val{localdir}/share/Rivet/rivet-completion}\\
\inp{. \val{localdir}/share/AGILe/agile-completion}\\
If there is already a \kbd{\val{localdir}/etc/bash_completion.d} directory in
your install path, Rivet and AGILe's installation scripts will install extra
copies into that location, since automatically sourcing all completion files in
such a path is quite standard.

Apologies to \{C,k,z,\dots\}-shell users, but this feature is currently only
available for the \kbd{bash} shell. Anyone who feels like supplying fixes or
additions for their favourite shell is very welcome to get in touch with the
developers.



\section{Running Rivet analyses}
\label{sec:rivetgun}

The \kbd{rivet} executable is the easiest way to use Rivet, and will be our
example throughout this manual. This command reads HepMC events in the standard
ASCII format, either from file or from a text stream.

\subsection{The FIFO idiom}

Since you rarely want to store simulated HepMC events and they are
computationally cheap to produce (at least when compared to the remainder of
experiment simulation chains), we recommend using a Unix \emph{named pipe} (or
``FIFO'' --- first-in, first-out) to stream the events. While this may seem
unusual at first, it is just a nice way of ``pretending'' that we are writing to
and reading from a file, without actually involving any slow disk access or
building of huge files: a 1M event LHC run would occupy $\sim 60 GB$ on disk,
and typically it takes twice as long to make and analyse the events when the 
filesystem is involved! Here is an example:\\
\inp{mkfifo fifo.hepmc}\\
\inp{agile-runmc Pythia:6418 -o fifo.hepmc \&}\\
\inp{rivet -a EXAMPLE fifo.hepmc}\\
%
Note that the generator process (\kbd{agile-runmc} in this case) is
\emph{backgrounded} before \kbd{rivet} is run. This is absolutely necessary,
since the buffer size of a pipe in Linux is only 64K --- about the space
required to store one LHC event in HepMC's textual event format. The generator
process will have to wait until the buffer is cleared, e.g. by being read by
\kbd{rivet}, before computing or writing any more events. By running the
generator and \kbd{rivet} at the same time, this flow control through the buffer
is invisible to the user.
%http://home.gna.org/pysfst/tests/pipe-limit.html

In the following command examples, we will assume that a generator has been set
up to write to the \kbd{fifo.hepmc} FIFO, and just list the \kbd{rivet} command
that reads from that location. Some typical \kbd{agile-runmc} commands are
listed in \AppendixRef{app:agilerunmc}.


\subsection{Example \kbd{rivet} commands}

\begin{itemize}

\item \paragraph{Getting help:}{\kbd{rivet --help} will print a (hopefully)
    helpful list of options which may be used with the \kbd{rivet} command, as
    well as other information such as environment variables which may affect the
    run.}

\item \paragraph{Choosing analyses:}{\kbd{rivet --list-analyses} will list the
    available analyses, including both those in the Rivet distribution and any
    plugins which are found at runtime. \kbd{rivet --show-analysis \val{patt}}
    will show a lot of details about any analyses whose name match the
    \val{patt} regular expression pattern --- simple bits of analysis name are a
    perfectly valid subset of this. For example, \kbd{rivet --show-analysis
      CDF_200} exploits the standard Rivet analysis naming scheme to show
    details of all available CDF experiment analyses published in the
    ``noughties.''}

\item \paragraph{Running particular analyses:}{\kbd{rivet -a~DELPHI_1996_S3430090
      in.hepmc} will run the Rivet \kbd{DELPHI_1996_S3430090}\cite{Abreu:1996na}
    analysis on the events in the \kbd{in.hepmc} data file. This analysis is the
    one originally used for the \Delphi automated ``\textsc{Professor}''
    generator tuning.  If the first event in the data file does not have
    appropriate beams, the analysis will be disabled; since there is only one
    analysis in this case, the command will exit immediately with a warning.}

\item \paragraph{Using all analyses:}{\kbd{rivet -n~50000 -A -} will read up to 
    50k events from standard input (specified by the special ``-'' input filename) 
    and analyse them with \emph{all} the Rivet library analyses. As above, 
    incompatible analyses (based on beam particle IDs), will be removed before 
    the main analysis run begins.}

\item \paragraph{Histogramming:}{\kbd{rivet in.hepmc -H~foo} will read all the
    events in the \kbd{in.hepmc} file. The \kbd{-H} switch is used to specify
    that the output histogram file will be named \kbd{foo.aida}. By default the
    output file is called \kbd{Rivet.aida}.}

\item \paragraph{Fine-grained logging:}{\kbd{rivet in.hepmc -A
      -l~Rivet.Analysis=DEBUG~\cmdbreak -l~Rivet.Projection=DEBUG
      -l~Rivet.Projection.FinalState=TRACE~\cmdbreak -l~RivetGun=WARN
      -l~NEvt=WARN} analyse events as before, but will print different status
    information as the run progresses. Hierarchical logging control is possible
    down to the level of individual analyses and projections as shown above;
    this is useful for debugging without getting overloaded with debug
    information from \emph{all} the components at once. The default level is
    ``\textsc{info}'', which lies between ``\textsc{debug}'' and
    ``\textsc{warning}''; the ``\textsc{trace}'' level is for very low level
    information, and probably isn't needed by normal users.}

\end{itemize}



\section{Using analysis data}

In this section, we summarise how to use the data files which Rivet produces for
plotting, validation and tuning.

\subsection{Histogram formats}

Rivet currently produces output histogram data in the AIDA XML format. Most
people aren't familiar with AIDA (and we recommend that you remain that way!),
and it will disappear entirely from Rivet in version 1.2.0. You will probably
wish to cast the AIDA files to a different format for plotting, and for this we
supply several scripts.

\paragraph{Conversion to ROOT}

Your knee-jerk reaction is probably to want to know how to plot your Rivet
histograms in ROOT. Don't worry; you can recover from this unfortunate behaviour
after only a few months of therapy. For unrepentant ROOT junkies, Rivet installs
an \kbd{aida2root} script, which converts the AIDA records to a \kbd{.root} file
full of ROOT \kbd{TGraph}s. One word of warning: a bug in ROOT means that
\kbd{TGraph}s do not render properly from file because the axis is not drawn by
default. To display the plots correctly in ROOT you will need to pass the
\kbd{"AP"} drawing option string to either the \kbd{TGraph::Draw()} method, or
in the options box in the \kbd{TBrowser} GUI interface.

\paragraph{Conversion to ``flat format''}

Most of our histogramming is based around the YODA ``flat'' plain text format,
which can easily be read (and written) by hand. We provide a script called
\kbd{aida2flat} to do this conversion. Run \kbd{aida2flat -h} to get usage
instructions; in particular the Gnuplot and ``split output'' options are useful
for further visualisation. Aside from anything else, this is useful for simply
checking the contents of an AIDA file, with \kbd{aida2flat Rivet.aida | less}.

\vspace{1.8em}

\begin{detail}
  We get asked a lot about why we don't use ROOT internally: aside from a
  general unhappiness about the design and quality of the data objects in ROOT,
  the monolithic nature of the system makes it a big dependency for a system as
  small as Rivet. While not an issue for experimentalists, most theorists and
  generator developers do not use ROOT and we preferred to embed the AIDA
  system, which in its LWH implementation requires no external package. The
  replacement for AIDA will be another lightweight system rather than ROOT, with
  an emphasis on friendly, intuitive data object design, and correct handling of
  sample merging statistics for all data objects.
\end{detail}


\subsection{Plotting and comparing data}

Rivet comes with two commands --- \kbd{compare-histos} and \kbd{make-plots} ---
for comparing and plotting data files. These commands produce nice comparison
plots of publication quality from the YODA format text files, e.g.:\\
\inp{compare-plots path/to/CDF_2001_S4751469.aida py.aida:'Pythia 6.418' \cmdbreak hw.aida:'Herwig++ 2.3.0'}

This command will have compared the three named data files (ending in
\kbd{.aida}), identified which plots are available in them, and combined the MC
and reference plots appropriately into a set of plot data files ending with
\kbd{.dat}. The strings after the ":" for the MC files are specifying ID strings to
appear in the plot legends. You can also run compare-plots to just compare MC--MC
data files. More options are described by running \kbd{compare-histos --help}.

Incidentally, the reference files for each Rivet analysis are to be found in the
installed Rivet shared data directory, \kbd{\val{installdir}/share/Rivet}. You
can find the location of this by using the \kbd{rivet-config} command:\\
\inp{rivet-config --datadir}

\noindent
You can now plot the created data files using the make-plots command:\\
\inp{make-plots --pdf *.dat}\\
The \kbd{--pdf} flag makes the output plots in PDF format: by default the output
is in PostScript (\kbd{.ps}), and flags for conversion to EPS and PNG are also
available.



\cleardoublepage
\part{Standard Rivet analyses}
\label{part:analyses}

%\section{Rivet analyses reference guide}

In this section we describe the standard experimental analyses included with the
Rivet library. To maintain synchronisation with the code, these descriptions are
generated automatically from the metadata in the analysis objects
themselves. This is currently rather sparse, hence the briefness of the
descriptions shown here. Richer metadata will be added to the code soon!

\makeatletter
\renewcommand{\d}[1]{\ensuremath{\mathrm{#1}}}
\let\old@eta\eta
\renewcommand{\eta}{\ensuremath{\old@eta}\xspace}
\let\old@phi\phi
\renewcommand{\phi}{\ensuremath{\old@phi}\xspace}
\providecommand{\pT}{\ensuremath{p_\perp}\xspace}
\providecommand{\pTmin}{\ensuremath{p_\perp^\text{min}}\xspace}
\makeatother

\section{LEP analyses}\typeout{Handling analysis ALEPH_1991_S2435284}
\subsection[ALEPH\_1991\_S2435284]{ALEPH\_1991\_S2435284\,\cite{Decamp:1991uz}}
\textbf{Hadronic Z decay charged multiplicity measurement}\newline
\textbf{Experiment:} ALEPH (LEP 1) \newline
\textbf{Spires ID:} \href{http://www.slac.stanford.edu/spires/find/hep/www?rawcmd=key+2435284}{2435284}\newline
\textbf{Status:} VALIDATED\newline
\textbf{Authors:}
 \penalty 100
\begin{itemize}
  \item Andy Buckley $\langle\,$\href{mailto:andy.buckley@durham.ac.uk}{andy.buckley@durham.ac.uk}$\,\rangle$;
\end{itemize}
\textbf{References:}
 \penalty 100
\begin{itemize}
  \item Phys. Lett. B, 273, 181 (1991)
\end{itemize}
\textbf{Run details:}
 \penalty 100
\begin{itemize}

  \item Hadronic Z decay events generated on the Z pole (\ensuremath{\sqrt{s}} = 91.2 GeV)\end{itemize}

\noindent The charged particle multiplicity distribution of hadronic Z decays, as measured on the peak of the Z resonance using the ALEPH detector at LEP. The unfolding procedure was model independent, and the distribution was found to have a mean of $20.85 \pm 0.24$, Comparison with lower energy data supports the KNO scaling hypothesis. The shape of the multiplicity distribution is well described by a log-normal distribution, as predicted from a cascading model for multi-particle production.

\clearpage


\clearpage

\typeout{Handling analysis ALEPH_1996_S3196992}
\subsection[ALEPH\_1996\_S3196992]{ALEPH\_1996\_S3196992\,\cite{Buskulic:1995au}}
\textbf{Measurement of the quark to photon fragmentation function}\newline
\textbf{Experiment:} ALEPH (LEP Run 1) \newline
\textbf{Spires ID:} \href{http://www.slac.stanford.edu/spires/find/hep/www?rawcmd=key+3196992}{3196992}\newline
\textbf{Status:} VALIDATED\newline
\textbf{Authors:}
 \penalty 100
\begin{itemize}
  \item Frank Siegert $\langle\,$\href{mailto:frank.siegert@durham.ac.uk}{frank.siegert@durham.ac.uk}$\,\rangle$;
\end{itemize}
\textbf{References:}
 \penalty 100
\begin{itemize}
  \item Z.Phys.C69:365-378,1996
  \item DOI: \href{http://dx.doi.org/10.1007/s002880050037}{10.1007/s002880050037}
\end{itemize}
\textbf{Run details:}
 \penalty 100
\begin{itemize}

  \item $e^+e^-\to$ jets with $\pi$ and $\eta$ decays turned off.\end{itemize}

\noindent Earlier measurements at LEP of isolated hard photons in hadronic Z decays, attributed to radiation from primary quark pairs, have been extended in the ALEPH experiment to include hard photon production inside hadron jets. Events are selected where all particles combine democratically to form hadron jets, one of which contains a photon with a fractional energy $z > 0.7$. After statistical subtraction of non-prompt photons, the quark-to-photon fragmentation function, $D(z)$, is extracted directly from the measured 2-jet rate.

\clearpage


\clearpage

\typeout{Handling analysis ALEPH_1996_S3486095}
\subsection[ALEPH\_1996\_S3486095]{ALEPH\_1996\_S3486095\,\cite{Barate:1996fi}}
\textbf{Studies of QCD with the ALEPH detector.}\newline
\textbf{Experiment:} ALEPH (LEP 1) \newline
\textbf{Spires ID:} \href{http://www.slac.stanford.edu/spires/find/hep/www?rawcmd=key+3486095}{3486095}\newline
\textbf{Status:} VALIDATED\newline
\textbf{Authors:}
 \penalty 100
\begin{itemize}
  \item Holger Schulz $\langle\,$\href{mailto:holger.schulz@physik.hu-berlin.de}{holger.schulz@physik.hu-berlin.de}$\,\rangle$;
\end{itemize}
\textbf{References:}
 \penalty 100
\begin{itemize}
  \item Phys. Rept., 294, 1--165 (1998)
\end{itemize}
\textbf{Run details:}
 \penalty 100
\begin{itemize}

  \item Hadronic Z decay events generated on the Z pole (\ensuremath{\sqrt{s}} = 91.2 GeV)\end{itemize}

\noindent Summary paper of QCD results as measured by ALEPH at LEP 1. The publication includes various event shape variables, multiplicities (identified particles and inclusive), and particle spectra.

\clearpage


\clearpage

\typeout{Handling analysis ALEPH_2004_S5765862}
\subsection[ALEPH\_2004\_S5765862]{ALEPH\_2004\_S5765862\,\cite{Heister:2003aj}}
\textbf{Jet rates and event shapes at LEP I and II}\newline
\textbf{Experiment:} ALEPH (LEP Run 1 and 2) \newline
\textbf{Spires ID:} \href{http://www.slac.stanford.edu/spires/find/hep/www?rawcmd=key+5765862}{5765862}\newline
\textbf{Status:} VALIDATED\newline
\textbf{Authors:}
 \penalty 100
\begin{itemize}
  \item Frank Siegert $\langle\,$\href{mailto:frank.siegert@durham.ac.uk}{frank.siegert@durham.ac.uk}$\,\rangle$;
\end{itemize}
\textbf{References:}
 \penalty 100
\begin{itemize}
  \item Eur.Phys.J.C35:457-486,2004
  \item DOI: \href{http://dx.doi.org/10.1140/epjc/s2004-01891-4}{10.1140/epjc/s2004-01891-4}
  \item http://cdsweb.cern.ch/record/690637/files/ep-2003-084.pdf
\end{itemize}
\textbf{Run details:}
 \penalty 100
\begin{itemize}

  \item $e^+ e^- \to$ jet jet (+ jets)\end{itemize}

\noindent Jet rates, event-shape variables and inclusive charged particle spectra are measured in $e^+ e^-$ collisions at CMS energies between 91 and 209 GeV. The previously published data at 91.2 GeV and 133 GeV have been re-processed and the higher energy data are presented here for the first time.

\clearpage


\clearpage

\typeout{Handling analysis DELPHI_1995_S3137023}
\subsection[DELPHI\_1995\_S3137023]{DELPHI\_1995\_S3137023\,\cite{Abreu:1995qx}}
\textbf{Strange baryon production in $Z$ hadronic decays at Delphi}\newline
\textbf{Experiment:} DELPHI (LEP 1) \newline
\textbf{Spires ID:} \href{http://www.slac.stanford.edu/spires/find/hep/www?rawcmd=key+3137023}{3137023}\newline
\textbf{Status:} VALIDATED\newline
\textbf{Authors:}
 \penalty 100
\begin{itemize}
  \item Hendrik Hoeth $\langle\,$\href{mailto:hendrik.hoeth@cern.ch}{hendrik.hoeth@cern.ch}$\,\rangle$;
\end{itemize}
\textbf{References:}
 \penalty 100
\begin{itemize}
  \item Z. Phys. C, 67, 543--554 (1995)
\end{itemize}
\textbf{Run details:}
 \penalty 100
\begin{itemize}

  \item Hadronic Z decay events generated on the Z pole (\ensuremath{\sqrt{s}} = 91.2 GeV)\end{itemize}

\noindent Measurement of the $\Xi^-$ and $\Sigma^+(1385)/\Sigma^-(1385)$ scaled momentum distributions by DELPHI at LEP 1. The paper also has the production cross-sections of these particles, but that's not implemented in Rivet.

\clearpage


\clearpage

\typeout{Handling analysis DELPHI_1996_S3430090}
\subsection[DELPHI\_1996\_S3430090]{DELPHI\_1996\_S3430090\,\cite{Abreu:1996na}}
\textbf{Delphi MC tuning on event shapes and identified particles.}\newline
\textbf{Experiment:} DELPHI (LEP 1) \newline
\textbf{Spires ID:} \href{http://www.slac.stanford.edu/spires/find/hep/www?rawcmd=key+3430090}{3430090}\newline
\textbf{Status:} VALIDATED\newline
\textbf{Authors:}
 \penalty 100
\begin{itemize}
  \item Andy Buckley $\langle\,$\href{mailto:andy.buckley@durham.ac.uk}{andy.buckley@durham.ac.uk}$\,\rangle$;
  \item Hendrik Hoeth $\langle\,$\href{mailto:hendrik.hoeth@cern.ch}{hendrik.hoeth@cern.ch}$\,\rangle$;
\end{itemize}
\textbf{References:}
 \penalty 100
\begin{itemize}
  \item Z.Phys.C73:11-60,1996
  \item DOI: \href{http://dx.doi.org/10.1007/s002880050295}{10.1007/s002880050295}
\end{itemize}
\textbf{Run details:}
 \penalty 100
\begin{itemize}

  \item \ensuremath{\sqrt{s}} = 91.2 GeV, $e^+ e^- \ensuremath{\to} Z^0$ production with hadronic decays only\end{itemize}

\noindent Event shape and charged particle inclusive distributions measured using 750000 decays of Z bosons to hadrons from the DELPHI detector at LEP. This data, combined with identified particle distributions from all LEP experiments, was used for tuning of shower-hadronisation event generators by the original PROFESSOR method.  This is a critical analysis for MC event generator tuning of final state radiation and both flavour and kinematic aspects of hadronisation models.

\clearpage


\clearpage

\typeout{Handling analysis DELPHI_2002_069_CONF_603}
\subsection{DELPHI\_2002\_069\_CONF\_603}
\textbf{Study of the b-quark fragmentation function at LEP 1}\newline
\textbf{Experiment:} DELPHI (LEP 1) \newline
\textbf{Status:} VALIDATED\newline
\textbf{Authors:}
 \penalty 100
\begin{itemize}
  \item Hendrik Hoeth $\langle\,$\href{mailto:hendrik.hoeth@cern.ch}{hendrik.hoeth@cern.ch}$\,\rangle$;
\end{itemize}
\textbf{References:}
 \penalty 100
\begin{itemize}
  \item DELPHI note 2002-069-CONF-603 (ICHEP 2002)
\end{itemize}
\textbf{Run details:}
 \penalty 100
\begin{itemize}

  \item Hadronic Z decay events generated on the Z pole (\ensuremath{\sqrt{s}} = 91.2 GeV)\end{itemize}

\noindent Measurement of the $b$-quark fragmentation function by DELPHI using 1994 LEP 1 data. The fragmentation function for both weakly decaying and primary $b$-quarks has been determined in a model independent way. Nevertheless the authors trust $f(x_B^\text{weak})$ more than $f(x_B^\text{prim})$.

\clearpage


\clearpage

\typeout{Handling analysis JADE_OPAL_2000_S4300807}
\subsection[JADE\_OPAL\_2000\_S4300807]{JADE\_OPAL\_2000\_S4300807\,\cite{Pfeifenschneider:1999rz}}
\textbf{Jet rates in $e^+e^-$ at JADE [35--44 GeV] and OPAL [91--189 GeV].}\newline
\textbf{Experiment:} JADE_OPAL (PETRA and LEP) \newline
\textbf{Spires ID:} \href{http://www.slac.stanford.edu/spires/find/hep/www?rawcmd=key+4300807}{4300807}\newline
\textbf{Status:} VALIDATED\newline
\textbf{Authors:}
 \penalty 100
\begin{itemize}
  \item Frank Siegert $\langle\,$\href{mailto:frank.siegert@durham.ac.uk}{frank.siegert@durham.ac.uk}$\,\rangle$;
\end{itemize}
\textbf{References:}
 \penalty 100
\begin{itemize}
  \item Eur.Phys.J.C17:19-51,2000
  \item arXiv: \href{http://arxiv.org/abs/hep-ex/0001055}{hep-ex/0001055}
\end{itemize}
\textbf{Run details:}
 \penalty 100
\begin{itemize}

  \item $e^+ e^- \to$ jet jet (+ jets)\end{itemize}

\noindent Differential and integrated jet rates for Durham and JADE jet algorithms.

\clearpage


\clearpage

\typeout{Handling analysis OPAL_1998_S3780481}
\subsection[OPAL\_1998\_S3780481]{OPAL\_1998\_S3780481\,\cite{Ackerstaff:1998hz}}
\textbf{Measurements of flavor dependent fragmentation functions in $Z^0 \ensuremath{\to} q \bar{q}$ events}\newline
\textbf{Experiment:} OPAL (LEP 1) \newline
\textbf{Spires ID:} \href{http://www.slac.stanford.edu/spires/find/hep/www?rawcmd=key+3780481}{3780481}\newline
\textbf{Status:} VALIDATED\newline
\textbf{Authors:}
 \penalty 100
\begin{itemize}
  \item Hendrik Hoeth $\langle\,$\href{mailto:hendrik.hoeth@cern.ch}{hendrik.hoeth@cern.ch}$\,\rangle$;
\end{itemize}
\textbf{References:}
 \penalty 100
\begin{itemize}
  \item Eur. Phys. J, C7, 369--381 (1999)
  \item hep-ex/9807004
\end{itemize}
\textbf{Run details:}
 \penalty 100
\begin{itemize}

  \item Hadronic Z decay events generated on the Z pole (\ensuremath{\sqrt{s}} = 91.2 GeV)\end{itemize}

\noindent Measurement of scaled momentum distributions and total charged multiplicities in flavour tagged events at LEP 1. OPAL measured these observables in uds-, c-, and b-events separately. An inclusive measurement is also included.

\clearpage


\clearpage

\typeout{Handling analysis OPAL_2004_S6132243}
\subsection[OPAL\_2004\_S6132243]{OPAL\_2004\_S6132243\,\cite{Abbiendi:2004qz}}
\textbf{Event shape distributions and moments in $e^+ e^-$ \ensuremath{\to} hadrons at 91--209 GeV}\newline
\textbf{Experiment:} OPAL (LEP 1 \& 2) \newline
\textbf{Spires ID:} \href{http://www.slac.stanford.edu/spires/find/hep/www?rawcmd=key+6132243}{6132243}\newline
\textbf{Status:} VALIDATED\newline
\textbf{Authors:}
 \penalty 100
\begin{itemize}
  \item Andy Buckley $\langle\,$\href{mailto:andy.buckley@cern.ch}{andy.buckley@cern.ch}$\,\rangle$;
\end{itemize}
\textbf{References:}
 \penalty 100
\begin{itemize}
  \item Eur.Phys.J.C40:287-316,2005
  \item hep-ex/0503051
\end{itemize}
\textbf{Run details:}
 \penalty 100
\begin{itemize}

  \item Hadronic $e^+ e^-$ events at 4 representative energies (91, 133, 177, 197). Runs with \ensuremath{\sqrt{s}} above the Z mass need to have ISR suppressed, since the data has been corrected to remove radiative return to the Z.\end{itemize}

\noindent Measurement of $e^+ e^-$ event shape variable distributions and their 1st to 5th moments in LEP running from the Z pole to the highest LEP 2 energy of 209 GeV.

\clearpage


\section{Tevatron analyses}\typeout{Handling analysis CDF_1988_S1865951}
\subsection[CDF\_1988\_S1865951]{CDF\_1988\_S1865951\,\cite{Abe:1988yu}}
\textbf{CDF transverse momentum distributions at 630 GeV and 1800 GeV.}\newline
\textbf{Experiment:} CDF (Tevatron Run I) \newline
\textbf{Spires ID:} \href{http://www.slac.stanford.edu/spires/find/hep/www?rawcmd=key+1865951}{1865951}\newline
\textbf{Status:} VALIDATED\newline
\textbf{Authors:}
 \penalty 100
\begin{itemize}
  \item Christophe Vaillant $\langle\,$\href{mailto:c.l.j.j.vaillant@durham.ac.uk}{c.l.j.j.vaillant@durham.ac.uk}$\,\rangle$;
  \item Andy Buckley $\langle\,$\href{mailto:andy.buckley@cern.ch}{andy.buckley@cern.ch}$\,\rangle$;
\end{itemize}
\textbf{References:}
 \penalty 100
\begin{itemize}
  \item Phys.Rev.Lett.61:1819,1988
  \item DOI: \href{http://dx.doi.org/10.1103/PhysRevLett.61.1819}{10.1103/PhysRevLett.61.1819}
\end{itemize}
\textbf{Run details:}
 \penalty 100
\begin{itemize}

  \item QCD min bias events at \ensuremath{\sqrt{s}} = 630 GeV and 1800 GeV, $|\eta| < 1.0$.\end{itemize}

\noindent Transverse momentum distributions at 630 GeV and 1800 GeV based on data from the CDF experiment at the Tevatron collider.

\clearpage


\clearpage

\typeout{Handling analysis CDF_1990_S2089246}
\subsection[CDF\_1990\_S2089246]{CDF\_1990\_S2089246\,\cite{Abe:1989td}}
\textbf{CDF pseudorapidity distributions at 630 and 1800 GeV}\newline
\textbf{Experiment:} CDF (Tevatron Run 0) \newline
\textbf{Spires ID:} \href{http://www.slac.stanford.edu/spires/find/hep/www?rawcmd=key+2089246}{2089246}\newline
\textbf{Status:} VALIDATED\newline
\textbf{Authors:}
 \penalty 100
\begin{itemize}
  \item Andy Buckley $\langle\,$\href{mailto:andy.buckley@cern.ch}{andy.buckley@cern.ch}$\,\rangle$;
\end{itemize}
\textbf{References:}
 \penalty 100
\begin{itemize}
  \item Phys.Rev.D41:2330,1990
  \item DOI: \href{http://dx.doi.org/10.1103/PhysRevD.41.2330}{10.1103/PhysRevD.41.2330}
\end{itemize}
\textbf{Run details:}
 \penalty 100
\begin{itemize}

  \item QCD min bias events at \ensuremath{\sqrt{s}} = 630 and 1800 GeV. Particles with $c \tau > 10$mm should be set stable.\end{itemize}

\noindent Pseudorapidity distributions based on the CDF 630 and 1800 GeV runs from 1987. All data is detector corrected. The data confirms the UA5 measurement of a $\d{N}/\d{\eta}$ rise with energy faster than $\ln{\sqrt{s}}$, and as such this analysis is important for constraining the energy evolution of minimum bias and underlying event characteristics in MC simulations.

\clearpage


\clearpage

\typeout{Handling analysis CDF_1996_S3418421}
\subsection[CDF\_1996\_S3418421]{CDF\_1996\_S3418421\,\cite{Abe:1996mj}}
\textbf{Dijet angular distributions}\newline
\textbf{Experiment:} CDF (Tevatron Run 1) \newline
\textbf{Spires ID:} \href{http://www.slac.stanford.edu/spires/find/hep/www?rawcmd=key+3418421}{3418421}\newline
\textbf{Status:} VALIDATED\newline
\textbf{Authors:}
 \penalty 100
\begin{itemize}
  \item Frank Siegert $\langle\,$\href{mailto:frank.siegert@durham.ac.uk}{frank.siegert@durham.ac.uk}$\,\rangle$;
\end{itemize}
\textbf{References:}
 \penalty 100
\begin{itemize}
  \item Phys.Rev.Lett.77:5336-5341,1996
  \item DOI: \href{http://dx.doi.org/10.1103/PhysRevLett.77.5336}{10.1103/PhysRevLett.77.5336}
  \item arXiv: \href{http://arxiv.org/abs/hep-ex/9609011}{hep-ex/9609011}
\end{itemize}
\textbf{Run details:}
 \penalty 100
\begin{itemize}

  \item QCD dijet events at Tevatron $\sqrt{s}=1.8$ TeV without MPI.\end{itemize}

\noindent Measurement of jet angular distributions in events with two jets in the final state in 5 bins of dijet invariant mass. Based on $106 \mathrm{pb}^{-1}$

\clearpage


\clearpage

\typeout{Handling analysis CDF_1998_S3618439}
\subsection[CDF\_1998\_S3618439]{CDF\_1998\_S3618439\,\cite{Abe:1997eua}}
\textbf{Differential cross-section for events with large total transverse energy}\newline
\textbf{Experiment:} CDF (Tevatron Run 1) \newline
\textbf{Spires ID:} \href{http://www.slac.stanford.edu/spires/find/hep/www?rawcmd=key+3618439}{3618439}\newline
\textbf{Status:} VALIDATED\newline
\textbf{Authors:}
 \penalty 100
\begin{itemize}
  \item Frank Siegert $\langle\,$\href{mailto:frank.siegert@durham.ac.uk}{frank.siegert@durham.ac.uk}$\,\rangle$;
\end{itemize}
\textbf{References:}
 \penalty 100
\begin{itemize}
  \item Phys.Rev.Lett.80:3461-3466,1998
  \item 10.1103/PhysRevLett.80.3461
\end{itemize}
\textbf{Run details:}
 \penalty 100
\begin{itemize}

  \item QCD events at Tevatron with $\sqrt{s}=1.8$ TeV without MPI.\end{itemize}

\noindent Measurement of the differential cross section $\mathrm{d}\sigma/\mathrm{d}E_\perp^j$ for the production of multijet events in $p\bar{p}$ collisions where the sum is over all jets with transverse energy $E_\perp^j > E_\perp^\mathrm{min}$.

\clearpage


\clearpage

\typeout{Handling analysis CDF_2000_S4155203}
\subsection[CDF\_2000\_S4155203]{CDF\_2000\_S4155203\,\cite{Affolder:1999jh}}
\textbf{Z \pT measurement in CDF Z \ensuremath{\to} $e^+e^-$ events}\newline
\textbf{Experiment:} CDF (Tevatron Run 1) \newline
\textbf{Spires ID:} \href{http://www.slac.stanford.edu/spires/find/hep/www?rawcmd=key+4155203}{4155203}\newline
\textbf{Status:} VALIDATED\newline
\textbf{Authors:}
 \penalty 100
\begin{itemize}
  \item Hendrik Hoeth $\langle\,$\href{mailto:hendrik.hoeth@cern.ch}{hendrik.hoeth@cern.ch}$\,\rangle$;
\end{itemize}
\textbf{References:}
 \penalty 100
\begin{itemize}
  \item Phys.Rev.Lett.84:845-850,2000
  \item arXiv: \href{http://arxiv.org/abs/hep-ex/0001021}{hep-ex/0001021}
  \item DOI: \href{http://dx.doi.org/10.1103/PhysRevLett.84.845}{10.1103/PhysRevLett.84.845}
\end{itemize}
\textbf{Run details:}
 \penalty 100
\begin{itemize}

  \item $p\bar{p}$ collisions at 1800 GeV. $Z/\gamma^*$ Drell-Yan events with $e^+e^-$ decay mode only.\end{itemize}

\noindent Measurement of transverse momentum and total cross section of $e^+e^-$ pairs in the Z-boson region of $66~\text{GeV}/c^2 < m_{ee} < 116~\text{GeV}/c^2$ from pbar-p collisions at \ensuremath{\sqrt{s}} = 1.8 TeV, with the Tevatron CDF detector.  The Z \pT, in a fully-factorised picture, is generated by the momentum balance against initial state radiation (ISR) and the primordial/intrinsic \pT of the Z's parent partons in the incoming hadrons. The Z \pT is important in generator tuning to fix the interplay of ISR and multi-parton interactions (MPI) ingenerating `underlying event' activity.
This analysis is subject to ambiguities in the experimental Z \pT definition, since the Rivet implementation reconstructs the Z momentum from the dilepton pair with finite cones for QED bremstrahlung summation, rather than non-portable direct use of the (sometimes absent) Z in the event record.

\clearpage


\clearpage

\typeout{Handling analysis CDF_2000_S4266730}
\subsection[CDF\_2000\_S4266730]{CDF\_2000\_S4266730\,\cite{Affolder:1999ua}}
\textbf{Differential Dijet Mass Cross Section}\newline
\textbf{Experiment:} CDF (Tevatron Run 1) \newline
\textbf{Spires ID:} \href{http://www.slac.stanford.edu/spires/find/hep/www?rawcmd=key+4266730}{4266730}\newline
\textbf{Status:} VALIDATED\newline
\textbf{Authors:}
 \penalty 100
\begin{itemize}
  \item Frank Siegert $\langle\,$\href{mailto:frank.siegert@durham.ac.uk}{frank.siegert@durham.ac.uk}$\,\rangle$;
\end{itemize}
\textbf{References:}
 \penalty 100
\begin{itemize}
  \item Phys.Rev.D61:091101,2000
  \item DOI: \href{http://dx.doi.org/10.1103/PhysRevD.61.091101}{10.1103/PhysRevD.61.091101}
  \item arXiv: \href{http://arxiv.org/abs/hep-ex/9912022}{hep-ex/9912022}
\end{itemize}
\textbf{Run details:}
 \penalty 100
\begin{itemize}

  \item Dijet events at Tevatron with $\sqrt{s}=1.8$ TeV\end{itemize}

\noindent Measurement of the cross section for production of two or more jets as a function of dijet mass in the range 180 to 1000 GeV. It is based on an integrated luminosity of $86 \mathrm{pb}^{-1}$.

\clearpage


\clearpage

\typeout{Handling analysis CDF_2001_S4517016}
\subsection[CDF\_2001\_S4517016]{CDF\_2001\_S4517016\,\cite{Affolder:2000ew}}
\textbf{Two jet triply-differential cross-section}\newline
\textbf{Experiment:} CDF (Tevatron Run 1) \newline
\textbf{Spires ID:} \href{http://www.slac.stanford.edu/spires/find/hep/www?rawcmd=key+4517016}{4517016}\newline
\textbf{Status:} VALIDATED\newline
\textbf{Authors:}
 \penalty 100
\begin{itemize}
  \item Frank Siegert $\langle\,$\href{mailto:frank.siegert@durham.ac.uk}{frank.siegert@durham.ac.uk}$\,\rangle$;
\end{itemize}
\textbf{References:}
 \penalty 100
\begin{itemize}
  \item Phys.Rev.D64:012001,2001
  \item DOI: \href{http://dx.doi.org/10.1103/PhysRevD.64.012001}{10.1103/PhysRevD.64.012001}
  \item arXiv: \href{http://arxiv.org/abs/hep-ex/0012013}{hep-ex/0012013}
\end{itemize}
\textbf{Run details:}
 \penalty 100
\begin{itemize}

  \item Dijet events at Tevatron with $\sqrt{s}=1.8$ TeV\end{itemize}

\noindent A measurement of the two-jet differential cross section, $\mathrm{d}^3\sigma/\mathrm{d}E_T \, \mathrm{d}\eta_1 \, \mathrm{d}\eta_2$, based on an integrated luminosity of $86 \mathrm{pb}^{-1}$. The differential cross section is measured as a function of the transverse energy, $E_\perp$, of a jet in the pseudorapidity region $0.1 < |\eta_1| < 0.7$ for four different pseudorapidity bins of a second jet restricted to $0.1 < |\eta_2| < 3.0$.

\clearpage


\clearpage

\typeout{Handling analysis CDF_2001_S4563131}
\subsection[CDF\_2001\_S4563131]{CDF\_2001\_S4563131\,\cite{Affolder:2001fa}}
\textbf{Inclusive jet cross section}\newline
\textbf{Experiment:} CDF (Tevatron Run 1) \newline
\textbf{Spires ID:} \href{http://www.slac.stanford.edu/spires/find/hep/www?rawcmd=key+4563131}{4563131}\newline
\textbf{Status:} VALIDATED\newline
\textbf{Authors:}
 \penalty 100
\begin{itemize}
  \item Frank Siegert $\langle\,$\href{mailto:frank.siegert@durham.ac.uk}{frank.siegert@durham.ac.uk}$\,\rangle$;
\end{itemize}
\textbf{References:}
 \penalty 100
\begin{itemize}
  \item Phys.Rev.D64:032001,2001
  \item DOI: \href{http://dx.doi.org/10.1103/PhysRevD.64.032001}{10.1103/PhysRevD.64.032001}
  \item arXiv: \href{http://arxiv.org/abs/hep-ph/0102074}{hep-ph/0102074}
\end{itemize}
\textbf{Run details:}
 \penalty 100
\begin{itemize}

  \item Dijet events at Tevatron with $\sqrt{s}=1.8$ TeV\end{itemize}

\noindent Measurement of the inclusive jet cross section for jet transverse energies from 40 to 465 GeV in the pseudo-rapidity range $0.1<|\eta|<0.7$. The results are based on 87 $\mathrm{pb}^{-1}$ of data.

\clearpage


\clearpage

\typeout{Handling analysis CDF_2001_S4751469}
\subsection[CDF\_2001\_S4751469]{CDF\_2001\_S4751469\,\cite{Affolder:2001xt}}
\textbf{Field \& Stuart Run I underlying event analysis.}\newline
\textbf{Experiment:} CDF (Tevatron Run 1) \newline
\textbf{Spires ID:} \href{http://www.slac.stanford.edu/spires/find/hep/www?rawcmd=key+4751469}{4751469}\newline
\textbf{Status:} VALIDATED\newline
\textbf{Authors:}
 \penalty 100
\begin{itemize}
  \item Andy Buckley $\langle\,$\href{mailto:andy.buckley@durham.ac.uk}{andy.buckley@durham.ac.uk}$\,\rangle$;
\end{itemize}
\textbf{References:}
 \penalty 100
\begin{itemize}
  \item Phys.Rev.D65:092002,2002
  \item FNAL-PUB 01/211-E
\end{itemize}
\textbf{Run details:}
 \penalty 100
\begin{itemize}

  \item $p\bar{p}$ QCD interactions at 1800 GeV. The leading jet is binned from 0--49 GeV, and histos can usually can be filled with a single generator run without kinematic sub-samples.\end{itemize}

\noindent The original CDF underlying event analysis, based on decomposing each event into a transverse structure with ``toward'', ``away'' and ``transverse'' regions defined relative to the azimuthal direction of the leading jet in the event. Since the toward region is by definition dominated by the hard process, as is the away region by momentum balance in the matrix element, the transverse region is most sensitive to multi-parton interactions. The transverse regions occupy $|\phi| \in [60\degree, 120\degree]$ for $|\eta| < 1$. The \pT ranges for the leading jet are divided experimentally into the `min-bias' sample from 0--20 GeV, and the `JET20' sample from 18--49 GeV.

\clearpage


\clearpage

\typeout{Handling analysis CDF_2002_S4796047}
\subsection[CDF\_2002\_S4796047]{CDF\_2002\_S4796047\,\cite{Acosta:2001rm}}
\textbf{CDF Run 1 charged multiplicity measurement}\newline
\textbf{Experiment:} CDF (Tevatron Run 1) \newline
\textbf{Spires ID:} \href{http://www.slac.stanford.edu/spires/find/hep/www?rawcmd=key+4796047}{4796047}\newline
\textbf{Status:} VALIDATED\newline
\textbf{Authors:}
 \penalty 100
\begin{itemize}
  \item Hendrik Hoeth $\langle\,$\href{mailto:hendrik.hoeth@cern.ch}{hendrik.hoeth@cern.ch}$\,\rangle$;
\end{itemize}
\textbf{References:}
 \penalty 100
\begin{itemize}
  \item Phys.Rev.D65:072005,2002
  \item DOI: \href{http://dx.doi.org/10.1103/PhysRevD.65.072005}{10.1103/PhysRevD.65.072005}
\end{itemize}
\textbf{Run details:}
 \penalty 100
\begin{itemize}

  \item QCD events at \ensuremath{\sqrt{s}} = 630 and 1800 GeV.\end{itemize}

\noindent A study of $p\bar{p}$ collisions at \ensuremath{\sqrt{s}} = 1800 and 630 GeV collected using a minimum bias trigger in which the data set is divided into two classes corresponding to `soft' and `hard' interactions. For each subsample, the analysis includes measurements of the multiplicity, transverse momentum (\pT) spectra, and the average \pT and event-by-event \pT dispersion as a function of multiplicity. A comparison of results shows distinct differences in the behavior of the two samples as a function of the center of mass energy. The properties of the soft sample are invariant as a function of c.m. energy.

\clearpage


\clearpage

\typeout{Handling analysis CDF_2004_S5839831}
\subsection[CDF\_2004\_S5839831]{CDF\_2004\_S5839831\,\cite{Acosta:2004wqa}}
\textbf{Transverse cone and `Swiss cheese' underlying event studies}\newline
\textbf{Experiment:} CDF (Tevatron Run 2) \newline
\textbf{Spires ID:} \href{http://www.slac.stanford.edu/spires/find/hep/www?rawcmd=key+5839831}{5839831}\newline
\textbf{Status:} VALIDATED\newline
\textbf{Authors:}
 \penalty 100
\begin{itemize}
  \item Andy Buckley $\langle\,$\href{mailto:andy.buckley@durham.ac.uk}{andy.buckley@durham.ac.uk}$\,\rangle$;
\end{itemize}
\textbf{References:}
 \penalty 100
\begin{itemize}
  \item Phys. Rev. D70, 072002 (2004)
  \item arXiv: \href{http://arxiv.org/abs/hep-ex/0404004}{hep-ex/0404004}
\end{itemize}
\textbf{Run details:}
 \penalty 100
\begin{itemize}

  \item QCD events at \ensuremath{\sqrt{s}} = 630 \& 1800 GeV. Several \pTmin cutoffs are probably required to fill the profile histograms, e.g. 0 (min bias), 30, 90, 150 GeV at 1800 GeV, and 0 (min bias), 20, 90, 150 GeV at 630 GeV.\end{itemize}

\noindent This analysis studies the underlying event via transverse cones of  $R = 0.7$ at 90 degrees in \phi relative to the leading (highest E) jet, at \ensuremath{\sqrt{s}} = 630 and 1800 GeV. This is similar to the 2001 CDF UE analysis, except that cones, rather than the whole central \eta range are used. The transverse cones are categorised as TransMIN and TransMAX on an event-by-event basis, to give greater sensitivity to the UE component.
`Swiss Cheese' distributions, where cones around the leading $n$ jets are excluded from the distributions, are also included for $n = 2, 3$.  This analysis is useful for constraining the energy evolution of the underlying event, since it performs the same analyses at two distinct CoM energies.
WARNING! The \pT plots are normalised to raw number of events. The min bias data have not been reproduced by MC, and are not recommended for tuning.

\clearpage


\clearpage

\typeout{Handling analysis CDF_2005_S6080774}
\subsection[CDF\_2005\_S6080774]{CDF\_2005\_S6080774\,\cite{Acosta:2004sn}}
\textbf{Differential cross sections for prompt diphoton production}\newline
\textbf{Experiment:} CDF (Tevatron Run 2) \newline
\textbf{Spires ID:} \href{http://www.slac.stanford.edu/spires/find/hep/www?rawcmd=key+6080774}{6080774}\newline
\textbf{Status:} VALIDATED\newline
\textbf{Authors:}
 \penalty 100
\begin{itemize}
  \item Frank Siegert $\langle\,$\href{mailto:frank.siegert@durham.ac.uk}{frank.siegert@durham.ac.uk}$\,\rangle$;
\end{itemize}
\textbf{References:}
 \penalty 100
\begin{itemize}
  \item Phys. Rev. Lett. 95, 022003
  \item DOI: \href{http://dx.doi.org/10.1103/PhysRevLett.95.022003}{10.1103/PhysRevLett.95.022003}
  \item arXiv: \href{http://arxiv.org/abs/hep-ex/0412050}{hep-ex/0412050}
\end{itemize}
\textbf{Run details:}
 \penalty 100
\begin{itemize}

  \item $p \bar{p} \to \gamma \gamma$ [+ jets] at 1960 GeV. The analysis uses photons with \pT larger then 13 GeV. To allow for shifts in the shower, the ME cut on the transverse photon momentum shouldn't be too hard, e.g. 5 GeV.\end{itemize}

\noindent Measurement of the cross section of prompt diphoton production in $p\bar{p}$ collisions at $\sqrt{s} = 1.96$ TeV using a data sample of 207~pb$^{-1}$ as a function of the diphoton mass, the transverse momentum of the diphoton system, and the azimuthal angle between the two photons.

\clearpage


\clearpage

\typeout{Handling analysis CDF_2005_S6217184}
\subsection[CDF\_2005\_S6217184]{CDF\_2005\_S6217184\,\cite{Acosta:2005ix}}
\textbf{CDF Run II jet shape analysis}\newline
\textbf{Experiment:} CDF (Tevatron Run 2) \newline
\textbf{Spires ID:} \href{http://www.slac.stanford.edu/spires/find/hep/www?rawcmd=key+6217184}{6217184}\newline
\textbf{Status:} VALIDATED\newline
\textbf{Authors:}
 \penalty 100
\begin{itemize}
  \item Lars Sonnenschein $\langle\,$\href{mailto:Lars.Sonnenschein@cern.ch}{Lars.Sonnenschein@cern.ch}$\,\rangle$;
  \item Andy Buckley $\langle\,$\href{mailto:andy.buckley@cern.ch}{andy.buckley@cern.ch}$\,\rangle$;
\end{itemize}
\textbf{References:}
 \penalty 100
\begin{itemize}
  \item Phys.Rev.D71:112002,2005
  \item DOI: \href{http://dx.doi.org/10.1103/PhysRevD.71.112002}{10.1103/PhysRevD.71.112002}
  \item arXiv: \href{http://arxiv.org/abs/hep-ex/0505013}{hep-ex/0505013}
\end{itemize}
\textbf{Run details:}
 \penalty 100
\begin{itemize}

  \item QCD events at \ensuremath{\sqrt{s}} = 1960 GeV. Jet \pTmin in plots is 37 GeV/c --- choose generator min \pT somewhere well below this.\end{itemize}

\noindent Measurement of jet shapes in inclusive jet production in p pbar collisions at center-of-mass energy \ensuremath{\sqrt{s}} = 1.96 TeV. The data cover jet transverse momenta from 37--380 GeV and absolute jet rapidities in the range 0.1--0.7.

\clearpage


\clearpage

\typeout{Handling analysis CDF_2006_S6450792}
\subsection[CDF\_2006\_S6450792]{CDF\_2006\_S6450792\,\cite{Abulencia:2005yg}}
\textbf{Inclusive jet cross section differential in \pT}\newline
\textbf{Experiment:} CDF (Tevatron Run 2) \newline
\textbf{Spires ID:} \href{http://www.slac.stanford.edu/spires/find/hep/www?rawcmd=key+6450792}{6450792}\newline
\textbf{Status:} VALIDATED\newline
\textbf{Authors:}
 \penalty 100
\begin{itemize}
  \item Frank Siegert $\langle\,$\href{mailto:frank.siegert@durham.ac.uk}{frank.siegert@durham.ac.uk}$\,\rangle$;
\end{itemize}
\textbf{References:}
 \penalty 100
\begin{itemize}
  \item Phys.Rev.D74:071103,2006
  \item DOI: \href{http://dx.doi.org/10.1103/PhysRevD.74.071103}{10.1103/PhysRevD.74.071103}
  \item arXiv: \href{http://arxiv.org/abs/hep-ex/0512020}{hep-ex/0512020}
\end{itemize}
\textbf{Run details:}
 \penalty 100
\begin{itemize}

  \item $p\bar{p}$ \ensuremath{\to} jets at 1960 GeV\end{itemize}

\noindent Measurement of the inclusive jet cross section in ppbar interactions at $\sqrt{s}=1.96$ TeV using 385 $\mathrm{pb}^{-1}$ of data. The data cover the jet transverse momentum range from 61 to 620 GeV/c in $0.1 < |y| < 0.7$. This analysis has been updated with more data in more rapidity bins in CDF_2008_S7828950.

\clearpage


\clearpage

\typeout{Handling analysis CDF_2007_S7057202}
\subsection[CDF\_2007\_S7057202]{CDF\_2007\_S7057202\,\cite{Abulencia:2007ez}}
\textbf{CDF Run II inclusive jet cross-section using the kT algorithm}\newline
\textbf{Experiment:} CDF (Tevatron Run 2) \newline
\textbf{Spires ID:} \href{http://www.slac.stanford.edu/spires/find/hep/www?rawcmd=key+7057202}{7057202}\newline
\textbf{Status:} VALIDATED\newline
\textbf{Authors:}
 \penalty 100
\begin{itemize}
  \item David Voong
  \item Frank Siegert $\langle\,$\href{mailto:frank.siegert@durham.ac.uk}{frank.siegert@durham.ac.uk}$\,\rangle$;
\end{itemize}
\textbf{References:}
 \penalty 100
\begin{itemize}
  \item Phys.Rev.D75:092006,2007
  \item Erratum-ibid.D75:119901,2007
  \item FNAL-PUB 07/026-E
  \item hep-ex/0701051
\end{itemize}
\textbf{Run details:}
 \penalty 100
\begin{itemize}

  \item p-pbar collisions at 1960~GeV. Jet \pT bins from 54~GeV to 700~GeV. Jet rapidity $< |2.1|$.\end{itemize}

\noindent CDF Run II measurement of inclusive jet cross sections at a p-pbar collision energy of 1.96~TeV. Jets are reconstructed in the central part of the detector ($|y|<2.1$) using the kT algorithm with an $R$ parameter of 0.7. The minimum jet \pT considered is 54~GeV, with a maximum around 700~GeV.  The inclusive jet \pT is plotted in bins of rapidity $|y|<0.1$, $0.1<|y|<0.7$, $0.7<|y|<1.1$, $1.1<|y|<1.6$ and $1.6<|y|<2.1$.

\clearpage


\clearpage

\typeout{Handling analysis CDF_2008_LEADINGJETS}
\subsection{CDF\_2008\_LEADINGJETS}
\textbf{CDF Run 2 underlying event in leading jet events}\newline
\textbf{Experiment:} CDF (Tevatron Run 2) \newline
\textbf{Spires ID:} \href{http://www.slac.stanford.edu/spires/find/hep/www?rawcmd=key+NONE}{NONE}\newline
\textbf{Status:} VALIDATED\newline
\textbf{Authors:}
 \penalty 100
\begin{itemize}
  \item Hendrik Hoeth $\langle\,$\href{mailto:hendrik.hoeth@cern.ch}{hendrik.hoeth@cern.ch}$\,\rangle$;
\end{itemize}
\textbf{No references listed}\\ 
\textbf{Run details:}
 \penalty 100
\begin{itemize}

  \item $p\bar{p}$ QCD interactions at 1960~GeV. Particles with $c \tau > {}$10 mm should be set stable. Several $p_\perp^\text{min}$ cutoffs are probably required to fill the profile histograms. $p_\perp^\text{min} = {}$ 0 (min bias), 10, 20, 50, 100, 150 GeV. The corresponding merging points are at $p_T = $ 0, 30, 50, 80, 130, 180 GeV\end{itemize}

\noindent Rick Field's measurement of the underlying event in leading jet events. If the leading jet of the event is within $|\eta| < 2$, the event is accepted and ``toward'', ``away'' and ``transverse'' regions are defined in the same way as in the original (2001) CDF underlying event analysis. The leading jet defines the $\phi$ direction of the toward region. The transverse regions are most sensitive to the underlying event.

\clearpage


\clearpage

\typeout{Handling analysis CDF_2008_NOTE_9351}
\subsection{CDF\_2008\_NOTE\_9351}
\textbf{CDF Run 2 underlying event in Drell-Yan}\newline
\textbf{Experiment:} CDF (Tevatron Run 2) \newline
\textbf{Spires ID:} \href{http://www.slac.stanford.edu/spires/find/hep/www?rawcmd=key+NONE}{NONE}\newline
\textbf{Status:} VALIDATED\newline
\textbf{Authors:}
 \penalty 100
\begin{itemize}
  \item Hendrik Hoeth $\langle\,$\href{mailto:hendrik.hoeth@cern.ch}{hendrik.hoeth@cern.ch}$\,\rangle$;
\end{itemize}
\textbf{References:}
 \penalty 100
\begin{itemize}
  \item CDF public note 9351
\end{itemize}
\textbf{Run details:}
 \penalty 100
\begin{itemize}

  \item ppbar collisions at 1960 GeV.
  \item Drell-Yan events with $Z/\gamma* \ensuremath{\to} e e$ and $Z/\gamma* \ensuremath{\to} \mu\mu$.
  \item A mass cut $m_{ll} > 70~\text{GeV}$ can be applied on generator level.
  \item Particles with $c \tau > 10~\text{mm}$ should be set stable.\end{itemize}

\noindent Deepak Kar and Rick Field's measurement of the underlying event in Drell-Yan events. $Z \ensuremath{\to} ee$ and $Z \ensuremath{\to} \mu\mu$ events are selected using a $Z$ mass window cut between 70 and 110~GeV. ``Toward'', ``away'' and ``transverse'' regions are defined in the same way as in the original (2001) CDF underlying event analysis. The reconstructed $Z$ defines the $\phi$ direction of the toward region. The leptons are ignored after the $Z$ has been reconstructed. Thus the region most sensitive to the underlying event is the toward region (the recoil jet is boosted into the away region).

\clearpage


\clearpage

\typeout{Handling analysis CDF_2008_S7540469}
\subsection[CDF\_2008\_S7540469]{CDF\_2008\_S7540469\,\cite{:2007cp}}
\textbf{Measurement of differential Z/$\gamma^*$ + jet + X cross sections}\newline
\textbf{Experiment:} CDF (Tevatron Run 2) \newline
\textbf{Spires ID:} \href{http://www.slac.stanford.edu/spires/find/hep/www?rawcmd=key+7540469}{7540469}\newline
\textbf{Status:} VALIDATED\newline
\textbf{Authors:}
 \penalty 100
\begin{itemize}
  \item Frank Siegert $\langle\,$\href{mailto:frank.siegert@durham.ac.uk}{frank.siegert@durham.ac.uk}$\,\rangle$;
\end{itemize}
\textbf{References:}
 \penalty 100
\begin{itemize}
  \item Phys.Rev.Lett.100:102001,2008
  \item arXiv: \href{http://arxiv.org/abs/0711.3717}{0711.3717}
\end{itemize}
\textbf{Run details:}
 \penalty 100
\begin{itemize}

  \item $p \bar{p} \to e^+ e^-$ + jets at 1960 GeV. Needs mass cut on lepton pair to avoid photon singularity, looser than $66 < m_{ee} < 116$\end{itemize}

\noindent Cross sections as a function of jet transverse momentum in 1 and 2 jet events, and jet multiplicity in ppbar collisions at \ensuremath{\sqrt{s}} = 1.96 TeV, based on an integrated luminosity of $1.7~\text{fb}^{-1}$. The measurements cover the rapidity region $|y_\text{jet}| < 2.1$ and the transverse momentum range $\pT^\text{jet} > 30~\text{GeV}/c$.

\clearpage


\clearpage

\typeout{Handling analysis CDF_2008_S7828950}
\subsection[CDF\_2008\_S7828950]{CDF\_2008\_S7828950\,\cite{Aaltonen:2008eq}}
\textbf{CDF Run II inclusive jet cross-section using the Midpoint algorithm}\newline
\textbf{Experiment:} CDF (Tevatron Run 2) \newline
\textbf{Spires ID:} \href{http://www.slac.stanford.edu/spires/find/hep/www?rawcmd=key+7828950}{7828950}\newline
\textbf{Status:} VALIDATED\newline
\textbf{Authors:}
 \penalty 100
\begin{itemize}
  \item Craig Group $\langle\,$\href{mailto:group@fnal.gov}{group@fnal.gov}$\,\rangle$;
  \item Frank Siegert $\langle\,$\href{mailto:frank.siegert@durham.ac.uk}{frank.siegert@durham.ac.uk}$\,\rangle$;
\end{itemize}
\textbf{References:}
 \penalty 100
\begin{itemize}
  \item arXiv: \href{http://arxiv.org/abs/0807.2204}{0807.2204}
  \item Phys.Rev.D78:052006,2008
\end{itemize}
\textbf{Run details:}
 \penalty 100
\begin{itemize}

  \item Requires $2\rightarrow{2}$ QCD scattering processes. The minimum jet $E_\perp$ is 62 GeV, so a cut on kinematic \pTmin may be required for good statistics.\end{itemize}

\noindent Measurement of the inclusive jet cross section in $p\bar{p}$ collisions at $\sqrt{s}=1.96$ TeV as a function of jet $E_\perp$, for $E_\perp >$ 62 GeV. The data is collected by the CDF II detector and has an integrated luminosity of 1.13 fb$^{-1}$. The measurement was made using the cone-based Midpoint jet clustering algorithm in rapidity bins within $|y|<2.1$. This measurement can be used to provide increased precision in PDFs at high parton momentum fraction $x$.

\clearpage


\clearpage

\typeout{Handling analysis CDF_2008_S8093652}
\subsection[CDF\_2008\_S8093652]{CDF\_2008\_S8093652\,\cite{Aaltonen:2008dn}}
\textbf{Dijet mass spectrum}\newline
\textbf{Experiment:} CDF (Tevatron Run 2) \newline
\textbf{Spires ID:} \href{http://www.slac.stanford.edu/spires/find/hep/www?rawcmd=key+8093652}{8093652}\newline
\textbf{Status:} VALIDATED\newline
\textbf{Authors:}
 \penalty 100
\begin{itemize}
  \item Frank Siegert $\langle\,$\href{mailto:frank.siegert@durham.ac.uk}{frank.siegert@durham.ac.uk}$\,\rangle$;
\end{itemize}
\textbf{References:}
 \penalty 100
\begin{itemize}
  \item arXiv: \href{http://arxiv.org/abs/0812.4036}{0812.4036}
\end{itemize}
\textbf{Run details:}
 \penalty 100
\begin{itemize}

  \item $p \bar{p} \to$ jets at 1960 GeV\end{itemize}

\noindent Dijet mass spectrum  from 0.2 TeV to 1.4 TeV in $p \bar{p}$ collisions at $\sqrt{s} = 1.96$ TeV, based on an integrated luminosity of 1.13 fb$^{-1}$.

\clearpage


\clearpage

\typeout{Handling analysis CDF_2009_S8233977}
\subsection[CDF\_2009\_S8233977]{CDF\_2009\_S8233977\,\cite{Aaltonen:2009ne}}
\textbf{CDF Run 2 min bias cross-section analysis}\newline
\textbf{Experiment:} CDF (Tevatron Run 2) \newline
\textbf{Spires ID:} \href{http://www.slac.stanford.edu/spires/find/hep/www?rawcmd=key+8233977}{8233977}\newline
\textbf{Status:} VALIDATED\newline
\textbf{Authors:}
 \penalty 100
\begin{itemize}
  \item Hendrik Hoeth $\langle\,$\href{mailto:hendrik.hoeth@cern.ch}{hendrik.hoeth@cern.ch}$\,\rangle$;
  \item Niccolo' Moggi $\langle\,$\href{mailto:niccolo.moggi@bo.infn.it}{niccolo.moggi@bo.infn.it}$\,\rangle$;
\end{itemize}
\textbf{References:}
 \penalty 100
\begin{itemize}
  \item Phys.Rev.D79:112005,2009
  \item DOI: \href{http://dx.doi.org/10.1103/PhysRevD.79.112005}{10.1103/PhysRevD.79.112005}
  \item arXiv: \href{http://arxiv.org/abs/0904.1098}{0904.1098}
\end{itemize}
\textbf{Run details:}
 \penalty 100
\begin{itemize}

  \item $p\bar{p}$ QCD interactions at 1960~GeV. Particles with $c \\tau > {}$10 mm should be set stable.\end{itemize}

\noindent Niccolo Moggi's min bias analysis. Minimum bias events are used to measure the average track \pT vs. charged multiplicity, a track \pT distribution and an inclusive $\sum E_T$ distribution.

\clearpage


\clearpage

\typeout{Handling analysis CDF_2009_S8383952}
\subsection[CDF\_2009\_S8383952]{CDF\_2009\_S8383952\,\cite{Aaltonen:2009pc}}
\textbf{Z rapidity measurement}\newline
\textbf{Experiment:} CDF (Tevatron Run 2) \newline
\textbf{Spires ID:} \href{http://www.slac.stanford.edu/spires/find/hep/www?rawcmd=key+8383952}{8383952}\newline
\textbf{Status:} VALIDATED\newline
\textbf{Authors:}
 \penalty 100
\begin{itemize}
  \item Frank Siegert $\langle\,$\href{mailto:frank.siegert@durham.ac.uk}{frank.siegert@durham.ac.uk}$\,\rangle$;
\end{itemize}
\textbf{References:}
 \penalty 100
\begin{itemize}
  \item arXiv: \href{http://arxiv.org/abs/0908.3914}{0908.3914}
\end{itemize}
\textbf{Run details:}
 \penalty 100
\begin{itemize}

  \item $p \bar{p} \to e^+ e^-$ + jets at 1960 GeV. Needs mass cut on lepton pair to avoid photon singularity, looser than $66 < m_{ee} < 116$ GeV\end{itemize}

\noindent CDF measurement of the total cross section and rapidity distribution, $\mathrm{d}\sigma/\mathrm{d}y$, for $q\bar{q}\to \gamma^{*}/Z\to e^{+}e^{-}$ events in the $Z$ boson mass region ($66<M_{ee}<116$ GeV/c$^2$) produced in $p\bar{p}$ collisions at $\sqrt{s}=1.96$ TeV with 2.1 fb$^{-1}$ of integrated luminosity.

\clearpage


\clearpage

\typeout{Handling analysis CDF_2009_S8436959}
\subsection[CDF\_2009\_S8436959]{CDF\_2009\_S8436959\,\cite{Aaltonen:2009ty}}
\textbf{Measurement of the inclusive isolated prompt photon cross-section}\newline
\textbf{Experiment:} CDF (Tevatron Run 2) \newline
\textbf{Spires ID:} \href{http://www.slac.stanford.edu/spires/find/hep/www?rawcmd=key+8436959}{8436959}\newline
\textbf{Status:} VALIDATED\newline
\textbf{Authors:}
 \penalty 100
\begin{itemize}
  \item Frank Siegert $\langle\,$\href{mailto:frank.siegert@durham.ac.uk}{frank.siegert@durham.ac.uk}$\,\rangle$;
\end{itemize}
\textbf{References:}
 \penalty 100
\begin{itemize}
  \item arXiv: \href{http://arxiv.org/abs/0910.3623}{0910.3623}
\end{itemize}
\textbf{Run details:}
 \penalty 100
\begin{itemize}

  \item $\gamma$ + jet processes in ppbar collisions at $\sqrt{s} = 1960$~GeV. Minimum \pT cut on the photon in the analysis is 30~GeV.\end{itemize}

\noindent A measurement of the cross section for the inclusive production of isolated photons. The measurement covers the pseudorapidity region $|\eta^\gamma|<1.0$ and the transverse energy range $E_T^\gamma>30$~GeV and is based on 2.5~fb$^{-1}$ of integrated luminosity. The cross section is measured differential in $E_\perp(\gamma)$.

\clearpage


\clearpage

\typeout{Handling analysis D0_2001_S4674421}
\subsection[D0\_2001\_S4674421]{D0\_2001\_S4674421\,\cite{Abazov:2001nta}}
\textbf{Tevatron Run I differential W/Z boson cross-section analysis}\newline
\textbf{Experiment:} D0 (Tevatron Run 1) \newline
\textbf{Spires ID:} \href{http://www.slac.stanford.edu/spires/find/hep/www?rawcmd=key+4674421}{4674421}\newline
\textbf{Status:} VALIDATED\newline
\textbf{Authors:}
 \penalty 100
\begin{itemize}
  \item Lars Sonnenschein $\langle\,$\href{mailto:Lars.Sonnenschein@cern.ch}{Lars.Sonnenschein@cern.ch}$\,\rangle$;
\end{itemize}
\textbf{References:}
 \penalty 100
\begin{itemize}
  \item Phys.Lett.B517:299-308,2001
  \item DOI: \href{http://dx.doi.org/10.1016/S0370-2693(01)01020-6}{10.1016/S0370-2693(01)01020-6}
  \item arXiv: \href{http://arxiv.org/abs/hep-ex/0107012v2}{hep-ex/0107012v2}
\end{itemize}
\textbf{Run details:}
 \penalty 100
\begin{itemize}

  \item W/Z events with decays to first generation leptons, in ppbar collisions at \ensuremath{\sqrt{s}} = 1800~GeV\end{itemize}

\noindent Measurement of differential W/Z boson cross section and ratio in $p \bar{p}$ collisions at center-of-mass energy \ensuremath{\sqrt{s}} = 1.8 TeV. The data cover electrons and neutrinos in a pseudo-rapidity range of -2.5 to 2.5.

\clearpage


\clearpage

\typeout{Handling analysis D0_2004_S5992206}
\subsection[D0\_2004\_S5992206]{D0\_2004\_S5992206\,\cite{Abazov:2004hm}}
\textbf{Run II jet azimuthal decorrelation analysis}\newline
\textbf{Experiment:} D0 (Tevatron Run 2) \newline
\textbf{Spires ID:} \href{http://www.slac.stanford.edu/spires/find/hep/www?rawcmd=key+5992206}{5992206}\newline
\textbf{Status:} VALIDATED\newline
\textbf{Authors:}
 \penalty 100
\begin{itemize}
  \item Lars Sonnenschein $\langle\,$\href{mailto:lars.sonnenschein@cern.ch}{lars.sonnenschein@cern.ch}$\,\rangle$;
\end{itemize}
\textbf{References:}
 \penalty 100
\begin{itemize}
  \item Phys. Rev. Lett., 94, 221801 (2005)
  \item arXiv: \href{http://arxiv.org/abs/hep-ex/0409040}{hep-ex/0409040}
\end{itemize}
\textbf{Run details:}
 \penalty 100
\begin{itemize}

  \item QCD events in ppbar interactions at \ensuremath{\sqrt{s}} = 1960 GeV.\end{itemize}

\noindent Correlations in the azimuthal angle between the two largest \pT jets have been measured using the D0 detector in ppbar collisions at 1960~GeV. The analysis is based on an inclusive dijet event sample in the central rapidity region. The correlations are determined for four different \pT intervals.

\clearpage


\clearpage

\typeout{Handling analysis D0_2006_S6438750}
\subsection[D0\_2006\_S6438750]{D0\_2006\_S6438750\,\cite{Abazov:2005wc}}
\textbf{Inclusive isolated photon cross-section, differential in \pT(gamma)}\newline
\textbf{Experiment:} D0 (Tevatron Run 2) \newline
\textbf{Spires ID:} \href{http://www.slac.stanford.edu/spires/find/hep/www?rawcmd=key+6438750}{6438750}\newline
\textbf{Status:} VALIDATED\newline
\textbf{Authors:}
 \penalty 100
\begin{itemize}
  \item Andy Buckley $\langle\,$\href{mailto:andy.buckley@durham.ac.uk}{andy.buckley@durham.ac.uk}$\,\rangle$;
  \item Gavin Hesketh $\langle\,$\href{mailto:gavin.hesketh@cern.ch}{gavin.hesketh@cern.ch}$\,\rangle$;
\end{itemize}
\textbf{References:}
 \penalty 100
\begin{itemize}
  \item Phys.Lett.B639:151-158,2006, Erratum-ibid.B658:285-289,2008
  \item DOI: \href{http://dx.doi.org/10.1016/j.physletb.2006.04.048}{10.1016/j.physletb.2006.04.048}
  \item arXiv: \href{http://arxiv.org/abs/hep-ex/0511054}{hep-ex/0511054}
\end{itemize}
\textbf{Run details:}
 \penalty 100
\begin{itemize}

  \item ppbar collisions at \ensuremath{\sqrt{s}} = 1960 GeV. Requires gamma + jet (q,qbar,g) hard processes, which for Pythia 6 means MSEL=10 for with MSUB indices 14, 18, 29, 114, 115 enabled.\end{itemize}

\noindent Measurement of differential cross section for inclusive production of isolated photons in p pbar collisions at \ensuremath{\sqrt{s}} = 1.96 TeV with the D\O detector at the Fermilab Tevatron collider. The photons span transverse momenta 23--300 GeV and have pseudorapidity $|\eta| < 0.9$. Isolated direct photons are probes of pQCD via the annihilation ($q \bar{q} \ensuremath{\to} \gamma g$) and quark-gluon Compton scattering ($q g \ensuremath{\to} \gamma q$) processes, the latter of which is also sensitive to the gluon PDF. The initial state radiation / resummation formalisms are sensitive to the resulting photon \pT spectrum

\clearpage


\clearpage

\typeout{Handling analysis D0_2007_S7075677}
\subsection[D0\_2007\_S7075677]{D0\_2007\_S7075677\,\cite{Abazov:2007jy}}
\textbf{$Z/\gamma^* + X$ cross-section shape, differential in $y(Z)$}\newline
\textbf{Experiment:} D0 (Tevatron Run 2) \newline
\textbf{Spires ID:} \href{http://www.slac.stanford.edu/spires/find/hep/www?rawcmd=key+7075677}{7075677}\newline
\textbf{Status:} VALIDATED\newline
\textbf{Authors:}
 \penalty 100
\begin{itemize}
  \item Andy Buckley $\langle\,$\href{mailto:andy.buckley@durham.ac.uk}{andy.buckley@durham.ac.uk}$\,\rangle$;
  \item Gavin Hesketh $\langle\,$\href{mailto:ghesketh@fnal.gov}{ghesketh@fnal.gov}$\,\rangle$;
  \item Frank Siegert $\langle\,$\href{mailto:frank.siegert@durham.ac.uk}{frank.siegert@durham.ac.uk}$\,\rangle$;
\end{itemize}
\textbf{References:}
 \penalty 100
\begin{itemize}
  \item Phys.Rev.D76:012003,2007
  \item arXiv: \href{http://arxiv.org/abs/hep-ex/0702025}{hep-ex/0702025}
\end{itemize}
\textbf{Run details:}
 \penalty 100
\begin{itemize}

  \item Drell-Yan $p \bar{p} \to Z/\gamma^*$ + jets events at $\sqrt{s}$ = 1960 GeV. Needs mass cut on lepton pair to avoid photon singularity, looser than  $71 < m_{ee} < 111$ GeV\end{itemize}

\noindent Cross sections as a function of boson rapidity in $p \bar{p}$ collisions at $\sqrt{s}$ = 1.96 TeV, based on an integrated luminosity of $0.4~\text{fb}^{-1}$.

\clearpage


\clearpage

\typeout{Handling analysis D0_2008_S6879055}
\subsection[D0\_2008\_S6879055]{D0\_2008\_S6879055\,\cite{Abazov:2006gs}}
\textbf{Measurement of the ratio sigma($Z/\gamma^*$ + $n$ jets)/sigma($Z/\gamma^*$)}\newline
\textbf{Experiment:} D0 (Tevatron Run 2) \newline
\textbf{Spires ID:} \href{http://www.slac.stanford.edu/spires/find/hep/www?rawcmd=key+6879055}{6879055}\newline
\textbf{Status:} VALIDATED\newline
\textbf{Authors:}
 \penalty 100
\begin{itemize}
  \item Giulio Lenzi
  \item Frank Siegert $\langle\,$\href{mailto:frank.siegert@durham.ac.uk}{frank.siegert@durham.ac.uk}$\,\rangle$;
\end{itemize}
\textbf{References:}
 \penalty 100
\begin{itemize}
  \item hep-ex/0608052
\end{itemize}
\textbf{Run details:}
 \penalty 100
\begin{itemize}

  \item $p \bar{p} \to e^+ e^-$ + jets at 1960~GeV. Needs mass cut on lepton pair to avoid photon singularity, looser than $75 < m_{ee} < 105$ GeV.\end{itemize}

\noindent Cross sections as a function of \pT of the three leading jets and $n$-jet cross section ratios in $p \bar{p}$ collisions at $\sqrt{s}$ = 1.96 TeV, based on an integrated luminosity of $0.4~\text{fb}^{-1}$.

\clearpage


\clearpage

\typeout{Handling analysis D0_2008_S7554427}
\subsection[D0\_2008\_S7554427]{D0\_2008\_S7554427\,\cite{:2007nt}}
\textbf{$Z/\gamma^* + X$ cross-section shape, differential in $\pT(Z)$}\newline
\textbf{Experiment:} D0 (Tevatron Run 2) \newline
\textbf{Spires ID:} \href{http://www.slac.stanford.edu/spires/find/hep/www?rawcmd=key+7554427}{7554427}\newline
\textbf{Status:} VALIDATED\newline
\textbf{Authors:}
 \penalty 100
\begin{itemize}
  \item Andy Buckley $\langle\,$\href{mailto:andy.buckley@durham.ac.uk}{andy.buckley@durham.ac.uk}$\,\rangle$;
  \item Frank Siegert $\langle\,$\href{mailto:frank.siegert@durham.ac.uk}{frank.siegert@durham.ac.uk}$\,\rangle$;
\end{itemize}
\textbf{References:}
 \penalty 100
\begin{itemize}
  \item arXiv: \href{http://arxiv.org/abs/0712.0803}{0712.0803}
\end{itemize}
\textbf{Run details:}
 \penalty 100
\begin{itemize}

  \item * $p \bar{p} \to e^+ e^-$ + jets at 1960~GeV.
  \item Needs mass cut on lepton pair to avoid photon singularity, looser than $40 < m_{ee} < 200$ GeV.\end{itemize}

\noindent Cross sections as a function of \pT of the vector boson inclusive and in forward region ($|y| > 2$, $\pT<30$ GeV) in $p \bar{p}$ collisions at $\sqrt{s}$ = 1.96 TeV, based on an integrated luminosity of 0.98~fb$^{-1}$.

\clearpage


\clearpage

\typeout{Handling analysis D0_2008_S7662670}
\subsection[D0\_2008\_S7662670]{D0\_2008\_S7662670\,\cite{:2008hua}}
\textbf{Measurement of D0 Run II differential jet cross sections}\newline
\textbf{Experiment:} D0 (Tevatron Run 2) \newline
\textbf{Spires ID:} \href{http://www.slac.stanford.edu/spires/find/hep/www?rawcmd=key+7662670}{7662670}\newline
\textbf{Status:} VALIDATED\newline
\textbf{Authors:}
 \penalty 100
\begin{itemize}
  \item Andy Buckley $\langle\,$\href{mailto:andy.buckley@durham.ac.uk}{andy.buckley@durham.ac.uk}$\,\rangle$;
  \item Gavin Hesketh $\langle\,$\href{mailto:gavin.hesketh@cern.ch}{gavin.hesketh@cern.ch}$\,\rangle$;
\end{itemize}
\textbf{References:}
 \penalty 100
\begin{itemize}
  \item Phys.Rev.Lett.101:062001,2008
  \item DOI: \href{http://dx.doi.org/10.1103/PhysRevLett.101.062001}{10.1103/PhysRevLett.101.062001}
  \item arXiv: \href{http://arxiv.org/abs/0802.2400v3}{0802.2400v3}
\end{itemize}
\textbf{Run details:}
 \penalty 100
\begin{itemize}

  \item QCD events at \ensuremath{\sqrt{s}} = 1960 GeV. A \pTmin cut is probably necessary since the lowest jet \pT bin is at 50 GeV\end{itemize}

\noindent Measurement of the inclusive jet cross section in $p \bar{p}$ collisions at center-of-mass energy \ensuremath{\sqrt{s}} = 1.96 TeV. The data cover jet transverse momenta from 50--600 GeV and jet rapidities in the range -2.4 to 2.4.

\clearpage


\clearpage

\typeout{Handling analysis D0_2008_S7719523}
\subsection[D0\_2008\_S7719523]{D0\_2008\_S7719523\,\cite{Abazov:2008er}}
\textbf{Isolated $\gamma$ + jet cross-sections, differential in \pT($\gamma$) for various $y$ bins}\newline
\textbf{Experiment:} D0 (Tevatron Run 2) \newline
\textbf{Spires ID:} \href{http://www.slac.stanford.edu/spires/find/hep/www?rawcmd=key+7719523}{7719523}\newline
\textbf{Status:} VALIDATED\newline
\textbf{Authors:}
 \penalty 100
\begin{itemize}
  \item Andy Buckley $\langle\,$\href{mailto:andy.buckley@durham.ac.uk}{andy.buckley@durham.ac.uk}$\,\rangle$;
  \item Gavin Hesketh $\langle\,$\href{mailto:gavin.hesketh@cern.ch}{gavin.hesketh@cern.ch}$\,\rangle$;
\end{itemize}
\textbf{References:}
 \penalty 100
\begin{itemize}
  \item Phys.Lett.B666:435-445,2008
  \item DOI: \href{http://dx.doi.org/10.1016/j.physletb.2008.06.076}{10.1016/j.physletb.2008.06.076}
  \item arXiv: \href{http://arxiv.org/abs/0804.1107v2}{0804.1107v2}
\end{itemize}
\textbf{Run details:}
 \penalty 100
\begin{itemize}

  \item Produce only gamma + jet ($q,\bar{q},g$) hard processes (for Pythia 6, this means MSEL=10 and MSUB indices 14, 29 \& 115 enabled). The lowest bin edge is at 30 GeV, so a kinematic \pTmin cut is probably required to fill the histograms.\end{itemize}

\noindent The process $p \bar{p}$ \ensuremath{\to} photon + jet + X as studied by the D0 detector at the Fermilab Tevatron collider at center-of-mass energy \ensuremath{\sqrt{s}} = 1.96 TeV. Photons are reconstructed in the central rapidity region $|y_\gamma| < 1.0$ with transverse momenta in the range 30--400 GeV, while jets are reconstructed in either the central $|y_\text{jet}| < 0.8$ or forward $1.5 < |y_\text{jet}| < 2.5$ rapidity intervals with $\pT^\text{jet} > 15~\text{GeV}$. The differential cross section $\mathrm{d}^3 \sigma / \mathrm{d}{\pT^\gamma} \mathrm{d}{y_\gamma} \mathrm{d}{y_\text{jet}}$ is measured as a function of $\pT^\gamma$ in four regions, differing by the relative orientations of the photon and the jet.  MC predictions have trouble with simultaneously describing the measured normalization and $\pT^\gamma$ dependence of the cross section in any of the four measured regions.

\clearpage


\clearpage

\typeout{Handling analysis D0_2008_S7837160}
\subsection[D0\_2008\_S7837160]{D0\_2008\_S7837160\,\cite{Abazov:2008qv}}
\textbf{Measurement of W charge asymmetry from D0 Run II}\newline
\textbf{Experiment:} D0 (Tevatron Run 2) \newline
\textbf{Spires ID:} \href{http://www.slac.stanford.edu/spires/find/hep/www?rawcmd=key+7837160}{7837160}\newline
\textbf{Status:} VALIDATED\newline
\textbf{Authors:}
 \penalty 100
\begin{itemize}
  \item Andy Buckley $\langle\,$\href{mailto:andy.buckley@durham.ac.uk}{andy.buckley@durham.ac.uk}$\,\rangle$;
  \item Gavin Hesketh $\langle\,$\href{mailto:gavin.hesketh@cern.ch}{gavin.hesketh@cern.ch}$\,\rangle$;
\end{itemize}
\textbf{References:}
 \penalty 100
\begin{itemize}
  \item Phys.Rev.Lett.101:211801,2008
  \item DOI: \href{http://dx.doi.org/10.1103/PhysRevLett.101.211801}{10.1103/PhysRevLett.101.211801}
  \item arXiv: \href{http://arxiv.org/abs/0807.3367v1}{0807.3367v1}
\end{itemize}
\textbf{Run details:}
 \penalty 100
\begin{itemize}

  \item * Event type: W production with decay to $e \, \nu_e$ only
  \item for Pythia 6: MSEL = 12, MDME(206,1) = 1
  \item Energy: 1.96 TeV\end{itemize}

\noindent Measurement of the electron charge asymmetry in $p \bar p \to W + X \to e \nu_e + X$ events at a center of mass energy of 1.96 TeV. The asymmetry is measured as a function of the electron transverse momentum and pseudorapidity in the interval (-3.2, 3.2).  This data is sensitive to proton parton distribution functions due to the valence asymmetry in the incoming quarks which produce the W. Initial state radiation should also affect the \pT distribution.

\clearpage


\clearpage

\typeout{Handling analysis D0_2008_S7863608}
\subsection[D0\_2008\_S7863608]{D0\_2008\_S7863608\,\cite{Abazov:2008ez}}
\textbf{Measurement of differential $Z/\gamma^*$ + jet + $X$ cross sections}\newline
\textbf{Experiment:} D0 (Tevatron Run 2) \newline
\textbf{Spires ID:} \href{http://www.slac.stanford.edu/spires/find/hep/www?rawcmd=key+7863608}{7863608}\newline
\textbf{Status:} VALIDATED\newline
\textbf{Authors:}
 \penalty 100
\begin{itemize}
  \item Andy Buckley $\langle\,$\href{mailto:andy.buckley@durham.ac.uk}{andy.buckley@durham.ac.uk}$\,\rangle$;
  \item Gavin Hesketh $\langle\,$\href{mailto:gavin.hesketh@fnal.gov}{gavin.hesketh@fnal.gov}$\,\rangle$;
  \item Frank Siegert $\langle\,$\href{mailto:frank.siegert@durham.ac.uk}{frank.siegert@durham.ac.uk}$\,\rangle$;
\end{itemize}
\textbf{References:}
 \penalty 100
\begin{itemize}
  \item arXiv: \href{http://arxiv.org/abs/0808.1296}{0808.1296}
\end{itemize}
\textbf{Run details:}
 \penalty 100
\begin{itemize}

  \item $p \bar{p} \to \mu^+ \mu^-$ + jets at 1960~GeV. Needs mass cut on lepton pair to avoid photon singularity, looser than $65 < m_{ee} < 115$ GeV.\end{itemize}

\noindent Cross sections as a function of \pT and rapidity of the boson and \pT and rapidity of the leading jet in $p \bar{p}$ collisions at $\sqrt{s}$ = 1.96 TeV, based on an integrated luminosity of 1.0 fb$^{-1}$.

\clearpage


\clearpage

\typeout{Handling analysis D0_2009_S8202443}
\subsection[D0\_2009\_S8202443]{D0\_2009\_S8202443\,\cite{Abazov:2009av}}
\textbf{$Z/\gamma^*$ + jet + $X$ cross sections differential in \pT(jet 1,2,3)}\newline
\textbf{Experiment:} D0 (Tevatron Run 2) \newline
\textbf{Spires ID:} \href{http://www.slac.stanford.edu/spires/find/hep/www?rawcmd=key+8202443}{8202443}\newline
\textbf{Status:} VALIDATED\newline
\textbf{Authors:}
 \penalty 100
\begin{itemize}
  \item Frank Siegert $\langle\,$\href{mailto:frank.siegert@durham.ac.uk}{frank.siegert@durham.ac.uk}$\,\rangle$;
\end{itemize}
\textbf{References:}
 \penalty 100
\begin{itemize}
  \item arXiv: \href{http://arxiv.org/abs/0903.1748}{0903.1748}
\end{itemize}
\textbf{Run details:}
 \penalty 100
\begin{itemize}

  \item $p \bar{p} \to e^+ e^-$ + jets at 1960~GeV. Needs mass cut on lepton pair to avoid photon singularity, looser than $65 < m_{ee} < 115$ GeV.\end{itemize}

\noindent Cross sections as a function of \pT of the three leading jets in $Z/\gamma^{*} (\to e^{+} e^{-})$ + jet + X production in $p \bar{p}$ collisions at $\sqrt{s} = 1.96$ TeV, based on an integrated luminosity of 1.0 fb$^{-1}$.

\clearpage


\clearpage

\typeout{Handling analysis D0_2009_S8320160}
\subsection[D0\_2009\_S8320160]{D0\_2009\_S8320160\,\cite{:2009mh}}
\textbf{Dijet angular distributions}\newline
\textbf{Experiment:} D0 (Tevatron Run 2) \newline
\textbf{Spires ID:} \href{http://www.slac.stanford.edu/spires/find/hep/www?rawcmd=key+8320160}{8320160}\newline
\textbf{Status:} VALIDATED\newline
\textbf{Authors:}
 \penalty 100
\begin{itemize}
  \item Frank Siegert $\langle\,$\href{mailto:frank.siegert@durham.ac.uk}{frank.siegert@durham.ac.uk}$\,\rangle$;
\end{itemize}
\textbf{References:}
 \penalty 100
\begin{itemize}
  \item arXiv: \href{http://arxiv.org/abs/0906.4819}{0906.4819}
\end{itemize}
\textbf{Run details:}
 \penalty 100
\begin{itemize}

  \item $p \bar{p} \to$ jets at 1960 GeV\end{itemize}

\noindent Dijet angular distributions in different bins of dijet mass from 0.25 TeV to above 1.1 TeV in $p \bar{p}$ collisions at $\sqrt{s} = 1.96$ TeV, based on an integrated luminosity of 0.7 fb$^{-1}$.

\clearpage


\clearpage

\typeout{Handling analysis D0_2009_S8349509}
\subsection[D0\_2009\_S8349509]{D0\_2009\_S8349509\,\cite{Abazov:2009pp}}
\textbf{Z+jets angular distributions}\newline
\textbf{Experiment:} D0 (Tevatron Run 2) \newline
\textbf{Spires ID:} \href{http://www.slac.stanford.edu/spires/find/hep/www?rawcmd=key+8349509}{8349509}\newline
\textbf{Status:} VALIDATED\newline
\textbf{Authors:}
 \penalty 100
\begin{itemize}
  \item Frank Siegert $\langle\,$\href{mailto:frank.siegert@durham.ac.uk}{frank.siegert@durham.ac.uk}$\,\rangle$;
\end{itemize}
\textbf{References:}
 \penalty 100
\begin{itemize}
  \item arXiv: \href{http://arxiv.org/abs/0907.4286}{0907.4286}
\end{itemize}
\textbf{Run details:}
 \penalty 100
\begin{itemize}

  \item $p \bar{p} \to \mu^+ \mu^-$ + jets at 1960~GeV. Needs mass cut on lepton pair to avoid photon singularity, looser than $65 < m_{ee} < 115$ GeV.\end{itemize}

\noindent First measurements at a hadron collider of differential cross sections for $Z$+jet+X production in $\Delta\phi(Z, j)$, $|\Delta y(Z, j)|$ and $|y_\mathrm{boost}(Z, j)|$. Vector boson production in association with jets is an excellent probe of QCD and constitutes the main background to many small cross section processes, such as associated Higgs production. These measurements are crucial tests of the predictions of perturbative QCD and current event generators, which have varied success in describing the data. Using these measurements as inputs in tuning event generators will increase the experimental sensitivity to rare signals.

\clearpage


\clearpage

\typeout{Handling analysis D0_2010_S8566488}
\subsection[D0\_2010\_S8566488]{D0\_2010\_S8566488\,\cite{Abazov:2010fr}}
\textbf{Dijet invariant mass}\newline
\textbf{Experiment:} D0 (Tevatron Run 2) \newline
\textbf{Spires ID:} \href{http://www.slac.stanford.edu/spires/find/hep/www?rawcmd=key+8566488}{8566488}\newline
\textbf{Status:} VALIDATED\newline
\textbf{Authors:}
 \penalty 100
\begin{itemize}
  \item Frank Siegert $\langle\,$\href{mailto:frank.siegert@durham.ac.uk}{frank.siegert@durham.ac.uk}$\,\rangle$;
\end{itemize}
\textbf{References:}
 \penalty 100
\begin{itemize}
  \item arXiv: \href{http://arxiv.org/abs/1002.4594}{1002.4594}
\end{itemize}
\textbf{Run details:}
 \penalty 100
\begin{itemize}

  \item $p \bar{p} \to$ jets at 1960 GeV. Analysis needs two hard jets above 40 GeV.\end{itemize}

\noindent The inclusive dijet production double differential cross section as a function of the dijet invariant mass and of the largest absolute rapidity ($|y|_\text{max}$) of the two jets with the largest transverse momentum in an event is measured using 0.7 fb$^{-1}$ of data. The measurement is performed in six rapidity regions up to $|y|_\text{max}=2.4$.

\clearpage


\clearpage

\typeout{Handling analysis D0_2010_S8570965}
\subsection[D0\_2010\_S8570965]{D0\_2010\_S8570965\,\cite{Abazov:2010ah}}
\textbf{Direct photon pair production}\newline
\textbf{Experiment:} CDF (Tevatron Run 2) \newline
\textbf{Spires ID:} \href{http://www.slac.stanford.edu/spires/find/hep/www?rawcmd=key+8570965}{8570965}\newline
\textbf{Status:} VALIDATED\newline
\textbf{Authors:}
 \penalty 100
\begin{itemize}
  \item Frank Siegert $\langle\,$\href{mailto:frank.siegert@durham.ac.uk}{frank.siegert@durham.ac.uk}$\,\rangle$;
\end{itemize}
\textbf{References:}
 \penalty 100
\begin{itemize}
  \item arXiv: \href{http://arxiv.org/abs/1002.4917}{1002.4917}
\end{itemize}
\textbf{Run details:}
 \penalty 100
\begin{itemize}

  \item All processes that can produce prompt photon pairs, e.g. $jj \to jj$, $jj \to j\gamma$ and $jj \to \gamma \gamma$. Non-prompt photons from hadron decays like $\pi$ and $\eta$ have been corrected for.\end{itemize}

\noindent Direct photon pair production cross sections are measured using 4.2 fb$^{-1}$ of data. They are binned in diphoton mass, the transverse momentum of the diphoton system, the azimuthal angle between the photons, and the polar scattering angle of the photons. Also available are double differential cross sections considering the last three kinematic variables in three diphoton mass bins.

\clearpage


\section{LHC analyses}\typeout{Handling analysis ATLAS_2010_S8591806}
\subsection[ATLAS\_2010\_S8591806]{ATLAS\_2010\_S8591806\,\cite{Aad:2010rd}}
\textbf{Charged particles at 900 GeV in ATLAS}\newline
\textbf{Experiment:} ATLAS (LHC 900GeV) \newline
\textbf{Spires ID:} \href{http://www.slac.stanford.edu/spires/find/hep/www?rawcmd=key+8591806}{8591806}\newline
\textbf{Status:} VALIDATED\newline
\textbf{Authors:}
 \penalty 100
\begin{itemize}
  \item Frank Siegert $\langle\,$\href{mailto:frank.siegert@durham.ac.uk}{frank.siegert@durham.ac.uk}$\,\rangle$;
\end{itemize}
\textbf{References:}
 \penalty 100
\begin{itemize}
  \item arXiv: \href{http://arxiv.org/abs/1003.3124}{1003.3124}
\end{itemize}
\textbf{Run details:}
 \penalty 100
\begin{itemize}

  \item pp QCD interactions at 900 GeV including diffractive events.\end{itemize}

\noindent The first measurements with the ATLAS detector at the LHC. Data were collected using a minimum-bias trigger in December 2009 during proton-proton collisions at a centre of mass energy of 900 GeV. The charged- particle density, its dependence on transverse momentum and pseudorapid- ity, and the relationship between transverse momentum and charged-particle multiplicity are measured for events with at least one charged particle in the kinematic range $|\eta| < 2.5$ and $p_\perp > 500$ MeV. All data is corrected to the particle level.

\clearpage


\section{SPS analyses}\typeout{Handling analysis UA1_1990_S2044935}
\subsection[UA1\_1990\_S2044935]{UA1\_1990\_S2044935\,\cite{Albajar:1989an}}
\textbf{UA1 multiplicities, transverse momenta and transverse energy distributions.}\newline
\textbf{Experiment:} UA1 (SPS) \newline
\textbf{Spires ID:} \href{http://www.slac.stanford.edu/spires/find/hep/www?rawcmd=key+2044935}{2044935}\newline
\textbf{Status:} VALIDATED\newline
\textbf{Authors:}
 \penalty 100
\begin{itemize}
  \item Andy Buckley $\langle\,$\href{mailto:andy.buckley@cern.ch}{andy.buckley@cern.ch}$\,\rangle$;
  \item Christophe Vaillant $\langle\,$\href{mailto:c.l.j.j.vaillant@durham.ac.uk}{c.l.j.j.vaillant@durham.ac.uk}$\,\rangle$;
\end{itemize}
\textbf{References:}
 \penalty 100
\begin{itemize}
  \item Nucl.Phys.B353:261,1990
\end{itemize}
\textbf{Run details:}
 \penalty 100
\begin{itemize}

  \item QCD min bias events at sqrtS = 63, 200, 500 and 900 GeV.\end{itemize}

\noindent Particle multiplicities, transverse momenta and transverse energy distributions at the UA1 experiment, at energies of 200, 500 and 900 GeV (with one plot at 63 GeV for comparison).

\clearpage


\clearpage

\typeout{Handling analysis UA5_1982_S875503}
\subsection[UA5\_1982\_S875503]{UA5\_1982\_S875503\,\cite{Alpgard:1982zx}}
\textbf{UA5 multiplicity and pseudorapidity distributions for $pp$ and $p\bar{p}$.}\newline
\textbf{Experiment:} UA5 (SPS) \newline
\textbf{Spires ID:} \href{http://www.slac.stanford.edu/spires/find/hep/www?rawcmd=key+875503}{875503}\newline
\textbf{Status:} VALIDATED\newline
\textbf{Authors:}
 \penalty 100
\begin{itemize}
  \item Andy Buckley $\langle\,$\href{mailto:andy.buckley@cern.ch}{andy.buckley@cern.ch}$\,\rangle$;
  \item Christophe Vaillant $\langle\,$\href{mailto:c.l.j.j.vaillant@durham.ac.uk}{c.l.j.j.vaillant@durham.ac.uk}$\,\rangle$;
\end{itemize}
\textbf{References:}
 \penalty 100
\begin{itemize}
  \item Phys.Lett.112B:183,1982
\end{itemize}
\textbf{Run details:}
 \penalty 100
\begin{itemize}

  \item Min bias QCD events at \ensuremath{\sqrt{s}} = 53~GeV. Run with both $pp$ and $p\bar{p}$ beams.\end{itemize}

\noindent Comparisons of multiplicity and pseudorapidity distributions for $pp$ and $p\bar{p}$ collisions at 53 GeV, based on the UA5 53~GeV runs in 1982. Data confirms the lack of significant difference between the two beams.

\clearpage


\clearpage

\typeout{Handling analysis UA5_1986_S1583476}
\subsection[UA5\_1986\_S1583476]{UA5\_1986\_S1583476\,\cite{Alner:1986xu}}
\textbf{Pseudorapidity distributions in $p\bar{p}$ (NSD, NSD+SD) events at \ensuremath{\sqrt{s}} = 200 and 900 GeV}\newline
\textbf{Experiment:} UA5 (CERN SPS) \newline
\textbf{Spires ID:} \href{http://www.slac.stanford.edu/spires/find/hep/www?rawcmd=key+1583476}{1583476}\newline
\textbf{Status:} VALIDATED\newline
\textbf{Authors:}
 \penalty 100
\begin{itemize}
  \item Andy Buckley $\langle\,$\href{mailto:andy.buckley@cern.ch}{andy.buckley@cern.ch}$\,\rangle$;
  \item Holger Schulz $\langle\,$\href{mailto:holger.schulz@physik.hu-berlin.de}{holger.schulz@physik.hu-berlin.de}$\,\rangle$;
  \item Christophe Vaillant $\langle\,$\href{mailto:c.l.j.j.vaillant@durham.ac.uk}{c.l.j.j.vaillant@durham.ac.uk}$\,\rangle$;
\end{itemize}
\textbf{References:}
 \penalty 100
\begin{itemize}
  \item Eur. Phys. J. C33, 1, 1986
\end{itemize}
\textbf{Run details:}
 \penalty 100
\begin{itemize}

  \item * Single- and double-diffractive, plus non-diffractive inelastic, events.
  \item $p\bar{p}$ collider, \ensuremath{\sqrt{s}} = 200 or 900 GeV.
  \item The trigger implementation for NSD events is the same as in, e.g., the UA5_1989 analysis. No further cuts are needed.\end{itemize}

\noindent This study comprises measurements of pseudorapidity distributions measured with the UA5 detector at 200 and 900 GeV center of momentum energy. There are distributions for non-single diffractive (NSD) events and also for the combination of single- and double-diffractive events. The NSD distributions are further studied for certain ranges of the events charged multiplicity.

\clearpage


\clearpage

\typeout{Handling analysis UA5_1989_S1926373}
\subsection[UA5\_1989\_S1926373]{UA5\_1989\_S1926373\,\cite{Ansorge:1988kn}}
\textbf{UA5 charged multiplicity measurements}\newline
\textbf{Experiment:} UA5 (CERN SPS) \newline
\textbf{Spires ID:} \href{http://www.slac.stanford.edu/spires/find/hep/www?rawcmd=key+1926373}{1926373}\newline
\textbf{Status:} VALIDATED\newline
\textbf{Authors:}
 \penalty 100
\begin{itemize}
  \item Holger Schulz $\langle\,$\href{mailto:holger.schulz@physik.hu-berlin.de}{holger.schulz@physik.hu-berlin.de}$\,\rangle$;
  \item Christophe L. J. Vaillant $\langle\,$\href{mailto:c.l.j.j.vaillant@durham.ac.uk}{c.l.j.j.vaillant@durham.ac.uk}$\,\rangle$;
  \item Andy Buckley $\langle\,$\href{mailto:andy.buckley@cern.ch}{andy.buckley@cern.ch}$\,\rangle$;
\end{itemize}
\textbf{References:}
 \penalty 100
\begin{itemize}
  \item Z. Phys. C - Particles and Fields 43, 357-374 (1989)
  \item DOI: \href{http://dx.doi.org/10.1007/BF01506531}{10.1007/BF01506531}
\end{itemize}
\textbf{Run details:}
 \penalty 100
\begin{itemize}

  \item Minimum bias events at \ensuremath{\sqrt{s}} = 200 and 900 GeV. Enable single and double diffractive events in addition to non-diffractive processes.\end{itemize}

\noindent Multiplicity distributions of charged particles produced in non-single-diffractive collisions between protons and antiprotons at centre-of-mass energies of 200 and 900 GeV. The data were recorded in the UA5 streamer chambers at the CERN collider, which was operated in a pulsed mode between the two energies. This analysis confirms the violation of KNO scaling in full phase space found by the UA5 group at an energy of 546 GeV, with similar measurements at 200 and 900 GeV.

\clearpage


\section{HERA analyses}\typeout{Handling analysis H1_1994_S2919893}
\subsection[H1\_1994\_S2919893]{H1\_1994\_S2919893\,\cite{Abt:1994ye}}
\textbf{H1 energy flow and charged particle spectra in DIS}\newline
\textbf{Experiment:} H1 (HERA) \newline
\textbf{Spires ID:} \href{http://www.slac.stanford.edu/spires/find/hep/www?rawcmd=key+2919893}{2919893}\newline
\textbf{Status:} VALIDATED\newline
\textbf{Authors:}
 \penalty 100
\begin{itemize}
  \item Peter Richardson $\langle\,$\href{mailto:peter.richardson@durham.ac.uk}{peter.richardson@durham.ac.uk}$\,\rangle$;
\end{itemize}
\textbf{References:}
 \penalty 100
\begin{itemize}
  \item Z.Phys.C63:377-390,1994
  \item DOI: \href{http://dx.doi.org/10.1007/BF01580319}{10.1007/BF01580319}
\end{itemize}
\textbf{Run details:}
 \penalty 100
\begin{itemize}

  \item $e^- p$ / $e^+ p$ deep inelastic scattering, 820~GeV protons colliding with 26.7~GeV electrons\end{itemize}

\noindent Global properties of the hadronic final state in deep inelastic scattering events at HERA are investigated. The data are corrected for detector effects. Energy flows in both the laboratory frame and the hadronic centre of mass system, and energy-energy correlations in the laboratory frame are presented.  Historically, the Ariadne colour dipole model provided the only satisfactory description of this data, hence making it a useful 'target' analysis for MC shower models.

\clearpage


\clearpage

\typeout{Handling analysis H1_2000_S4129130}
\subsection[H1\_2000\_S4129130]{H1\_2000\_S4129130\,\cite{Adloff:1999ws}}
\textbf{H1 energy flow in DIS}\newline
\textbf{Experiment:} H1 (HERA) \newline
\textbf{Spires ID:} \href{http://www.slac.stanford.edu/spires/find/hep/www?rawcmd=key+4129130}{4129130}\newline
\textbf{Status:} VALIDATED\newline
\textbf{Authors:}
 \penalty 100
\begin{itemize}
  \item Peter Richardson $\langle\,$\href{mailto:peter.richardson@durham.ac.uk}{peter.richardson@durham.ac.uk}$\,\rangle$;
\end{itemize}
\textbf{References:}
 \penalty 100
\begin{itemize}
  \item Eur.Phys.J.C12:595-607,2000
  \item DOI: \href{http://dx.doi.org/10.1007/s100520000287}{10.1007/s100520000287}
  \item arXiv: \href{http://arxiv.org/abs/hep-ex/9907027v1}{hep-ex/9907027v1}
\end{itemize}
\textbf{Run details:}
 \penalty 100
\begin{itemize}

  \item $e^+ p$ deep inelastic scattering with $p$ at 820 GeV, $e^+$ at 27.5 GeV \ensuremath{\to} \ensuremath{\sqrt{s}} = 300 GeV\end{itemize}

\noindent Measurements of transverse energy flow for neutral current deep- inelastic scattering events produced in positron-proton collisions at HERA. The kinematic range covers squared momentum transfers $Q^2$ from 3.2 to 2200 GeV$^2$; the Bjorken scaling variable $x$ from $8 \times 10^{-5}$ to 0.11 and the hadronic mass $W$ from 66 to 233 GeV. The transverse energy flow is measured in the hadronic centre of mass frame and is studied as a function of $Q^2$, $x$, $W$ and pseudorapidity. The behaviour of the mean transverse energy in the central pseudorapidity region and an interval corresponding to the photon fragmentation region are analysed as a function of $Q^2$ and $W$.  This analysis is useful for exploring the effect of photon PDFs and for tuning models of parton evolution and treatment of fragmentation and the proton remnant in DIS.

\clearpage


\section{RHIC analyses}\typeout{Handling analysis STAR_2006_S6500200}
\subsection[STAR\_2006\_S6500200]{STAR\_2006\_S6500200\,\cite{Adams:2006nd}}
\textbf{Identified hadron spectra in pp at 200 GeV}\newline
\textbf{Experiment:} STAR (RHIC pp 200 GeV) \newline
\textbf{Spires ID:} \href{http://www.slac.stanford.edu/spires/find/hep/www?rawcmd=key+6500200}{6500200}\newline
\textbf{Status:} VALIDATED\newline
\textbf{Authors:}
 \penalty 100
\begin{itemize}
  \item Bedanga Mohanty $\langle\,$\href{mailto:bedanga@rcf.bnl.gov}{bedanga@rcf.bnl.gov}$\,\rangle$;
  \item Hendrik Hoeth $\langle\,$\href{mailto:hendrik.hoeth@cern.ch}{hendrik.hoeth@cern.ch}$\,\rangle$;
\end{itemize}
\textbf{References:}
 \penalty 100
\begin{itemize}
  \item Phys. Lett. B637, 161
  \item nucl-ex/0601033
\end{itemize}
\textbf{Run details:}
 \penalty 100
\begin{itemize}

  \item pp at 200 GeV\end{itemize}

\noindent \pT distributions of charged pions and (anti)protons in pp collisions at $\sqrt{s} = 200$ GeV, measured by the STAR experiment at RHIC in non-single-diffractive minbias events.

\clearpage


\clearpage

\typeout{Handling analysis STAR_2006_S6860818}
\subsection[STAR\_2006\_S6860818]{STAR\_2006\_S6860818\,\cite{Abelev:2006cs}}
\textbf{Strange particle production in pp at 200 GeV}\newline
\textbf{Experiment:} STAR (RHIC pp 200 GeV) \newline
\textbf{Spires ID:} \href{http://www.slac.stanford.edu/spires/find/hep/www?rawcmd=key+6860818}{6860818}\newline
\textbf{Status:} VALIDATED\newline
\textbf{Authors:}
 \penalty 100
\begin{itemize}
  \item Hendrik Hoeth $\langle\,$\href{mailto:hendrik.hoeth@cern.ch}{hendrik.hoeth@cern.ch}$\,\rangle$;
\end{itemize}
\textbf{References:}
 \penalty 100
\begin{itemize}
  \item Phys. Rev. C75, 064901
  \item nucl-ex/0607033
\end{itemize}
\textbf{Run details:}
 \penalty 100
\begin{itemize}

  \item pp at 200 GeV\end{itemize}

\noindent \pT distributions of identified strange particles in pp collisions at $\sqrt{s} = 200$ GeV, measured by the STAR experiment at RHIC in non-single-diffractive minbias events. WARNING The $\langle \pT \rangle$ vs. particle mass plot is not validated yet and might be wrong.

\clearpage


\clearpage

\typeout{Handling analysis STAR_2006_S6870392}
\subsection[STAR\_2006\_S6870392]{STAR\_2006\_S6870392\,\cite{Abelev:2006uq}}
\textbf{Inclusive jet cross-section in pp at 200 GeV}\newline
\textbf{Experiment:} STAR (RHIC pp 200 GeV) \newline
\textbf{Spires ID:} \href{http://www.slac.stanford.edu/spires/find/hep/www?rawcmd=key+6870392}{6870392}\newline
\textbf{Status:} VALIDATED\newline
\textbf{Authors:}
 \penalty 100
\begin{itemize}
  \item Hendrik Hoeth $\langle\,$\href{mailto:hendrik.hoeth@cern.ch}{hendrik.hoeth@cern.ch}$\,\rangle$;
\end{itemize}
\textbf{References:}
 \penalty 100
\begin{itemize}
  \item Phys. Rev. Lett. 97, 252001
  \item hep-ex/0608030
\end{itemize}
\textbf{Run details:}
 \penalty 100
\begin{itemize}

  \item pp at 200 GeV\end{itemize}

\noindent Inclusive jet cross section as a function of \pT in pp collisions at $\sqrt{s} = 200$ GeV, measured by the STAR experiment at RHIC.

\clearpage


\section{Monte Carlo analyses}\typeout{Handling analysis MC_DIPHOTON}
\subsection{MC\_DIPHOTON}
\textbf{Monte Carlo validation observables for diphoton production at LHC}\newline
\textbf{Status:} VALIDATED\newline
\textbf{Authors:}
 \penalty 100
\begin{itemize}
  \item Frank Siegert $\langle\,$\href{mailto:frank.siegert@durham.ac.uk}{frank.siegert@durham.ac.uk}$\,\rangle$;
\end{itemize}
\textbf{No references listed}\\ 
\textbf{Run details:}
 \penalty 100
\begin{itemize}

  \item LHC pp \ensuremath{\to} jet+jet, photon+jet, photon+photon, all with EW+QCD shower\end{itemize}

\noindent Different observables related to the two photons

\clearpage


\clearpage

\typeout{Handling analysis MC_JETS}
\subsection{MC\_JETS}
\textbf{Monte Carlo validation observables for jet production}\newline
\textbf{Status:} VALIDATED\newline
\textbf{Authors:}
 \penalty 100
\begin{itemize}
  \item Frank Siegert $\langle\,$\href{mailto:frank.siegert@durham.ac.uk}{frank.siegert@durham.ac.uk}$\,\rangle$;
\end{itemize}
\textbf{No references listed}\\ 
\textbf{Run details:}
 \penalty 100
\begin{itemize}

  \item Pure QCD jet production events at an arbitrary collider.\end{itemize}

\noindent Jets with $p_\perp>20$ GeV are constructed with a $k_\perp$ jet finder with $D=0.7$ and projected onto many different observables.

\clearpage


\clearpage

\typeout{Handling analysis MC_LEADINGJETS}
\subsection{MC\_LEADINGJETS}
\textbf{Underlying event in leading jet events, extended to LHC}\newline
\textbf{Status:} VALIDATED\newline
\textbf{Authors:}
 \penalty 100
\begin{itemize}
  \item Andy Buckley $\langle\,$\href{mailto:andy.buckley@cern.ch}{andy.buckley@cern.ch}$\,\rangle$;
\end{itemize}
\textbf{No references listed}\\ 
\textbf{Run details:}
 \penalty 100
\begin{itemize}

  \item LHC pp QCD interactions at 0.9, 10 or 14 TeV. Particles with $c \tau > 10$ mm should be set stable. Several $p_\perp^\text{min}$ cutoffs are probably required to fill the profile histograms.\end{itemize}

\noindent Rick Field's measurement of the underlying event in leading jet events, extended to the LHC. As usual, the leading jet of the defines an azimuthal toward/transverse/away decomposition, in this case the event is accepted within $|\eta| < 2$, as in the CDF 2008 version of the analysis. Since this isn't the Tevatron, I've chosen to use $k_\perp$ rather than midpoint jets.

\clearpage


\clearpage

\typeout{Handling analysis MC_PHOTONJETS}
\subsection{MC\_PHOTONJETS}
\textbf{Monte Carlo validation observables for photon + jets production}\newline
\textbf{Status:} VALIDATED\newline
\textbf{Authors:}
 \penalty 100
\begin{itemize}
  \item Frank Siegert $\langle\,$\href{mailto:frank.siegert@durham.ac.uk}{frank.siegert@durham.ac.uk}$\,\rangle$;
\end{itemize}
\textbf{No references listed}\\ 
\textbf{Run details:}
 \penalty 100
\begin{itemize}

  \item Tevatron Run II ppbar \ensuremath{\to} gamma + jets.\end{itemize}

\noindent Different observables related to the photon and extra jets.

\clearpage


\clearpage

\typeout{Handling analysis MC_SUSY}
\subsection{MC\_SUSY}
\textbf{Validate generic SUSY events, including various lepton invariant mass}\newline
\textbf{Status:} VALIDATED\newline
\textbf{Authors:}
 \penalty 100
\begin{itemize}
  \item Andy Buckley $\langle\,$\href{mailto:andy.buckley@cern.ch}{andy.buckley@cern.ch}$\,\rangle$;
\end{itemize}
\textbf{No references listed}\\ 
\textbf{Run details:}
 \penalty 100
\begin{itemize}

  \item SUSY events at any energy. \pT cutoff at 10 GeV may be advised.\end{itemize}

\noindent Analysis of generic SUSY events at the LHC, based on Atlas Herwig++ validation analysis contents. Plotted are eta, phi and \pT observables for charged tracks, photons, isolated photons, electrons, muons, and jets, as well as various dilepton mass `edge' plots for different event selection criteria.

\clearpage


\clearpage

\typeout{Handling analysis MC_WJETS}
\subsection{MC\_WJETS}
\textbf{Monte Carlo validation observables for $W[e \, \nu]$ + jets production}\newline
\textbf{Status:} VALIDATED\newline
\textbf{Authors:}
 \penalty 100
\begin{itemize}
  \item Frank Siegert $\langle\,$\href{mailto:frank.siegert@durham.ac.uk}{frank.siegert@durham.ac.uk}$\,\rangle$;
\end{itemize}
\textbf{No references listed}\\ 
\textbf{Run details:}
 \penalty 100
\begin{itemize}

  \item $e \, \nu$ + jets analysis.\end{itemize}

\noindent Available observables are W mass, \pT of jets 1-4, jet multiplicity, $\Delta\eta(W, \text{jet1})$, $\Delta R(\text{jet2}, \text{jet3})$, differential jet rates 0\ensuremath{\to}1, 1\ensuremath{\to}2, 2\ensuremath{\to}3, 3\ensuremath{\to}4, integrated 0--4 jet rates.

\clearpage


\clearpage

\typeout{Handling analysis MC_ZJETS}
\subsection{MC\_ZJETS}
\textbf{Monte Carlo validation observables for $Z[e^+ \, e^-]$ + jets production}\newline
\textbf{Status:} VALIDATED\newline
\textbf{Authors:}
 \penalty 100
\begin{itemize}
  \item Frank Siegert $\langle\,$\href{mailto:frank.siegert@durham.ac.uk}{frank.siegert@durham.ac.uk}$\,\rangle$;
\end{itemize}
\textbf{No references listed}\\ 
\textbf{Run details:}
 \penalty 100
\begin{itemize}

  \item $e^+ e^-$ + jets analysis. Needs mass cut on lepton pair to avoid photon singularity, e.g. a min range of $66 < m_{ee} < 116$ GeV\end{itemize}

\noindent Available observables are Z mass, \pT of jets 1-4, jet multiplicity, $\Delta\eta(Z, \text{jet1})$, $\Delta R(\text{jet2}, \text{jet3})$, differential jet rates 0\ensuremath{\to}1, 1\ensuremath{\to}2, 2\ensuremath{\to}3, 3\ensuremath{\to}4, integrated 0--4 jet rates.

\clearpage


\section{Example analyses}\typeout{Handling analysis EXAMPLE}
\subsection{EXAMPLE}
\textbf{A demo to show aspects of writing a Rivet analysis}\newline
\textbf{Status:} EXAMPLE\newline
\textbf{Authors:}
 \penalty 100
\begin{itemize}
  \item Andy Buckley $\langle\,$\href{mailto:andy.buckley@durham.ac.uk}{andy.buckley@durham.ac.uk}$\,\rangle$;
\end{itemize}
\textbf{No references listed}\\ 
\textbf{Run details:}
 \penalty 100
\begin{itemize}

  \item All event types will be accepted.\end{itemize}

\noindent This analysis is a demonstration of the Rivet analysis structure and functionality: booking histograms; the initialisation, analysis and finalisation phases; and a simple loop over event particles. It has no physical meaning, but can be used as a simple pedagogical template for writing real analyses.

\clearpage


\section{Misc. analyses}\typeout{Handling analysis BELLE_2006_S6265367}
\subsection{BELLE\_2006\_S6265367}
\textbf{Charm hadrons from fragmentation and B decays on the $\Upsilon(4S)$}\newline
\textbf{Status:} VALIDATED\newline
\textbf{Authors:}
 \penalty 100
\begin{itemize}
  \item Jan Eike von Seggern $\langle\,$\href{mailto:jan.eike.von.seggern@physik.hu-berlin.de}{jan.eike.von.seggern@physik.hu-berlin.de}$\,\rangle$;
\end{itemize}
\textbf{References:}
 \penalty 100
\begin{itemize}
  \item Phys.Rev.D73:032002,2006.
  \item arXiv: \href{http://arxiv.org/abs/hep-ex/0506068}{hep-ex/0506068}
  \item DOI: \href{http://dx.doi.org/10.1103/PhysRevD.73.032002}{10.1103/PhysRevD.73.032002}
\end{itemize}
\textbf{Run details:}
 \penalty 100
\begin{itemize}

  \item $e^+ e^-$ analysis on the $\Upsilon(4S)$ resonance, with CoM boost -- 8.0~GeV~($e^−$) and 3.5~GeV~($e^+$)\end{itemize}

\noindent Analysis of charm quark fragmentation at 10.6 GeV, based on a data sample of 103 fb collected by the Belle detector at the KEKB accelerator. Fragmentation into charm is studied for the main charmed hadron ground states, namely $D^0$, $D^+$, $D^+_s$ and $\Lambda_c^+$, as well as the excited states $D^{*0}$ and $D^{*+}$. This analysis can be used to constrain charm fragmentation in Monte Carlo generators. Additionally, we determine the average number of these charmed hadrons produced per B decay at the $\Upsilon(4S)$ resonance and measure the distribution of their production angle in $e^+ e^-$ annihilation events and in B decays.

\clearpage


\clearpage

\typeout{Handling analysis PDG_HADRON_MULTIPLICITIES}
\subsection[PDG\_HADRON\_MULTIPLICITIES]{PDG\_HADRON\_MULTIPLICITIES\,\cite{Amsler:2008zzb}}
\textbf{Hadron multiplicities in hadronic $e^+e^-$ events}\newline
\textbf{Experiment:} PDG (Various) \newline
\textbf{Spires ID:} \href{http://www.slac.stanford.edu/spires/find/hep/www?rawcmd=key+7857373}{7857373}\newline
\textbf{Status:} VALIDATED\newline
\textbf{Authors:}
 \penalty 100
\begin{itemize}
  \item Hendrik Hoeth $\langle\,$\href{mailto:hendrik.hoeth@cern.ch}{hendrik.hoeth@cern.ch}$\,\rangle$;
\end{itemize}
\textbf{References:}
 \penalty 100
\begin{itemize}
  \item Phys. Lett. B, 667, 1 (2008)
\end{itemize}
\textbf{Run details:}
 \penalty 100
\begin{itemize}

  \item Hadronic events in $e^+ e^-$ collisions\end{itemize}

\noindent Hadron multiplicities in hadronic $e^+e^-$ events, taken from Review of Particle Properties 2008, table 40.1, page 355.   Average hadron multiplicities per hadronic $e^+e^-$ annihilation event at $\sqrt{s} \approx {}$ 10, 29--35, 91, and 130--200 GeV. The numbers are averages from various experiments. Correlations of the systematic uncertainties were considered for the calculation of the averages.

\clearpage


\clearpage

\typeout{Handling analysis PDG_HADRON_MULTIPLICITIES_RATIOS}
\subsection[PDG\_HADRON\_MULTIPLICITIES\_RATIOS]{PDG\_HADRON\_MULTIPLICITIES\_RATIOS\,\cite{Amsler:2008zzb}}
\textbf{Ratios (w.r.t. $\pi^+/\pi^-$) of hadron multiplicities in hadronic $e^+e^-$ events}\newline
\textbf{Experiment:} PDG (Various) \newline
\textbf{Spires ID:} \href{http://www.slac.stanford.edu/spires/find/hep/www?rawcmd=key+7857373}{7857373}\newline
\textbf{Status:} VALIDATED\newline
\textbf{Authors:}
 \penalty 100
\begin{itemize}
  \item Holger Schulz $\langle\,$\href{mailto:holger.schulz@physik.hu-berlin.de}{holger.schulz@physik.hu-berlin.de}$\,\rangle$;
\end{itemize}
\textbf{References:}
 \penalty 100
\begin{itemize}
  \item Phys. Lett. B, 667, 1 (2008)
\end{itemize}
\textbf{Run details:}
 \penalty 100
\begin{itemize}

  \item Hadronic events in $e^+ e^-$ collisions\end{itemize}

\noindent Ratios (w.r.t. $\pi^+/\pi^-$) of hadron multiplicities in hadronic $e^+ e^-$ events, taken from Review of Particle Properties 2008, table 40.1, page 355.  Average hadron multiplicities per hadronic $e^+ e^-$ annihilation event at $\sqrt{s} \approx$ 10, 29--35, 91, and 130--200 GeV, normalised to the pion multiplicity. The numbers are averages from various experiments. Correlations of the systematic uncertainties were considered for the calculation of the averages.

\clearpage






\cleardoublepage
\part{How Rivet works}
\label{part:writinganalyses}
%\label{part:internals}

Hopefully by now you've run Rivet a few times and got the hang of the command
line interface and viewing the resulting analysis data files. Maybe you've got
some ideas of analyses that you would like to see in Rivet's library. If so,
then you'll need to know a little about Rivet's internal workings before you can
start coding: with any luck by the end of this section that won't seem
particularly intimidating.

The core objects in Rivet are ``projections'' and ``analyses''. Hopefully
``analyses'' isn't a surprise --- that's just the collection of routines that
will make histograms to compare with reference data, and the only things that
might differ there from experieces with HZTool are the new histogramming system
and the fact that we've used some object orientation concepts to make life a bit
easier. The meaning of ``projections'', as applied to event analysis, will
probably be less obvious. We'll discuss them now.


\section{Projections}

The name ``projection'' is meant to evoke thoughts of projection operators,
low-dimensional slices/views of high-dimensional spaces, and other things that
might appeal to physicists who view the world through quantum-tinted lenses. A
more mundane, but equally applicable, name would be ``observable calculators'',
but since that's a long name, the things they return aren't \emph{necessarily}
observable, and they all inherit from the \kbd{Projection} base class, we'll
stick to that name. It doesn't take long to get used to using the name as a
synonym for ``calculator'', without being intimidated by ideas that they might
be some sort of high-powered deep magic. 90\% of them is simple and
self-explanatory, as a peek under the bonnet of e.g. the all-important
\kbd{FinalState} projection will reveal.

Projections can be relatively simple things like event shapes (i.e. scalar,
vector or tensor quantities), or arbitrarily complex things like lossy or
selective views of the event final state. Most users will see them attached to
analyses by declarations in each analysis' constructor, but they can also be
recursively ``nested'' inside other projections\footnote{Provided there are no
  dependency loops in the projection chains! Strictly, only acyclic graphs of
  projection dependencies are valid, but there is currently no code in Rivet
  that will attempt to verify this restriction.} (provided there are no infinite
loops in the nesting chain.) Calling a complex projection in an analysis may
actually transparently execute many projections on each event.


\subsection{Projection caching}

Aside from semantic issues of how the class design assigns the process of
analysing events, projections are important computationally because they live in
a framework which automatically stores (``caches'') their results between
events. This is a crucial feature for the long-term scalability of Rivet, as the
previous experience with HZTool was that HERA validation code ran very slowly
due to repeated calculation of the same $k_\perp$ clustering algorithm (at that
time notorious for scaling as the 3rd power of the number of particles.)

A concrete example may help in understanding how this works. Let's say we have
two analyses which have the same run conditions, i.e. incoming beam types, beam
energies, etc. Each also uses the thrust event shape measure to define a set of
basis vectors for their analysis. For each event that gets passed to Rivet,
whichever analysis gets called first will immediately (although maybe
indirectly) call a \kbd{FinalState} projection to get a list of stable, physical
particles (filtering out the intermediate and book-keeping entries in the HepMC
event record). That FS projection is then ``attached'' to the event. Next, the
first analysis will call a \kbd{Thrust} projection which internally uses the
same final state projection to define the momentum vectors used in calculating
the thrust. Once finished, the thrust projection will also be attached to the
event.

So far, projections have offered no benefits. However, when the second analysis
runs it will similarly try to apply its final state and thrust projections to
the event. Rather than repeat the calculations, Rivet's infrastructure will
detect that an equivalent calculation has already been run and will just return
references to the already-run projections. Since projections can also contain
and use other projections, this model allows some substantial computational
savings, without the analysis author even needing to be particularly aware of
what is going on.

Observant readers may have noticed a problem with all this projection caching
cleverness: what if the final states aren't defined the same way? One might
provide charged final state particles only, or the acceptances (defined in
rapidity range and a IR \pT cutoff) might differ. Rivet handles this by
making each projection provide a comparison operator which is used to decide
whether the cached version is acceptable or if the calculation must be re-run
with different settings. Because projections can be nested, applying a top-level
projection to an event can spark off a cascade of comparisons, calculations and
cache accesses, making use of existing results wherever possible.


\subsection{Using projection caching}
So far this is all theory --- how does one actually use projections in Rivet?
First, you should understand that projections, while semantically stored within
each other, are actually all registered with a central \code{ProjectionHandler}
object.\footnote{As of version 1.1 onwards --- previously, they were stored as
  class members inside other \code{Projection}s and \code{Analysis} classes.}
The reason for this central registration is to ensure that all projections'
lifespans are managed in a consistent way, and to protect projection and
analysis authors from some technical subtleties in how C++ polymorphism works.

Inside the constructor of a \code{Projection} or \code{Analysis} class, you must
call the \code{addProjection} function. This takes two arguments, the projection
to be registered (by \code{const} reference), and a name. The name is local to
the parent object, so you need not worry about name clashes between objects. A
very important point is that the passed \code{Projection} is not the one that is
actually centrally registered --- that distinction belongs to a newly created
heap object which is created within the \code{addProjection} method by means of
the overloaded \code{Projection::clone()} method. Hence it is completely safe
--- and recommended --- to use only local (stack) objects in \code{Projection}
and \code{Analysis} constructors.


\begin{philosophy}
  At this point, if you have rightly bought into C++ ideas like super-strong
  type-safety, this proliferation of dynamic casting may worry you: the compiler
  can't possibly check if a projection of the requested name has been
  registered, nor whether the downcast to the requested concrete type is
  legal. These are very legitimate concerns!

  In truth, we'd \emph{like} to have this level of extra safety but in the past,
  when projections were held as members of \code{ProjectionApplier} classes
  rather than in the central \code{ProjectionHandler} repository, the benefits
  of the strong typing were outweighed by more serious and subtle bugs relating
  to projection lifetime and object ``slicing''. At least when the current
  approach goes wrong it will throw an unmissable \emph{runtime} error every
  time that you run it (until it's fixed, of course!) rather than silently do
  the wrong thing, as was the previous behaviour.

  Our problems here are a microcosm of the perpetual language battle between
  strict and dynamic typing, runtime versus compile time errors. In practice,
  this manifests itself as a trade-off between the benefits of static type
  safety and the inconvenience of the type-system gymnastics that it engenders.
  We take some comfort from the number of very good programs have been and are
  still written in dynamically typed, interpreted languages like Python, where
  virtually all error checking (barring first-scan parsing errors) must be done
  at runtime. By pushing \emph{some} checking to the domain of runtime errors,
  Rivet's code is (we believe) in practice safer, and certainly more clear and
  elegant. However, we believe that with runtime checking should come a culture
  of unit testing, which is not yet in place in Rivet.

  As a final thought, one reason for Rivet's internal complexity is that C++ is
  just not a very good language for this sort of thing: we are operating on the
  boundary between event generator codes, number crunching routines (including
  third party libraries like FastJet) and user routines. The former set
  unavoidably require native interfaces and benefit from static typing; the
  latter benefit from interface flexibility, fast prototyping and syntactic
  clarity. Maybe a future version of Rivet will break through the technical
  barriers to a hybrid approach and allow users to run compiled projections from
  interpreted analysis code. For now, however, we hope that our brand of
  ``slightly less safe C++'' will be a pleasant compromise.
\end{philosophy}


% \begin{detail}
% \TODO{How projection caching \emph{really} works}
% (skippable, but useful as a reference)
% \end{detail}

% \subsection{Standard projection summary}
% \TODO{foo}

% \subsection{Example projection}
% \TODO{bar}

% \subsection{Cuts and constraints}
% \TODO{baz}


\section{Analyses}

\subsection{Writing a new analysis}

This section provides a recipe that can be followed to write a new analysis
using the Rivet projections.

Every analysis must inherit from \code{Rivet::Analysis} and, in addition to the
constructor, must implement a minimum of three methods.  Those methods are
\code{init()}, \code{analyze(const Rivet::Event\&)} and \code{finalize()}, which
are called once at the beginning of the analysis, once per event and once at the
end of the analysis respectively.

The new analysis should include the header for the base analysis class plus
whichever Rivet projections are to be used and should work under the
\code{Rivet} namespace.  The header for a new analysis named \code{UserAnalysis}
that uses the \code{FinalState} projection might therefore start off looking
like this:
%
\begin{snippet}
#include "Rivet/Analysis.hh"

namespace Rivet {

  class UserAnalysis : public Analysis {
  public:
    UserAnalysis();
    void init();
    void analyze(const Event& event);
    void finalize();
  };

}
\end{snippet}

\subsubsection{Analysis constructor}
The constructor for the \code{UserAnalysis} class should add to the analysis all
of the projections that will be used.  Projections can be added to an analysis
with a call to \code{addProjection(Projection, std::string)}, which takes as
argument the projection to be added and a name by which that projection can
later be referenced.  For this example the \code{FinalState} projection is to be
referenced by the string \code{"FS"} to provide access to all of the final state
particles inside a detector pseudorapidity coverage of $\pm 5.0$.  The syntax to
create and add that projection inside the constructor for \code{UserAnalysis} is
as follows:
%
\begin{snippet}
Rivet::UserAnalysis() {
  const FinalState fs(-5.0, 5.0);
  addProjection(fs, "FS");
}
\end{snippet}

In addition to adding projections, the constructor may also impose certain
requirements upon the events that the analysis will work with.  A call to the
\code{setBeams} method declares that the analysis may only be run on events with
specific types of beam particles, for example adding the line
%
\begin{snippet}
  setBeams(PROTON, PROTON);
\end{snippet}
%
\noindent ensures that the analysis can only be run on events from proton-proton
collisions.  Other types of beam particles that may be used include
\code{ANTIPROTON}, \code{ELECTRON}, \code{POSITRON}, \code{MUON} and \code{ALL}.
The later of these declares that the analysis is suitable for use with any type
of collision and is the default.

Some analyses need to know the interaction cross section that was generated by
the Monte Carlo generator, typically in order to normalise histograms.
Depending on the Monte Carlo that is used and its interface to Rivet, the cross
section may or may not be known.  An analysis can therefore declare at the
beginning of a run that it will need the cross section information during the
finalisation stages.  Such a declaration can be used to prevent what would
otherwise be fruitless analyses from running.  An analysis sets itself as
requiring the cross section by calling inside the constructor
%
\begin{snippet}
  setNeedsCrossSection(true);
\end{snippet}
%
\noindent In the absence of this call the default is to assume that the analysis
does not need to know the cross section.

% It is often the case that an analysis is only appropriate for a limited range of
% Monte Carlo kinematic settings.  For example, an analysis may only be suitable
% if the minimum \pT in the hard scatter is above a certain value.  A mechanism
% exists with the \code{Analysis} object to declare the existence of such a cut.
% However, the relevant information must be made available by the Monte Carlo
% generator and, given the potentially large number of such generator-dependent
% cuts, this information is not currently checked by the AGILe generator
% interface.  Nevertheless, an analysis can add a cut in the constructor with a
% call to \code{addCut(const string\&, const Comparison\&, const double)}, which
% takes as its arguments the name of the cut, the comparison operator and a value
% to compare the cut quantity with.  For example, to add a cut stating that the
% hard scatter \pT must be above \unit{3}{\GeV} one should call

% \begin{snippet}
%   addCut("PT", MORE_EQ, 3.0*GeV);
% \end{snippet}



\subsection{Histogramming}
\TODO{Histo interfaces, AIDA and YODA. Formats.}

\subsection{Analysis histo autobinning}
\subsection{Pluggable analyses}
\subsection{Example analysis}

\section{Viewing output data files}

\section{Comparing with reference data}


\cleardoublepage
\part{Appendices}
\appendix


\section{Typical \kbd{agile-runmc} commands}
\label{app:agilerunmc}

\begin{itemize}
\item \paragraph{Simple run:}{\kbd{agile-runmc Herwig:6510 -P~lep1.params --beams=LEP:91.2
      -n~1000} will use the Fortran Herwig 6.5.10 generator (the \kbd{-g} option
    switch) to generate 1000 events (the \kbd{-n} switch) in LEP1 mode,
    i.e. $\Ppositron\Pelectron$ collisions at $\sqrt{s} = \unit{91.2}{\GeV}$.}
  
\item \paragraph{Parameter changes:}{\kbd{agile-runmc Pythia6:418
      --beams=LEP:91.2 -n~1000 \cmdbreak -P~myrun.params -p~"PARJ(82)=5.27"}
    will generate 1000 events using the Fortran Pythia 6.4.18 generator, again
    in LEP1 mode. The \kbd{-P} switch is actually the way of specifying a
    parameters file, with one parameter per line in the format ``\val{key}
    \val{value}'': in this case, the file \kbd{lep1.params} is loaded from the
    \kbd{\val{installdir}/share/AGILe} directory, if it isn't first found in the
    current directory.  The \kbd{-p} (lower-case) switch is used to change a
    named generator parameter, here Pythia's \kbd{PARJ(82)}, which sets the
    parton shower cutoff scale. Being able to change parameters on the command
    line is useful for scanning parameter ranges from a shell loop, or rapid
    testing of parameter values without needing to write a parameters file for
    use with~\kbd{-P}.}
  
\item \paragraph{Writing out HepMC events:}{\kbd{agile-runmc Pythia6:418
      --beams=LHC:14TeV -n~50 -o~out.hepmc -R} will generate 50 LHC events with
    Pythia. The~\kbd{-o} switch is being used here to tell \kbd{agile-runmc} to
    write the generated events to the \kbd{out.hepmc} file. This file will be a
    plain text dump of the HepMC event records in the standard HepMC format. Use
    of filename ``-'' will result in the event stream being written to standard
    output (i.e. dumping to the terminal.}
\end{itemize}


\cleardoublepage
\part{Bibliography}
\bibliographystyle{h-physrev3}
{\raggedright
  \bibliography{refs}
}


\end{document}
